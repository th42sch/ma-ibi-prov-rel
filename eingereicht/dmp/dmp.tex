% !TeX spellcheck = de_DE
%\documentclass[11pt,DIV=11,a4paper,BCOR=15mm,twoside=on,bibliography=leveldown]{scrbook}
%\documentclass[11pt,DIV=11,a4paper,BCOR=15mm]{scrbook}
\documentclass[%
  11pt,
  DIV=16,
  a4paper,
  BCOR=15mm,
  twoside=on,
  bibliography=totoc,
  headings=normal,
%  titlepage,
  numbers=noendperiod,
%  captions=tableheading,
%  chapterprefix=true% like in standard class "report"
]{scrartcl}
%\documentclass[11pt,DIV=11,a4paper]{scrartcl}
\usepackage{dmp}

\selectlanguage{ngerman}
\title{Datenmanagementplan}
\subtitle{zur Masterarbeit \enquote{Modelling and Automated Retrieval of Provenance Relationships}}
\author{Thomas Schneider}

\begin{document}

\selectlanguage{ngerman}
\maketitle

% ------------------------------------------------------------------
\section{Allgemein}

% - - - - - - - - - - - - - - - - - - - - - - - - - - - - - - - - -
\subsection{Thema}

% ..................................................................
\subsubsection{Wie lautet die primäre Forschungsfrage der Abschlussarbeit?}

Wie können Provenienzbeziehungen modelliert und maschinell gestützt aufgefunden werden?

% ..................................................................
\subsubsection{Bitte geben Sie einige Schlagwörter zur Forschungsfrage bzw. Fragestellung an.}

\begin{itemize}
  \item 
    DDC:\footnote{\url{https://deweysearchde.pansoft.de/webdeweysearch/mainClasses.html?catalogs=DNB}}
    %
%    \begin{description}
%      \item[005.72]
%        Datenaufbereitung und Datenrepräsentation
%      \item[006.332]
%        Wissensrepräsentation
%      \item[020.0113]
%        Computermodellierung in Bibliotheks- \& Informationswissenschaften
%    \end{description}  
    \begin{itemize}
      \item
        005.72~
        Datenaufbereitung und Datenrepräsentation
      \item
        006.332~
        Wissensrepräsentation
      \item
        020.0113~
        Computermodellierung in Bibliotheks- \& Informationswissenschaften
    \end{itemize}  
%  \item 
%    GND:\footnote{\url{https://gnd.network/Webs/gnd/DE/Home/home_node.html}} *********
  \item 
    2012 ACM Computing Classification System:\footnote{\url{https://dl.acm.org/ccs}}
%    
%    \begin{CCSXML}
%    <ccs2012>
%    <concept>
%    <concept_id>10002951.10003317.10003347.10003348</concept_id>
%    <concept_desc>Information systems~Question answering</concept_desc>
%    <concept_significance>500</concept_significance>
%    </concept>
%    <concept>
%    <concept_id>10002951.10003317.10003347.10003352</concept_id>
%    <concept_desc>Information systems~Information extraction</concept_desc>
%    <concept_significance>300</concept_significance>
%    </concept>
%    </ccs2012>
%    \end{CCSXML}
%    
%    \ccsdesc[500]{Information systems~Question answering}
%    \ccsdesc[300]{Information systems~Information extraction}
%    
    \begin{itemize}
      \item 
        Information systems / Information retrieval / Retrieval tasks and goals / Question answering
      \item 
        Information systems / Information retrieval / Retrieval tasks and goals / Information extraction
    \end{itemize}
\end{itemize}

% ..................................................................
\subsubsection{Welchen Regeln oder Richtlinien (HU) zum Umgang mit den in der Abschlussarbeit erhobenen Forschungsdaten folgen Sie für den DMP? Bitte referenzieren Sie diese hier inklusive Version bzw. Veröffentlichungsjahr.}

Institut für Bibliotheks- und Informationswissenschaft:
Leitlinie zum Umgang mit Forschungsdaten in Abschlussarbeiten.
Beschlossen im Institutsrat des IBI am 08.12.2021, in Kraft getreten am 01.02.2022.%
\footnote{\url{https://www.ibi.hu-berlin.de/de/studium/rundumdasstudium/fdm-fuer-studierende}}

% ------------------------------------------------------------------
\section{Inhaltliche Einordnung}

\mybold{NB: Bitte beschreiben Sie jeden Datensatztyp oder Datensammlung einzeln in dem jeweiligen Kapitel, wo sinnvoll.}

% - - - - - - - - - - - - - - - - - - - - - - - - - - - - - - - - -
\subsection{Datensatz}

% ..................................................................
\subsubsection{Um welche Arten von Daten handelt es sich? Bitte in wenigen Zeilen kurz beschreiben.}

Für die Literaturstudie und Analyse der Datenquellen wurden folgende Daten gesammelt:
%
\begin{enumerate}[(1)]
  \item
    bibliographische Metadaten zu relevanten Arbeiten aus der Literatur und Webseiten
  \item
    für ausgewählte Arbeiten: Volltexte im PDF-Format
  \item
    statistische Angaben zu Datenquellen, z.\,B. die Anzahl der darin enthaltenen Datensätze
\end{enumerate}

% - - - - - - - - - - - - - - - - - - - - - - - - - - - - - - - - -
\subsection{Datenursprung}

% ..................................................................
\subsubsection{Werden die Daten selbst erzeugt oder nachgenutzt?}

\begin{enumerate}[(1)]
  \item
    teilweise nachgenutzt aus Portalen, teilweise selbst erzeugt
  \item
    heruntergeladen von Verlags- und Aggregatorplattformen
  \item
    aus der Literatur und von Webseiten selbst extrahiert
\end{enumerate}

% ..................................................................
\subsubsection{Wenn die Daten nachgenutzt werden, wer hat die Daten erzeugt? Bitte mit Angabe des Identifiers, falls vorhanden, z.B. DOI.}

Volltexte und deren Metadaten wurden von diversen Portalen heruntergeladen, darunter:
%
\begin{itemize}
  \item
    \url{https://hu-berlin.hosted.exlibrisgroup.com/primo-explore/search?vid=hub_ub}
  \item
    \url{https://www.sciencedirect.com/}
  \item
    \url{https://ebookcentral.proquest.com/lib/huberlin-ebooks/home.action}
  \item
    \url{https://www.tandfonline.com/}
  \item
    \url{https://dblp.uni-trier.de/}
  \item
    \url{https://gvk.k10plus.de}
\end{itemize}

% - - - - - - - - - - - - - - - - - - - - - - - - - - - - - - - - -
\subsection{Reproduzierbarkeit}

% ..................................................................
\subsubsection{Sind die Daten reproduzierbar, d. h. ließen sie sich, wenn sie verloren gingen, erneut erstellen oder erheben?}

Die Daten sind nur bedingt reproduzierbar:
Anhand des Literaturverzeichnisses der Arbeit lassen sich alle Quellen wiederfinden,
aber bei den meisten Webseiten besteht die Gefahr, dass der Inhalt sich ändert oder
die Webseite inaktiv wird. Letzteres kann auch bei denjenigen der wissenschaftlichen
Arbeiten passieren, die nur online verfügbar sind.

% - - - - - - - - - - - - - - - - - - - - - - - - - - - - - - - - -
\subsection{Nachnutzung}

% ..................................................................
\subsubsection{Für welche Personen, Gruppen oder Institutionen könnte dieser Datensatz (für die Nachnutzung) von Interesse sein? Für welche Szenarien ist dies denkbar?}

Der Datensatz könnte von Interesse sein für Forschende, die sich mit derselben Thematik befassen
und einige Aspekte aus oder neben dieser Arbeit vertiefen möchten.
Er könnte auch als (unvollständige) Basis für einen Übersichtsartikel über bibliographische Datenquellen
oder digitale Provenienzforschung dienen.

% ------------------------------------------------------------------
\section{Technische Einordnung}

% - - - - - - - - - - - - - - - - - - - - - - - - - - - - - - - - -
\subsection{Datenerhebung}

% ..................................................................
\subsubsection{Wann erfolgt(e) die Erhebung bzw. Erstellung der Daten?}

Die Sammlung der Daten erfolgte über gesamten Bearbeitungszeitraum der Masterarbeit hinweg, d.\,h.\ vom 16.02.2023 bis 14.06.2023.

% ..................................................................
\subsubsection{Wann erfolgt(e) die Datenbereinigung/-aufbereitung bzw. Datenanalyse?}

\begin{enumerate}[(1)]
  \item
    Bei jedem Fund einer wissenschaftlichen Arbeit wurden die bibliographischen Metadaten von der entsprechenden Plattform heruntergeladen
    und eine Masterdatei importiert (bzw.\ von Hand eingetragen, falls nötig). Die Daten wurden bei jedem Import sofort geprüft und ggf.\
    korrigiert. Eine weitere Datenbereinigung oder -aufbereitung ist nicht erforderlich.
  \item
    Die elektronisch vorliegenden wissenschaftlichen Arbeiten, von denen abzusehen war, dass sie im Bearbeitungszeitraum der Masterarbeit
    länger oder mehrfach konsultiert werden mussten, wurden als PDF-Datei gespeichert.
    Eine Datenbereinigung oder -aufbereitung scheint hier nicht sinnvoll.
  \item
    Wie (1); hier wurden alle Metadaten von Hand eingegeben.
  \item
    Die statistischen Daten der Datenquellen wurden am 09.06.2023 in einer Datei gebündelt.
\end{enumerate}

% - - - - - - - - - - - - - - - - - - - - - - - - - - - - - - - - -
\subsection{Datengröße}

% ..................................................................
\subsubsection{Was ist die tatsächliche oder erwartete Größe der Daten(typen)?}

Stand 09.06.2023:
%
\begin{enumerate}[(1)]
  \item
    eine Datei zu 115 kB
  \item
    ca.\ 150 Dateien zu insgesamt 412,8 MB
  \item
    eine Datei zu 1,4 kB
\end{enumerate}

\goodbreak
% - - - - - - - - - - - - - - - - - - - - - - - - - - - - - - - - -
\subsection{Formate}

% ..................................................................
\subsubsection{In welchen Formaten liegen die Daten vor?}
\label{sec:3.3.1}
%4 Vgl. z. B. DROID zur Format-Erkennung, http://digital-preservation.github.io/droid/

\begin{enumerate}[(1)]
  \item
    .bib (BibTeX)
  \item
    .pdf (Portable Document Format)
  \item
    .md (Markdown)
\end{enumerate}

% - - - - - - - - - - - - - - - - - - - - - - - - - - - - - - - - -
\subsection{Werkzeuge}

% ..................................................................
\subsubsection{Welche Instrumente, Software, Technologien oder Verfahren werden zur Erzeugung, Erfassung, Bereinigung, Analyse und/oder Visualisierung der Daten genutzt? Bitte (falls möglich) mit Versionsnummer und Referenz in Form einer Adresse jeweils angeben.}

\begin{enumerate}[(1)]
  \item
    \emph{Download:} \\
    Firefox 113.0.2 (64-bit)~
    \url{https://www.mozilla.org/en-GB/firefox/new/}
    
    \emph{Verwaltung:} \\
    JabRef 5.9--2023-01-08--76253f1a7~ 
    \url{https://www.jabref.org/}
  \item
    \emph{Download:}~ wie (1)
    
    \emph{Betrachtung:} \\
    Skim 1.6.16 (146)~
    \url{https://skim-app.sourceforge.io/}\\
    Adobe Acrobat Reader 2023.001.20177~
    \url{https://www.adobe.com/}
  \item
    MacDown 0.7.3 (1008.4)~
    \url{https://macdown.uranusjr.com/}\\
    Obsidian 1.3.5 (Installer 0.15.9)~
    \url{https://obsidian.md/}
\end{enumerate}

% ..................................................................
\subsubsection{Welche Software, Verfahren oder Technologien sind notwendig, um die Daten zu nutzen?}

\begin{enumerate}[(1)]
  \item
    mindestens: Texteditor; besser: Literaturverwaltungsprogramm
  \item
    PDF-Reader wie z.\,B.\ Adobe Acrobat Reader oder Preview (MacOS)
  \item
    Texteditor oder Markdown-Editor
\end{enumerate}

% - - - - - - - - - - - - - - - - - - - - - - - - - - - - - - - - -
\subsection{Versionierung}

% ..................................................................
\subsubsection{Werden verschiedene Versionen der Daten erzeugt (\texorpdfstring{z.\,B.}{z.B.} durch verschiedene Weiterbearbeitungsprozesse bzw.\ Bereinigung von Daten)?}

Die Datei mit den bibliographischen Metadaten (1) wurde über den gesamten Bearbeitungszeitraum hinweg erweitert;
dabei wurde jeweils die alte Version durch die neue überschrieben.
Die Volltexte (2) wurden ebenfalls über den gesamten Bearbeitungszeitraum gesammelt; die PDF-Dateien wurden in der Originalversion abgespeichert
und danach nicht geändert.
Die Datei mit den statistischen Angaben zu den Datenquellen (3) wurde am 09.06.2023 erzeugt und danach nicht geändert.

% ------------------------------------------------------------------
\section{Datennutzung}

% - - - - - - - - - - - - - - - - - - - - - - - - - - - - - - - - -
\subsection{Datenorganisation}

% ..................................................................
\subsubsection{Gibt es eine Strategie zur Benennung der Daten? Wenn ja, bitte skizzieren Sie sie kurz.}

Für die bibliographischen Metadaten (1) und die statistischen Angaben (3) gibt es jeweils nur eine Datei;
diese heißen \verb!masters_thesis.bib! und \verb!statistics_of_data_sources.md! .

Die Volltexte (2) wurden in einem separaten Ordner gespeichert, der in themenspezifische Unterordner
untergliedert ist. Die Dateinamen der Volltexte von Artikeln, Konferenzbeiträgen, Buchbeiträgen und Qualifikationsarbeiten
bestehen aus einem drei- bis vierstelligen Namenskürzel für die Autor*innen,
einer zweistelligen Jahreszahl und einer Kurzversion des Titels;
beispielsweise hat der Konferenzbeitrag
%
\begin{quote}
  {\textup
    Sina Menzel, Mark-Jan Bludau, Elena Leitner, Marian Dörk, Julián Moreno-Schneider, 
    Vivien Petras, und Georg Rehm (2020).
    \enquote{Graph Technologies for the Analysis of Historical Social Networks Using Heterogeneous Data Sources}.
    In: \emph{Graph Technologies in the Humanities 2020 (GRAPH 2020)}. Band 3110 von CEUR Workshop Proceedings, S.\,124–149%
  }
\end{quote}
%
den Dateinamen \verb!MBL+20_Graph_Technologies_HNA.pdf! .

Die Dateinamen der Volltexte aus Büchern bestehen hingegen nur aus einer Kurzversion des Titels;
diese Dateien sind nicht thematisch einsortiert, sondern in einem eigenen Unterordner abgelegt.
Zwei der Bücher haben dort eigene Unterordner, weil der Download nur in kapitelweisen PDF-Dateien
möglich war; das Namensschema dieser Dateien wurde leicht gekürzt vom Anbieter der Plattform übernommen.

% - - - - - - - - - - - - - - - - - - - - - - - - - - - - - - - - -
\subsection{Datenspeicherung und -sicherheit}

% ..................................................................
\subsubsection{Wer darf (zukünftig) auf die Daten zugreifen?}

Die bibliographischen Metadaten (1) und die statistischen Angaben (3) werden öffentlich zugänglich gemacht;
auf die Volltexte (2) dürfen aus Lizenzgründen nur der Autor der Arbeit und die Gutachter zugreifen.

% ..................................................................
\subsubsection{Wie und wie oft werden Backups der Daten erstellt?}

An den Tagen der Bearbeitung werden regelmäßig Backups auf zwei Memory-Sticks und einer externen Festplatte erstellt.
Außerdem befinden sich die Dateien in einem Ordner der DFN-Cloud der Universität Erfurt,
dessen Inhalt auf zwei Computern des Autors synchronisiert wird.

% - - - - - - - - - - - - - - - - - - - - - - - - - - - - - - - - -
\subsection{Interoperabilität}

% ..................................................................
\subsubsection{Sind die Datenformate im Sinne der FAIR-Prinzipien interoperabel, \texorpdfstring{d.\,h.}{d.h.} geeignet für den Datenaustausch und die Nachnutzung zwischen bzw.\ von unterschiedlichen Forschenden, Institutionen, Organisationen und Ländern?}

Alle verwendeten Dateiformate (siehe Abschnitt~\ref{sec:3.3.1}) sind offen.

% - - - - - - - - - - - - - - - - - - - - - - - - - - - - - - - - -
\subsection{Weitergabe und Veröffentlichung}

% ..................................................................
\subsubsection{Ist es geplant, die Daten nach Abgabe der Abschlussarbeit zu veröffentlichen oder zu teilen?}
\label{sec:4.4.1}

Die Daten sind bereits im Online-Speicherdienst Zenodo abgelegt:
%
\begin{itemize}
  \item
    \url{https://doi.org/10.5281/zenodo.8036824} \\
    öffentlich zugänglicher Datensatz mit den Metadaten und statistischen Angaben (1,\,3)
  \item
    \url{https://doi.org/10.5281/zenodo.8036903} \\
    \emph{nicht} öffentlich zugänglicher Datensatz mit den Volltexten (2)
\end{itemize}
%

% ..................................................................
\subsubsection{Wenn nicht, skizzieren Sie kurz rechtliche und/oder vertragliche Gründe und freiwillige Einschränkungen.}

Die Volltexte können aus lizenzrechtlichen Gründen nicht öffentlich zugänglich gemacht.

% ..................................................................
\subsubsection{Wenn ja, unter welchen Nutzungsbedingungen oder welcher Lizenz sollen die Daten veröffentlicht bzw. geteilt werden?}

Der öffentlich zugängliche Datensatz ist unter der freien Lizenz \href{https://creativecommons.org/licenses/by/4.0/}{CC BY 4.0} veröffentlicht.

% - - - - - - - - - - - - - - - - - - - - - - - - - - - - - - - - -
\subsection{Qualitätssicherung}

% ..................................................................
\subsubsection{Welche Maßnahmen zur Qualitätssicherung (\texorpdfstring{z.\,B.}{z.B.} Plausibilitätsprüfung von Datenwerten) werden für die Daten ergriffen?}

Die bibliographischen Metadaten wurden manuell mit den Angaben auf der jeweiligen Publikation verglichen und ggf.\ korrigiert.
Die Statistiken der Datenquellen stammen aus Dokumenten, die die Anbieter der Datenquellen zur Verfügung stellen,
und sind deshalb nur schwer einer Plausibilitätsprüfung zu unterziehen.

% - - - - - - - - - - - - - - - - - - - - - - - - - - - - - - - - -
\subsection{Datenintegration}

% ..................................................................
\subsubsection{Falls Daten aus verschiedenen Quellen (\texorpdfstring{z.\,B.}{z.B.} Anpassung Skalierung, Zeiträume, Ortsangaben) integriert werden, wie wird dies gewährleistet?}

Trifft nicht zu. Die Dateien sind in verschiedenen Formaten und unabhängig voneinander.

% ------------------------------------------------------------------
\section{Metadaten und Referenzierung}

% - - - - - - - - - - - - - - - - - - - - - - - - - - - - - - - - -
\subsection{Metadaten}

% ..................................................................
\subsubsection{Welche Informationen sind für Außenstehende notwendig, um die Daten zu verstehen (\texorpdfstring{d.\,h.}{d.h.} ihre Erhebung bzw. Entstehung, Analyse sowie die auf ihrer Basis gewonnenen Forschungsergebnisse nachvollziehen) und nachnutzen zu können?}

Der Inhalt der Dateien ist selbsterklärend.

% ..................................................................
\subsubsection{Welche Standards, Ontologien, Klassifikationen etc.\ werden zur Beschreibung der Daten und Kontextinformation genutzt?}

Trifft nicht zu.

% ------------------------------------------------------------------
\section{Rechtliche und ethische Fragen}

% - - - - - - - - - - - - - - - - - - - - - - - - - - - - - - - - -
\subsection{Personenbezogene Daten}

% ..................................................................
\subsubsection{Enthalten die Daten personenbezogene Informationen?}

Nein.

% - - - - - - - - - - - - - - - - - - - - - - - - - - - - - - - - -
\subsection{Sensible Daten}

% ..................................................................
\subsubsection{Enthalten die Forschungsdaten besondere Kategorien personenbezogener Daten nach Artikel 9 der DSGVO („Angaben über die rassische und ethnische Herkunft, politische Meinungen, religiöse oder philosophische Überzeugungen, Gewerkschaftszugehörigkeit, Gesundheit oder Sexualleben“)?}

Nein. Deshalb wurden Abschnitte 6.2.2–6.2.6 ausgelassen.

% - - - - - - - - - - - - - - - - - - - - - - - - - - - - - - - - -
\subsection{Urheber- oder verwandte Schutzrechte}

% ..................................................................
\subsubsection{Werden Daten genutzt und/oder erstellt, die durch Urheber- oder verwandte Schutzrechte geschützt sind?}

Die Volltexte sind teilweise urheberrechtlich geschützt.

% ------------------------------------------------------------------
\section{Speicherung und Langzeitarchivierung}

% - - - - - - - - - - - - - - - - - - - - - - - - - - - - - - - - -
\subsection{Wo werden die Daten (einschließlich Metadaten, Dokumentation und ggf.\ relevantem Code bzw.\ relevanter Software) während Phase der Erarbeitung der Abschlussarbeit gespeichert?}

Die Daten werden auf dem dienstlichen und dem privaten Notebook des Autors, in der DFN-Cloud der Universität Erfurt und teilweise auf GitHub gespeichert.

% - - - - - - - - - - - - - - - - - - - - - - - - - - - - - - - - -
\subsection{Wo werden die Daten (einschließlich Metadaten, Dokumentation und ggf.\ relevantem Code bzw.\ relevanter Software) nach dem Ende der Abschlussarbeit gespeichert bzw. archiviert?}

Die Daten werden auf der externen Festplatte des Autors
sowie in zwei Zenodo-Repositorien (siehe Abschnitt~\ref{sec:4.4.1}) gespeichert.

% - - - - - - - - - - - - - - - - - - - - - - - - - - - - - - - - -
\subsection{Handelt es sich dabei um ein zertifiziertes Repositorium oder Datenzentrum (z.B. durch das CoreTrustSeal, nestor-Siegel oder ISO 163639)? (Wurden mehrere Langzeitarchivierungsoptionen ausgewählt, kann die Frage bejaht werden, wenn dies auf mindestens eine der Optionen zutrifft).}

Das Zenodo-Repositorium ist (noch) nicht zertifiziert, wird aber von \href{https://fairsharing.org/}{FAIRsharing.org} empfohlen.
Auf der FAQ-Seite von Zenodo \url{https://help.zenodo.org/} heißt es dazu:

\begin{quote}
  \itshape
  \textbf{\textsf{Is Zenodo OAIS compliant?}}
  
  Zenodo and the underlying Invenio Framework for digital repositories were designed according to the OAIS reference model. Full OAIS compliance can only be proven through ISO 16363 certification which is a recent standard with very few repositories worldwide certified to date. See our infrastructure page for further details on Zenodo’s organizational and technical infrastructure.
  
  \textbf{\textsf{Does Zenodo have the CoreTrustSeal or other certification?}}
  
  Not yet since CoreTrustSeal at the moment only certifies repositories that serve a specifically designated community and therefore not generalist repositories like Zenodo. CoreTrustSeal is considering including generalist repositories in CoreTrustSeal 2022, at which point we would apply for certification. We have provided our input to CoreTrustSeal’s request for community feedback on this issue.
  
  \textbf{\textsf{Is Zenodo certified?}}
  
  Zenodo is recommended by \href{https://fairsharing.org/}{FAIRsharing.org}. We also have self-assessments for the FAIR principles and Plan S in our \href{https://about.zenodo.org/principles/}{Principles page}.
\end{quote}




\end{document}
