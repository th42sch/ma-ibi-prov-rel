% !TeX spellcheck = en_GB
% =================================================================
\chapter{Introduction}
\label{chap:intro}

% -----------------------------------------------------------------
\section{Background}
\label{sec:background}

Provenance research is concerned with the origins and ownership history
of cultural objects. Its main objective is the reconstruction of
\enquote{object biographies} in the historical context.
Application areas 
%of provenance research 
include
the study of private and public collections,
the detection of forgery,
and the identification and restitution of loot.
Since the release of the
\emph{Washington Conference Principles on Nazi-Confiscated Art} \autocite{WashingtonPrinciples}
in 1998,
provenance research has received increased attention.%
\footnote{%
  This paragraph is a brief summary of the introductory chapter in \citeauthor{Zuschlag2022}'s
  Introduction to Provenance Research \autocite[§1]{Zuschlag2022}.%
}

Regarding the holdings of university and research libraries,
particularly interesting provenances are those related to the change of
owners when an item such as a copy of a book is passed on or distributed. %\autocite[p.\,2]{Hakelberg2016}.
Provenances can be reconstructed on the basis of \emph{provenance marks} that 
are present in the item and which indicate ownership,
such as stamps, bookplates (ex libris), or handwritten signatures.
Provenance marks are essential for retracing
the \enquote{history} of a (book) item
or the extent of library holdings that have been scattered in the meantime.% \autocite[p.\,2]{Hakelberg2016}.%
\footnote{%
  This paragraph is a brief summary of the introductory section 
  in \citeauthor{Hakelberg2016}'s master's thesis on the 
  status and the perspectives of provenance indexing with authority data \autocite[§1.1]{Hakelberg2016}.%
}

In order to enable provenance research,
libraries record the provenances of their historical holdings
in their electronic catalogues.
A \emph{provenance entry} for an item usually consists of a reference to an entity such as a person, corporate body, or collection,
together with a provenance mark.
The identification and disambiguation of entities and frequently occurring provenance marks
is achieved via links to records in authority files.
%Integrated Authority File (GND) of the German National Library \autocite{GND}.
%\footnote{%
%  \url{https://www.dnb.de/EN/Professionell/Standardisierung/GND/gnd.html}%
%}
Provenance entries make it possible, for example,
to query and reconstruct the items owned or held by a single person,
to query the whereabouts of relevant items,
or to retrace the distribution of all indexed exemplars derived from a given work.

Nowadays, provenance entries are an established part of 
many electronic catalogues,
and are taken into account by the underlying data formats.
For example, the union catalogue 
\glsunset{K10plus}%
\gls{K10plus},
which is
the central database maintained and used by the libraries in the
\glsunset{GBV}%
\glsunset{SWB}%
German library networks \gls{GBV} and \gls{SWB},
provides a dedicated data field for recording provenance entries in a structured way,
plus several multi-purpose data fields which can also take provenance entries.
Despite the uniform data format,
there are several possibilities to record provenance entries.
As Hakelberg \autocite*[§4]{Hakelberg2016} explains,
libraries even within the same network often use diverse representations
for the same type of provenance entry, and the differences are considerable:
for example, some \gls{GBV} libraries record their provenance entries
in data fields at the bibliographic level (\enquote{Titelebene}, describing a manifestation),
while others use data fields at the exemplar level (\enquote{Exemplarebene}, describing held exemplars of the manifestation).
These deviations lead to large differences in the presentation
of the holdings in the online catalogue,
hindering the retrieval of relevant items and historical collections.

Data about persons and corporate bodies are recorded by librarians
in national and international authority files,
and by the general public 
in open, cross-domain knowledge bases such as Wikidata \autocite{Wikidata}.
Records on persons typically contain, among other things,
a unique identifier, the preferred name, alternative name forms,
places and times of birth and death,
as well as relationships to corporate bodies and other persons
(e.g., places of work, and family members or students).
The extent of a dataset of the same person can differ between data sources;
compare, for example, the records on the great mathematician and astronomer
Nicolaus Copernicus (1473--1543)
in various national and international authority files,
which are linked at the very end of the Wikipedia page on Copernicus
in the fold-out table \enquote{Authority Control}
\autocite{WikiCopernicusExternalLinks}.
Hence, the state of data on persons and corporate bodies is heterogeneous as well,
and it may be necessary
to consult several data sources for a given entity and combine the obtained data.

Given this diversity and heterogeneity of the existing data sources,
it is currently very difficult to retrieve provenance relationships
which require data that is distributed over several sources.
This requirement applies to a number of research questions
that can arise in cultural, historical, or provenance research.
The following list contains some examples for such questions,
which were obtained in personal communication with researchers.%
\footnote{%
%  personal communication with
  Dietrich Hakelberg 
  (head of Dept.\ \enquote{Collection Development and Cataloguing}, Research Library Gotha of the University of Erfurt)
  \autocite{HakelbergWeb},
  Michaela Scheibe
  (deputy head of Dept.\ \enquote{Manuscripts and Early Printed Books}),
  \autocite{ScheibeWeb},
  and Jo\"elle Weis 
  (head of Research Area \enquote{Digital Literary and Cultural Studies}, University of Trier), 
  \autocite{WeisWeb}.
}
%
\begin{enumerate}
  \item[\exaquery{1}]
    Who read %\todo[color=red!30]{read $\neq$ own; make clear what is meant}
    work $X$, in which manifestation and in which year?
  \item[\exaquery{2}]
    Which exemplars%
    \footnote{%
      Conforming to the FRBR model \autocite{FRBR1998},
      the precise wording should be \enquote{examplars of manifestations of expressions of $X$},
      but
      we omit the intermediate entities for brevity whenever there is no chance of misunderstanding.%
    }
    of work $X$
    were passed from one of its owners to a student?
  \item[\exaquery{3}]
    What are the relationships between the recipients of manifestation $Y$ of work $X$?
%    (or of manifestation $M$ of $W$ or of exemplar $C$ of $W$, respectively)?
  \item[\exaquery{4}]
    Which items from a collection $X$ were passed on by its owner to a family member?
  \item[\exaquery{5}]
    Which items from the holdings of library $X$ were acquired from bookseller $Y$
    between 1933 and 1945?
  \item[\exaquery{6}]
    Who participated in the sale of collection $X$?
  \item[\exaquery{7}]
    Via which paths did items from collection $X$ enter library $Y$?
  \item[\exaquery{8}]
    Which libraries own the items once owned by person $X$?
  \item[\exaquery{9}]
    Where did person $X$ acquire items and did they know the previous owners?
\end{enumerate}
%
In these examples, variables $X,Y$ are used as abstract placeholders for specific objects
such as works, manifestations, collections, etc. Therefore, \exaquery{1}--\exaquery{9} are actually
query \emph{patterns}, each of which represents a huge set of possible queries
obtained by instantiating each variable with any such object.
For the introductory purposes of this chapter,
we continue to use the abstract placeholders but refer to \exaquery{1}--\exaquery{9}
simply as \emph{queries}. % We will get back this distinction in Chapter~\ref{chap:prototype_queries}.

%Query~\exaquery{1} addresses works as well as their manifestations
%(e.g., editions of the same work in various languages).
%Answering this query would help trace the reception
%of the same work over several eras.
%For example, Duchess Luise Dorothea of Saxe-Gotha-Altenburg
%read French editions of English works by John Milton and Alexander Pope.%
%\footnote{See, for example, the provenance entries of the respective exemplars in the Research Library Gotha \autocite{OPACLuiseDorotheaMiltonPope}.}
%Obviously, there is a difference between the action \enquote{read}
%used in this query and the relation \enquote{owned} represented by
%provenance (entries). We neglect this difference for the moment
%and will get back to it in Chapter~\ref{chap:prototype_queries}.
%
Queries \exaquery{1}--\exaquery{9} are examples for a multitude of queries that arise
in the context of disciplines such as history, the social sciences, cultural studies, or law,
or in an interdisciplinary context.
For example, answers to
Query~\exaquery{1} can help trace the reception
of the same work over several eras.
Queries~\exaquery{2}--\exaquery{4}
aim at exploring the network
that spans between the recipients of a work or collection.
Queries~\exaquery{5}--\exaquery{8} are relevant for research on Nazi loot and restitution.

%Queries~\exaquery{1}--\exaquery{9}
%share the common feature of 
%
%\todo[inline]{Discuss SNA/HNA and general significance of relations; see intro report SoNAR AP~2}
%
%If a researcher
%For example, one of the two exemplars of Nicolaus Copernicus's
%main work \emph{De revolutionibus orbium coelestium} \autocite{Kopernikus1543}
%that are now held by the Gotha Research Library of the University of Erfurt
%have been owned by several scholars
%from the circle around the author,
%some of which were in the teacher-student relation
%\autocites[see, e.g.,][p.\,69]{Gingerich2002}[p.\,142]{Salatowsky2015}.
%The ownership part of this information is recorded in the library's catalogue,
%and the relationships between the owners are contained in the
%\gls{GND}.
%Hence, researchers can answer Queries~\exaquery{2} and~\exaquery{3} manually
%by inspecting the item's entry in the library's catalogue,
%following the links to the GND entries of the owners,
%and inspecting those entries for personal relationships.

%An important difference between Queries~\exaquery{2} and~\exaquery{3} is the following.
%While \exaquery{2} fixes a relation between persons (\enquote{student})
%and asks for works in whose context this relation has instances,
%\exaquery{3} does not fix a particular relation but asks for the entire context of
%the manifestation.
%
%Query \exaquery{4} is very similar in structure to \exaquery{2}.
%Query \exaquery{5} is important in the context of research on Nazi loot.
%
%\todo[inline]{Discuss \emph{all} of \exaquery{1}--\exaquery{9}?}

Queries~\exaquery{1}--\exaquery{9}
have in common that relations of bibliographic and inter-personal (or inter-institutional)
nature play an important role.
These relations are typically distributed over several data sources.
For example, in order to answer Query~\exaquery{2},
a researcher would have to find exemplars of work $X$
in the catalogues of various libraries or library networks,
inspect the provenance entries of each exemplar for owners,
find records for those owners in one or several authority files,
and inspect those records for professional relationships.
This process is highly laborious
and involves interaction with a multitude of heterogeneous data sources
at an expert level.

It is obvious that the described process would benefit greatly 
from automation, i.e., from a procedure that receives a query as an input,
consults the various data sources autonomously, collects answers, and presents them
to the user. In order to be as generic as possible,
such a procedure requires an abstract model of data sources and possible queries,
and it needs to be based on a careful analysis of the available data sources and their data models,
and on data integration techniques. These requirements are closely linked
to the question raised by \textcite[p.\,46, translated from German]{Hakelberg2016}:
\enquote{How can historical provenance relationships be formulated and represented
in a machine-readable way?}

%In order to implement suitable tools,
%it is necessary to analyse available data sources, data models, and data integration techniques,
%to develop an abstract model of data sources and possible queries,
%and to devise a method for obtaining answers in this abstract framework.
%


%\todo[think,inline]{Further thoughts on example queries:}
%%
%\begin{itemize}
%  \item
%    While the difference between persons and institutions may be negligible
%    from a modelling point of view, it is highly relevant concerning the
%    amount of query answers and their handling.
%\end{itemize}
%
% -----------------------------------------------------------------
\section{Goal and Research Question}
\label{sec:research_questions}

In this thesis, we pursue the goal of facilitating
the automated retrieval of provenance relationships.
More precisely,
we want to develop a method for answering provenance queries that refer to bibliographic entities, people, and corporate bodies,
as well as bibliographic and inter-personal or inter-institutional relationships.
This method should, on input of a query,
autonomously consult relevant data sources,
retrieve the required information, and return the set of all answers to the user.
The method should furthermore provide a high-level specification
for the implementation of a retrieval system
that supports the user in formulating their queries, answering them, and exploring the data that supports the query answers.
It is our long-term vision that such a retrieval system will support provenance research
by prospectively retrieving potentially interesting constellations.

%
%\todo[inline]{Describe intended group of users (and align begin of §\ref{chap:method})}

The described goal leads to the following
central research question for this thesis.
%
%\begin{quote}
%  \begin{itshape}
%    How can provenance relationships be modelled and automatically retrieved?
%  \end{itshape}
%\end{quote}
\begin{mdframed}[
  linewidth=1pt,
  linecolor=black!50,
%  innertopmargin=-3pt,
  innerleftmargin=0pt,innerrightmargin=0pt,
  leftline=false,rightline=false
]
  \begin{enumerate}
    \item[\RQ\phantom{\mybold{1}}]
  %    \begin{mdframed}[roundcorner=10pt]
        \mybold{How can provenance relationships be modelled and automatically retrieved?}
  %    \end{mdframed}
  \end{enumerate}
\end{mdframed}
%
This question implies several subordinate questions:
%
\begin{mdframed}[
  linewidth=1pt,
  linecolor=black!50,
%  innertopmargin=-3pt,
  innerleftmargin=0pt,innerrightmargin=0pt,
  leftline=false,rightline=false
]
  \begin{enumerate}
    \item[\subquestion{1}]
      \emph{What is the state of research on infrastructures for the automated retrieval
      of provenance relationships? Which approach(es) is/are most closely related?}
    \item[\subquestion{2}]
      \emph{What are the general challenges for answering queries such as \exaquery{1}--\exaquery{9},
      and what are the specific challenges for an automated approach?}
    \item[\subquestion{3}]
      \emph{Which data sources, standards, data formats, and further tools
      are available for answering provenance queries using multiple, heterogeneous data sources?}
%    \item[\subquestion{1}]
%      \emph{Which data sources are available for answering provenance queries?}
%  %  \item[\subquestion{2}]
%  %    \emph{What are the overlap and differences between the contents of these data sources?}
%  %    \todo{Restrict or omit later!}
%    \item[\subquestion{2}]
%      \emph{Which techniques and tools are available for integrating
%      data from heterogeneous sources?}
    \item[\subquestion{4}]
      \emph{Based on the structure of the identified data sources,
      how can data sources, queries, and query answers be modelled in an abstract framework?}
    \item[\subquestion{5}]
      \emph{What is a suitable method for retrieving provenance relationships
      in that framework?}
  \end{enumerate}
\end{mdframed}


% -----------------------------------------------------------------
\section{Methods and Outline}
\label{sec:methods}

In order to answer our research questions, we will proceed as follows.

In Chapter~\ref{chap:rel_work}, we address \subquestion{1} by reviewing existing work 
and relating it to the goal of this thesis.
The literature review covers works on research infrastructures
for (social/historical) network analysis,
data integration techniques in general and in the library domain,
and provenance indexing.

In Chapter~\ref{chap:prototype_queries}, we address \subquestion{2}
via a case study based on the exemplary queries from Section~\ref{sec:background}.
We demonstrate a manual attempt at answering them,
discuss the expected quality of query answers,
and compare the challenges of this manual process and of an automated approach.

In order to address Questions~\subquestion{3}, we review
data sources, techniques, and tools in Chapter~\ref{chap:analysis}.
This review will cover standards for the description of bibliographic resources,
data communication protocols, data description and exchange formats,
and a range of data sources.

Based on the insights from the previous two reviews and the case study,
we address Question~\subquestion{4} in Chapter~\ref{chap:modelling} by
developing an abstract model of
data sources, queries, and query answers.
This mathematical model subsumes the previous examples
while being vastly more general: it gives a formal description of how to
build admissible queries, without restricting their contents
(i.e., the specific names, attributes, concepts, and relations used)
or their complexity. 

Based on our model and all previous insights,
we address Question~\subquestion{5} in Chapter~\ref{chap:retrieval} by
developing a method for formulating queries, retrieving query answers,
and presenting them to the user.
This model should serve as the basis for a future implementation of a retrieval system
as indicated in the previous section.

Finally, we draw conclusions in Chapter~\ref{chap:conclusion}.

Our central research question originates from library and information science (LIS).
With the development of our model and method,
we hope to provide a means for LIS to benefit 
from established methods from mathematics and computer science.

% -----------------------------------------------------------------
\section{Acknowledgements}
\label{sec:acks}

The author expresses his sincere thanks to 
Prof.\ Dr.\ Robert Jäschke for supervising and agreeing to review this thesis
and for valuable suggestions; Christian Rüter for agreeing to review this thesis and for valuable feedback;
Dr.\ Dietrich Hakelberg for an inspiring introduction to provenance research
and stimulating discussions from which the main idea for this thesis originates;
\glsunset{SoNAR}%
Prof.\ Vivien Petras, PhD, for valuable suggestions and the pointer to the \gls{SoNAR} project;
\glsreset{SoNAR}%
Dr.\ Joëlle Weis and Michaela Scheibe for inspiring conversations and valuable suggestions;
Dr.\ Nadine Neute for inspiring discussions and valuable feedback;
and, last but not least,
Dr.\ Renate Stein for proofreading, valuable feedback, and ongoing support,
especially during the challenging four months allotted for preparing this thesis.

