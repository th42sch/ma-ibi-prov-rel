% !TeX spellcheck = de_DE
%\thispagestyle{plain}
%\begin{titlepage}
%
%  \includepdf[%
%%    addtotoc={1,chapter,0,Titelei,titelei},
%    pages=-,
%    pagecommand={\thispagestyle{empty}}%
%  ]{%
%    BHR_template/BHR_Titelei.pdf%
%  }
%
%\end{titlepage}


\begin{titlepage}
\centering

{\Huge \scshape Humboldt-Universität zu Berlin \par}
{\Large \scshape Institut für Bibliotheks- und Informationswissenschaft \par}

\vspace{2cm}

\alt{Logo der Humboldt-Universität zu Berlin. Köpfe von Wilhelm und Alexander von Humboldt nach links schauend umrandet mit dem Namen der Universität.}
\includegraphics[scale=.7]{BHR_template/husiegel-sw.png}\\

\vspace{2cm}

{\huge \scshape Berliner Handreichungen \\zur Bibliotheks- und Informationswissenschaft \par}

\vspace{2cm}

{\LARGE \scshape Heft XXX \par} %bei XXX bitte die Heftnummer eintragen

\vspace{2,5cm}

{\Large \scshape \foreignlanguage{english}{Modelling and Automated Retrieval} \\ \foreignlanguage{english}{of Provenance Relationships} \par}

\vspace{2cm}

{\LARGE \scshape von \\ Thomas Schneider}

\end{titlepage}

\clearpage{\thispagestyle{empty}\cleardoublepage}
\clearpage{\thispagestyle{empty}\cleardoublepage}
\centering
\phantom{x}
\vspace{5cm}

{\Large \scshape \foreignlanguage{english}{Modelling and Automated Retrieval} \\ \foreignlanguage{english}{of Provenance Relationships} \par}

\vspace{4cm}

{\Large \scshape von \\ Thomas Schneider \par}

\vspace{3cm}
\hrule
\vspace{1cm}
\raggedleft
{\Large Berliner Handreichungen zur \\
Bibliotheks- und Informationswissenschaft \\
\vspace{1cm}
Begründet von Peter Zahn \\
Herausgegeben von \\
Vivien Petras \\
Humboldt-Universität zu Berlin \\
\vspace{.5cm}
Heft XXX} % bei XXX bitte die Heftnummer eintragen

\newpage
\thispagestyle{empty}
\justifying
\phantom{x}
\vspace{.5cm}

{ \large \textbf{Schneider, Thomas} \par

\foreignlanguage{english}{Modelling and Automated Retrieval of Provenance Relationships}
/ von Thomas Schneider. -- Berlin : Institut für Bibliotheks- und Informationswissenschaft
der Humboldt-Universität zu Berlin, 2024. -- XX S. : graph. Darst. -- (Berliner Handreichungen zur Bibliotheks- und Informationswissenschaft
; XXX) \par %bei XXX bitte die Heftnummer eintragen und ggf. Jahreszahl überprüfen/korrigieren

 ISSN 14 38-76 62 \par} % ISSN ändert sich nicht


\vspace*{\fill}
\selectlanguage{english}
\subsection*{\textrm{Abstract}}
This study investigates the querying and retrieval of provenance information,
with an emphasis on modelling queries, data sources, and query answers.
As a starting point, a set of prototypical queries in natural language is collected;
these queries refer to bibliographic resources as well as relations between non-bibliographhic entities such as owners
of a resource. 
A literature review provides an overview of available data sources and techniques.
In the main part, a generic approach to modelling data sources,
queries, and query answers is developed, resulting in a graph-based model.
Finally, an abstract method for implementing this model in a retrieval system is designed and discussed.

%Abstände ggf. entsprechend der Länge des Abstract anpassen
\selectlanguage{ngerman}

\vspace*{\fill}
Diese Veröffentlichung geht zurück auf eine Masterarbeit im weiterbildenden Ma\-ster\-stu\-di\-en\-gang im Fernstudium Bibliotheks- und Informationswissenschaft (\foreignlanguage{english}{Library and Information Science}, M. A. (LIS)) an der Humboldt- Universität zu Berlin aus dem Jahr 2023.\par
%
%bei XXX bitte die Jahreszahl eintragen, wann die Arbeit ursprünglich eingereicht wurde
%
%Diese Veröffentlichung geht zurück auf eine Masterarbeit im Studiengang \foreignlanguage{english}{Information Science}, M. A. an der Humboldt-Universität zu Berlin aus dem Jahr XXX.\par
%
%Diese Veröffentlichung geht zurück auf eine Bachelorarbeit im Studiengang Bibliotheks- und Informationswissenschaft (\foreignlanguage{english}{Library and Information Science}) an der Humboldt-Universität zu Berlin aus dem Jahr XXX.\par
%
Eine Online-Version ist auf dem edoc Publikationsserver der Humboldt-Universität zu Berlin verfügbar. \par

\begin{wrapfigure}{L}{0.185\textwidth}
\vspace{-.481cm}
\includegraphics[scale=.7]{BHR_template/by.png}
%\includegraphics[scale=.7]{by-sa.png}
%\includegraphics[scale=.7]{by-nd.png}
%\includegraphics[scale=.2]{by-nc-eu.png}
%\includegraphics[scale=.2]{by-nc-sa-eu.png}
%\includegraphics[scale=.2]{by-nc-nd-eu.png}
\end{wrapfigure}

\begin{small}
  Sofern nicht anders angegeben, ist dieses Werk in seiner Gesamtheit verfügbar unter einer \href{https://creativecommons.org/licenses/by/4.0/}{\foreignlanguage{english}{Creative Commons} Namensnennung 4.0 International} Lizenz. %Lizenz und URL ggf. anpassen
  Einzelne Bestandteile, für die diese Lizenz keine Anwendung findet und die daher nicht unter deren Lizenzbedingungen verwendet werden dürfen, sind mit ihren jeweiligen lizenzrechtlichen Bestimmungen in Form zusätzlicher Texthinweise gekennzeichnet.
  \par
\end{small}
