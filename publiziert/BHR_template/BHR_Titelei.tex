% !TeX spellcheck = en_GB
%letzte Aktualisierung: 2024-01-18

\documentclass[paper=a4,11pt,twoside,parskip=half-]{scrreprt} %scrreprt ist die empfohlene Klasse in Kombination mit dem accessibility package

%------------------------- Sprache/fonts ------------------------------
\usepackage[utf8]{inputenc} %encoding of the input text
\usepackage[T1]{fontenc} %font encoding of the output text
%\usepackage[utf8]{luainputenc}% alternative for \usepackage[utf8]{inputenc} when using LuaLaTeX as engine
\usepackage{lmodern} %Latin Modern fonts
\usepackage[english,ngerman]{babel} %Sprachpaket
%Babel bietet 3 Möglichkeiten zwischen Sprachen umzuschalten:
%1) \selectlanguage{language} ist für recht lange Abschnitte gedacht. 
%2) otherlanguage-Umgebung ist für kürzere Abschnitte gedacht. 
%3) \foreignlanguage{language}{text} kann z.B. für einzelne Begriffe genutzt werden.
%Die im Header bei \usepackage[english,ngerman]{babel} zuletzt angegebene Sprache, ist die Hauptsprache des Dokuments.
%Siehe: https://hickerspace.org/wiki/Mehrere_Sprachen_in_einem_Latex-Dokument
%\usepackage[xcolornames,svgnames,dvipsnames,RGB]{xcolor}  %% TS: Option ''xcolornames'' verursacht Fehler: ''Unknown option 'xcolornames' for package xcolor.''
\usepackage[svgnames,dvipsnames,RGB]{xcolor}
%Farbpaket für das Einfärben von Schrift
\definecolor{hublau}{RGB}{0,55,108}
%\usepackage{dtk-logos} %Latex logos
\usepackage{blindtext}
%----------------------------------------------------------------------

%--------------------------- Satzspiegel ------------------------------
\usepackage{setspace} %Zeilenabstand
\onehalfspacing %1.5 Zeilenabstand
\usepackage{multicol} %mehrspaltiger Satz
\usepackage[inner=30mm, % inner margin 
outer=25mm, %  outer margin
top=25mm,   % top margin of the page.
bottom=25mm]{geometry}
\KOMAoption{cleardoublepage}{plain}
\usepackage{ragged2e} %bietet u.a. den Befehl \justifying um zwischen Blocksatz, rechtsbündig und linksbündig zu wechseln
%\usepackage{typearea}
%\usepackage{showframe}
%\usepackage{scrlayer-scrpage}
%\pagestyle{scrheadings}
%----------------------------------------------------------------------

%------------------------------ Mathe ---------------------------------
%\usepackage{siunitx} %package zur korrekten Darstellung von SI Einheiten
\usepackage{amsmath} %Mathematikumgebung (American Mathematical Society Standard)
\usepackage{amsfonts} %Schriften
\usepackage{amssymb} %Symbole
\usepackage[inline]{enumitem} %customizing the three basic list environments: enumerate, itemize and description; option inline: three environments for inline lists are defined: enumerate*, itemize*, and description*
%\usepackage{cancel} %diagonale Linien ziehen
%\usepackage{minted} %for code in LaTeX
%----------------------------------------------------------------------


%---------------------------- Grafiken --------------------------------
\usepackage{graphicx} %Paket zum Einbinden von Grafiken
\usepackage{wrapfig} %Paket zum Einbinden von Schrift umflossenen Bildern und Tabellen
\usepackage{rotating} %Paket zum Drehen von Bildern und Tabellen
\usepackage{tabularx} % table package, man kann eine definierte Breite festlegen
%\newcolumntype{Y}{>{\centering\arraybackslash}X} %for centering X columns
\usepackage{booktabs} %table package that offers the commands \toprule \cmidrule \bottomrule
%----------------------------------------------------------------------


%---------------------------- Anhänge ---------------------------------
\usepackage[style=german]{csquotes} %Kontextsensitive Zitatanlage: Intelligente Zitierungen welche sich dynamisch an ihren Inhalt anpassen - Anführungszeichen wechseln automatisch im Fall, dass die Zitate verschachtelt sind
%\usepackage[style=apa,backend=biber]{biblatex} %Literaturverzeichnis im APA Stil
%\DeclareLanguageMapping{ngerman}{ngerman-apa} %Sprachmapping wichtig für biber, sonst Fehler beim compiling
%\addbibresource{Datei.bib} % bib-Datei laden
%\usepackage[printonlyused,withpage]{acronym} 
%Abkürzungsverzeichnis
%\usepackage{caption} %caption erzwingen mit \captionof auch ohne figure oder table Umgebung
%----------------------------------------------------------------------

%------------------------- new commands -------------------------------
\makeatletter
\newcommand*\addsubsec{\secdef\@addsubsec\@saddsubsec}
\newcommand*{\@addsubsec}{}
\def\@addsubsec[#1]#2{\subsection*{#2}\addcontentsline{toc}{subsection}{#1}
  \if@twoside\ifx\@mkboth\markboth\markright{#1}\fi\fi
}
\newcommand*{\@saddsubsec}[1]{\subsection*{#1}\@mkboth{}{}}
\makeatother
\makeatletter
% bewirkt die Verwendung des Befehls \addsubsec
% Hinzufügen von subsections, die ohne Nummerierung im ToC erscheinen
%----------------------------------------------------------------------


%-------------------------- hyperref/hypersetup ------------------------------------------
\usepackage{hyperref} %% TS: ''Package hyperxmp Error: hyperref must be loaded before hyperxmp.''
\usepackage{hyperxmp}
\hypersetup{%
pdfauthor={Thomas Schneider},
pdfkeywords={information retrieval, data integration, linked data, research infrastructure},
pdflang={en},
pdftitle={Modelling and Automated Retrieval of Provenance Relationships},
pdfcopyright={CC BY 4.0}, %Lizenz ggf. ändern
%pdfdate={YYYY-MM-DD},
pdflicenseurl={https://creativecommons.org/licenses/by/4.0/}, %URL ggf. ändern
bookmarksopen={true}, % open up bookmark tree
bookmarksnumbered={true},
breaklinks={true},
pdfborder={0 0 0}, %Setzt den Rahmen um einen Link. {0 0 0} erzeugt keinen Rahmen.
colorlinks={true}, %Legt fest, ob die Schrift von Links farbig sein soll.
linkcolor={black},	%Farbe für Dokument interne Links
urlcolor={hublau},	%Farbe für URL-Links (Web, Mail).
citecolor={black},
citebordercolor={0 0 0}, %Farbe für den Rahmen um die Links für Zitate.
urlbordercolor={0 0 0},
}
\usepackage[ngerman,noabbrev]{cleveref}
%--------------------------------------------------------------------------

%------------------------------ accesibility ------------------------------
\usepackage[highstructure]{accessibility}

%============================= Ende der Präambel ==========================================
\begin{document}
\begin{titlepage}
\centering

{\Huge \scshape Humboldt-Universität zu Berlin \par}
{\Large \scshape Institut für Bibliotheks- und Informationswissenschaft \par}

\vspace{2cm}

\alt{Logo der Humboldt-Universität zu Berlin. Köpfe von Wilhelm und Alexander von Humboldt nach links schauend umrandet mit dem Namen der Universität.}
\includegraphics[scale=.7]{husiegel-sw.png}\\

\vspace{2cm}

{\huge \scshape Berliner Handreichungen \\zur Bibliotheks- und Informationswissenschaft \par}

\vspace{2cm}

{\LARGE \scshape Heft XXX \par} %bei XXX bitte die Heftnummer eintragen

\vspace{2,5cm}

{\Large \scshape Modelling and Automated Retrieval \\ of Provenance Relationships \par}

\vspace{2cm}

{\LARGE \scshape von \\ Thomas Schneider}

\end{titlepage}

\clearpage{\thispagestyle{empty}\cleardoublepage}
\clearpage{\thispagestyle{empty}\cleardoublepage}
\centering
\phantom{x}
\vspace{5cm}

{\Large \scshape Modelling and Automated Retrieval \\ of Provenance Relationships \par}

\vspace{4cm}

{\Large \scshape von \\ Thomas Schneider \par}

\vspace{3cm}
\hrule
\vspace{1cm}
\raggedleft
{\Large Berliner Handreichungen zur \\
Bibliotheks- und Informationswissenschaft \\
\vspace{1cm}
Begründet von Peter Zahn \\
Herausgegeben von \\
Vivien Petras \\
Humboldt-Universität zu Berlin \\
\vspace{.5cm}
Heft XXX} % bei XXX bitte die Heftnummer eintragen

\newpage
\thispagestyle{empty}
\justifying
\phantom{x}
\vspace{.5cm}

{ \large \textbf{Schneider, Thomas} \par

Modelling and Automated Retrieval of Provenance Relationships
/ von Thomas Schneider. -- Berlin : Institut für Bibliotheks- und Informationswissenschaft
der Humboldt-Universität zu Berlin, 2024. -- XX S. : graph. Darst. -- (Berliner Handreichungen zur Bibliotheks- und Informationswissenschaft
; XXX) \par %bei XXX bitte die Heftnummer eintragen und ggf. Jahreszahl überprüfen/korrigieren

 ISSN 14 38-76 62 \par} % ISSN ändert sich nicht


\vspace*{\fill}
\subsection*{\textrm{\foreignlanguage{english}{Abstract}}}
*** TBD ***
%Abstände ggf. entsprechend der Länge des Abstract anpassen

\vspace*{\fill}
Diese Veröffentlichung geht zurück auf eine Masterarbeit im weiterbildenden Ma\-ster\-stu\-di\-en\-gang im Fernstudium Bibliotheks- und Informationswissenschaft (\foreignlanguage{english}{Library and Information Science}, M. A. (LIS)) an der Humboldt- Universität zu Berlin aus dem Jahr XXX.\par
%
%bei XXX bitte die Jahreszahl eintragen, wann die Arbeit ursprünglich eingereicht wurde
%
%Diese Veröffentlichung geht zurück auf eine Masterarbeit im Studiengang \foreignlanguage{english}{Information Science}, M. A. an der Humboldt-Universität zu Berlin aus dem Jahr XXX.\par
%
%Diese Veröffentlichung geht zurück auf eine Bachelorarbeit im Studiengang Bibliotheks- und Informationswissenschaft (\foreignlanguage{english}{Library and Information Science}) an der Humboldt-Universität zu Berlin aus dem Jahr XXX.\par
%
Eine Online-Version ist auf dem edoc Publikationsserver der Humboldt-Universität zu Berlin verfügbar. \par

\begin{wrapfigure}{l}{0.185\textwidth}
\vspace{-.481cm}
\includegraphics[scale=.7]{by.png}
%\includegraphics[scale=.7]{by-sa.png}
%\includegraphics[scale=.7]{by-nd.png}
%\includegraphics[scale=.2]{by-nc-eu.png}
%\includegraphics[scale=.2]{by-nc-sa-eu.png}
%\includegraphics[scale=.2]{by-nc-nd-eu.png}
\end{wrapfigure}

\small  Sofern nicht anders angegeben, ist dieses Werk in seiner Gesamtheit verfügbar unter einer \href{https://creativecommons.org/licenses/by/4.0/}{\foreignlanguage{english}{Creative Commons} Namensnennung 4.0 International} Lizenz. %Lizenz und URL ggf. anpassen
Einzelne Bestandteile, für die diese Lizenz keine Anwendung findet und die daher nicht unter deren Lizenzbedingungen verwendet werden dürfen, sind mit ihren jeweiligen lizenzrechtlichen Bestimmungen in Form zusätzlicher Texthinweise gekennzeichnet. 
%\newpage
%\tableofcontents
%\chapter{Kapitel 1}
%\blindtext
%\section{Abschnitt 1}
%\blindtext
%\subsection{Unterabschnitt 1}
%\chapter{Kapitel 2}

\end{document}