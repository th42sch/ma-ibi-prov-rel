% !TeX spellcheck = en_GB
% =================================================================
\chapter{Conclusion}
\label{chap:conclusion}

% -----------------------------------------------------------------
\section{Summary of the Results}
\label{sec:summary}

In this thesis, we have examined the problem of modelling and automated retrieval of provenance relationships
that require information which is usually distributed over several data sources.
We have provided a review of the related literature,
a case study with example queries, and a review of available standards, data sources,
and techniques for data exchange and integration.
As the main contribution, we have developed an abstract and general model for queries,
data sources, and query answers, and we have devised a method for implementing this model
in an answer retrieval system, together with a discussion of various design choices.

In order to return to the initial research question and its subquestions,
we list them here once more.
\begin{mdframed}[
  linewidth=1pt,
  linecolor=black!50,
%  innertopmargin=-3pt,
  innerleftmargin=0pt,innerrightmargin=0pt,
  leftline=false,rightline=false
]
  \begin{enumerate}
    \item[\RQ\phantom{\mybold{1}}]
  %    \begin{mdframed}[roundcorner=10pt]
        \mybold{How can provenance relationships be modelled and automatically retrieved?}
  %    \end{mdframed}
  \end{enumerate}
\end{mdframed}
%
This question implies several subordinate questions:
%
\begin{mdframed}[
  linewidth=1pt,
  linecolor=black!50,
%  innertopmargin=-3pt,
  innerleftmargin=0pt,innerrightmargin=0pt,
  leftline=false,rightline=false
]
  \begin{enumerate}
    \item[\subquestion{1}]
      \emph{What is the state of research on infrastructures for the automated retrieval
      of provenance relationships? Which approach(es) is/are most closely related?}
    \item[\subquestion{2}]
      \emph{What are the general challenges for answering queries such as \exaquery{1}--\exaquery{9},
      and what are the specific challenges for an automated approach?}
    \item[\subquestion{3}]
      \emph{Which data sources, standards, data formats, and further tools
      are available for answering provenance queries using multiple, heterogeneous data sources?}
    \item[\subquestion{4}]
      \emph{Based on the structure of the identified data sources,
      how can data sources, queries, and query answers be modelled in an abstract framework?}
    \item[\subquestion{5}]
      \emph{What is a suitable method for retrieving provenance relationships
      in that framework?}
  \end{enumerate}
\end{mdframed}
%
We now summarise the insights obtained in this thesis relating to each of these questions.

\paragraph{\RQ[1] (Chapter~\ref{chap:context})}

Our literature review has shown that
social network analysis and historical network analysis are generic methods closely related to the goal of this thesis;
the general context of social and historical research produces research questions similar to
those underlying provenance research.
The SoNAR project is a rich source of relevant insights concerning the support for historical research
by research data infrastructures.
While network analysis aims at answering questions of a \enquote{global} nature,
our scenario covers more \enquote{local} questions.

Furthermore, our review has also shown that
linked (open) data and data integration techniques have been used extensively in the library domain,
and that the state of provenance indexing varies greatly between, and sometimes even within,
institutions and library networks.

\paragraph{\RQ[2] (Chapter~\ref{chap:prototype_queries})}

The general challenges for answering queries such as~\exaquery{1}--\exaquery{9}
include several sources for missing and spurious answers.
A prominent reason is the mismatch between natural-language terms for concepts or relations
and the vocabulary of the data sources.
A possible remedy is the use of \enquote{substitute information}, hypotheses, or reasoning.
This problem carries even more weight in an automated approach, where
a machine cannot easily rely on human intuition or experience for identifying substitute information
and hypothesising.

\paragraph{\RQ[3] (Chapter~\ref{chap:analysis})}

There is a large number of bibliographic standards, bibliographic and generic data formats,
communication protocols, and data sources. We have identified four groups of altogether 22
obvious data sources, and this list is by far not exhaustive. Our detailed analysis of four
of the 22 data sources shows that, in particular, linked (open) data and the relevant
interfaces are provided by all of them. 
In order to implement a prototype of a retrieval system that is supposed to provide a proof of concept,
it certainly suffices to include these four data sources.
However, a fully-fledged retrieval system should probably include a substantially larger selection of data sources;
in this case, our analysis will need to be extended accordingly.

\paragraph{\RQ[4] (Chapter~\ref{chap:modelling})}

We have developed a basic, conceptually simple, graph-based model for data sources, queries, and query answers,
and sketched several extensions. For the basic version and some of the extensions,
the query answering problem has the same
good computational properties as standard query answering over relational databases.
However, some desirable features in queries may require extensions of the model that no longer lend themselves
to intuitive visualisations; in order to capture these features, an \gls{SQL}-like syntax could be appropriate.
Finally, some aspects used in queries cannot be captured in any formal model
and require manual intervention.

\paragraph{\RQ[5] (Chapter~\ref{chap:method})}

We have developed an abstract method consisting of two main phases and three subphases.
The decision between the dynamic and the static setting is a fundamental design choice,
as it affects all phases. The comparison of the two settings does not yield a clear tendency
towards one or the other. The challenges that both systems have in common imply
that the system needs to be constantly maintained by domain experts and implementers,
which favours a web application.
The developed method is a high-level specification for a future implementation of a retrieval system,
and many details still remain be clarified or studied further.

\paragraph{\RQ}

There is neither a single model nor a single method,
but our graph-based model and two-phase method seem suitable.
The abstract model captures most of the exemplary questions, 
and the general method has a number of parameters
that need to be set or require further study.

% -----------------------------------------------------------------
\section{Outlook}
\label{sec:outlook}

As immediate next steps, further data sources should be analysed, and the generic method
should be elaborated in more detail. Subsequently, a prototype retrieval system should be implemented
and tested. Ideally, this prototype is then turned into a fully-fledged web application.

The work reported here can be continued in a number of further directions:
A systematic quantitative analysis of relevant queries and data sources
in the form of an extensive user study
would help underpin our ad-hoc selection of example queries
as well as, hopefully, the abstract model. 
The testing phase for the prototype system could be integrated in this study,
leading to an agile development cycle.
The testing of the system should also include a pilot study on the correctness and completeness of query answers
with expert users in order to validate and improve the overall approach.
Furthermore, the system could be integrated in the research infrastructure
in the context of the \gls{SoNAR} project or in large portals such as Proveana or Europeana.
Finally, the use of semantic web technologies such as ontologies and reasoning
should be explored in detail.

