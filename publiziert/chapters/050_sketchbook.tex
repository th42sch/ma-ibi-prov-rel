% !TeX spellcheck = en_GB
% =================================================================
\chapter*{Sketchbook}

% -------------------------------------------------------------------
\section*{``Storyline''}

\begin{tikzpicture}[
  >=Latex,
  every node/.style={draw=none,fill=none,thick,inner sep=1.5mm},
  every edge/.style={draw=black,thick}  
]
  \node                                 (queries)         {Prototype queries};
  \node [below=10mm of queries]         (datasources)     {\tikztab[c]{Requirement analysis:\\[-1pt]Analysis of available data sources\\[-1pt]Analysis of possible queries}};
  \node [below=10mm of datasources]     (modelling)       {Modelling};
  \node [below=10mm of modelling]       (dataintegration) {Data integration};
  \node [below=10mm of dataintegration] (method)          {Method};
  
  \path[->]
    (queries)         edge (datasources)
    (datasources)     edge (modelling)
    (modelling)       edge (dataintegration)
    (dataintegration) edge (method)
  ;
  
\end{tikzpicture}

\begin{itemize}
  \item 
    Modelling depends on requirement analysis: why is it appropriate to use graphs rather than ``fully-fledged'' database theory?
  \item
    OTOH, the early specification of prototype queries will necessarily remain vague and extensional
    as long as we don't have an abstract model!
  \item
    perhaps use ``lightweight'' def. of a conjunctive query?
\end{itemize}

