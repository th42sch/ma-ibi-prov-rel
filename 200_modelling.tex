% !TeX spellcheck = en_GB

% =================================================================
\chapter{Modelling Provenance Relationships and Queries}

In this chapter, we develop a generic approach to modelling provenance relationships
that will serve as a basis for the investigations in the following chapters.
The central notion of this approach is that of a directed graph with edge labels,
which is a standard notion in computer science and discrete mathematics.
\todo{citation}

In a nutshell, a directed graph consists of a set of nodes, a set of directed edges between
the nodes, and a function that assigns labels to the edges.
In our setting, these three abstract components have the following meaning:
%
\begin{itemize}
  \item 
    Nodes represent entities such as works, expressions, manifestations, copies,
    persons, or corporations.
  \item 
    Edges represent relationships between nodes, which are typically directed:
    e.g., \term{has\_owner} points \emph{from} a book copy \emph{to} a person or corporation,
    whereas \term{is\_owner\_of} points into the opposite direction.%
    \footnote{%
      Symmetric (i.e., bidirectional) relationships, such as \term{is\_collaborator\_of},
      can be represented via two edges, one for each direction.%
    }
  \item 
    Edge labels allow the specification of one or several relation names for each edge.
    For example, if a person $p$ is author \emph{and} owner of a book copy $b$,
    then this can be represented via an edge from $b$ to $p$ with the label
    $\{\term{has\_author},\term{has\_owner}\}$
    (and/or an edge from $b$ to $p$ with the label $\{\term{is\_author\_of},\term{is\_owner\_of}\}$).
\end{itemize}
%
A directed graph with edge labels can be visualised in the obvious way:
nodes are represented using circles or rectangles,
edges are denoted by edges,
and edge labels are written next to the respective edge,
see Figure~\ref{fig:example_eld_graph}.

\begin{figure}
  \missingfigure{example}
  \caption{An example graph}
  \label{fig:example_eld_graph}
\end{figure}

As we will see in the following, edge-labelled directed graphs can be used in our setting
to represent (combinations of) data sources as well as queries.
They allow us to draw on standard notions from graph theory and query answering
in order to define admissible query answers and to devise methods for obtaining those.

% ------------------------------------------------------------------
\section{Basic Definitions from Graph Theory}

\begin{definition}
  Let $R$ be a set of \emph{relation names}.
  A \emph{directed edge-labelled graph over $(N,R)$} is a triple $G = (V,E,\Lmc)$,
  where
  %
  \begin{itemize}
    \item
    $V$ is a set, whose members are called or \emph{nodes};\footnote{%
      In classical graph theory, nodes are called \emph{vertices}; thus the set of
      nodes of a graph is denoted by $V$. We adopt the denotation $V$ for conformity
      and the more modern term ``node'' for brevity.%
    }      
    \item 
    $E \subseteq V \times V$ is a set of pairs of nodes, whose members are called \emph{edges};
    \item
    $\Lmc : E \to 2^R$ is a function that assigns to each edge a non-empty set of relation names,
    called the \emph{labels} of that edge; we call \Lmc a \emph{labelling function}.
  \end{itemize}
\end{definition}
%
\missingfigure{example}


% ------------------------------------------------------------------
\section{Modelling Data Sources}

% ------------------------------------------------------------------
\section{Modelling Queries and Answers}


% ------------------------------------------------------------------
\section{OLD: Provenance Relationship Networks}

In order to define provenance relationship networks, we now fix sets $N$ and $R$ to consist
of elements that are relevant for provenance research:
The set $N$ consists of \emph{entities} such as 
works, expressions, manifestations, copies, people, and corporations
that are contained in a fixed collection of data sources (such as library catalogues,
authority files, and knowledge bases). The set $R$ consists of relationships
between entities that are taken either from the FRBR model\todo{cite, introduce}
or from the same collection of data sources (in particular, relationships between people and/or corporations
such as ``is student of'' or ``collaborated with'' or ...)

*** FIRST THINK ABOUT QUERY TYPES, THEN CONTINUE WITH THESE DEF.S ***


A \emph{provenance relationship network} is a directed edge-labelled graph over $(N,R)$,
where $N$ ... and $R$ ... \todo{continue}
