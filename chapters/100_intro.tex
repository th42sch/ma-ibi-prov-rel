% !TeX spellcheck = en_GB
% =================================================================
\chapter{Introduction}
\label{chap:intro}

% -----------------------------------------------------------------
\section{Setting}
\label{sec:setting}

In research on the historical holdings of university and research libraries,
the origins of book copies\todo{really restrict to books?} are of central importance.
The origin of an item is also called its \emph{provenance} 
and comprises the people or corporations that owned this item over time.
Particularly interesting provenances are those related to the change of
owners when book copies are passed on or distributed \autocite[p.\,2]{Hakelberg2016}.
Provenances can be reconstructed by marks of ownership
such as stamps, bookplates (ex libris), or handwritten signatures:
with the help of these features, it is possible to retrace
the ``history'' of a book copy
or the extent of library holdings that have been scattered in the meantime \autocite[p.\,2]{Hakelberg2016}.

In order to enable provenance research,
libraries index the provenances of their historical holdings
and make them available to users via their catalogues.
A provenance entry refers to a person or corporation
and to a feature that indicates ownership.
In order for owners to be identified unequivocally,
they are often given via a reference to authority files such as the
Integrated Authority File (GND) of the German National Library.%
\footnote{%
  \url{https://www.dnb.de/EN/Professionell/Standardisierung/GND/gnd.html}%
}
The indexed provenance data makes it possible, for example,
to query and reconstruct the books owned or held by a single person or corporation,
to query the whereabouts of relevant copies,
or to retrace the distribution of all indexed copies of a work.

In the digital age, provenance entries are recorded in electronic catalogues.
German university and research libraries typically do not 
maintain their own individual catalogue; rather they are provided with a central
catalogue by the library network in which they participate.%
\footnote{%
  There are six library networks \emph{(Bibliotheksverbünde)} for scientific libraries
  in Germany: \url{https://de.wikipedia.org/wiki/Bibliotheksverbund\#Deutschland}%
}
These central catalogues are equipped with an underlying database
and standardised data formats for internal representation and data export.
For example, the networks GBV and SWB maintain and use the common catalogue (database)
\emph{K10plus},%
\footnote{\url{https://www.bszgbv.de/services/k10plus/}}
which internally uses the data format PICA%
\footnote{\url{https://format.gbv.de/pica}}
and allows for exports in the data formats
MARC 21, MAB2, and Pica+.%
\footnote{\url{https://wiki.k10plus.de/display/K10PLUS/Exportformate}}
%  \footnote{\url{https://de.wikipedia.org/wiki/Machine-Readable_Cataloging}}%
%  \footnote{\url{https://www.dnb.de/DE/Professionell/Metadatendienste/Exportformate/MARC21/marc21_node.html}}
Despite the use of a uniform data format,
there are several possibilities to record provenance entries.
As Hakelberg \autocite[Chapter~4]{Hakelberg16} explains,
libraries even within the same network often use diverse representations
for the same type of provenance entry, and the differences are considerable:
for example, some GBV libraries record their provenance entries
in data fields on the bibliographic level,
while others use the level on the exemplar level.
These deviations lead to large differences in the presentation
of the holdings in the online catalogue,
which impedes the retrieval of relevant copies and historical holdings.


\todo[inline]{TODO: continue import from \emph{Exposé}}


% -----------------------------------------------------------------
\section{Research Question(s) and Aim of This Thesis}
\label{sec:research_questions}

\todo[inline]{TODO: import from \emph{Exposé}}

We want to develop a method for answering provenance queries that refer to bibliographic entities, people, and corporations
as well as the relationships between those. This method should consult standard data sources such as 
library catalogues, authority files, and knowledge bases. The method should furthermore be implementable as a software tool
that supports the user in formulating their queries, answering them, and exploring the data that supports the query answers.
In the long term, we envisage that such a tool will support provenance research
by prospectively retrieving potentially interesting constellations.

% -----------------------------------------------------------------
\section{Outline}
\label{sec:outline}

\begin{itemize}
  \item
    review of related work: Chapter~\ref{chap:rel_work}
  \item
    discussion of prototype queries: Chapter~\ref{chap:prototype_queries}
  \item
    analysis of available data sources and techniques: Chapter~\ref{chap:analysis}
  \item
    generic model of data sources, queries, and answers: Chapter~\ref{chap:modelling}
  \item
    method for answering queries: Chapter~\ref{chap:retrieval}
  \item
    conclusion and outlook: Chapter~\ref{chap:conclusion}
\end{itemize}


