% !TeX spellcheck = en_GB
% =================================================================
\chapter{Introduction}
\label{chap:intro}

% -----------------------------------------------------------------
\section{Background}
\label{sec:background}

\todo[inline,color=red!30]{Address \enquote{Raubgut} (loot?), items acquired without purchase $\leadsto$ add relevant prototype queries?}

In research on the historical holdings of university and research libraries,
the origins of book copies\todo[color=red!30]{restrict to books only later!} are of central importance.
The origin of an item is also called its \emph{provenance} 
and comprises the people or corporations\todo[color=red!30]{institution? see RDA toolkit. Throughout.}
that owned this item over time.
Particularly interesting provenances are those related to the change of
owners when book copies are passed on or distributed \autocite[p.\,2]{Hakelberg2016}.
Provenances can be reconstructed by marks of ownership
such as stamps, bookplates (ex libris), or handwritten signatures:
with the help of these features, it is possible to retrace
the \enquote{history} of a book copy
or the extent of library holdings that have been scattered in the meantime \autocite[p.\,2]{Hakelberg2016}.

In order to enable provenance research,
libraries index the provenances of their historical holdings
and make them available to users via their catalogues.
A provenance entry refers to a person or corporation
and to a feature that indicates ownership.
In order for owners to be identified unequivocally,
they are often given via a reference to authority files such as the
Integrated Authority File (GND) of the German National Library.%
\footnote{%
  \url{https://www.dnb.de/EN/Professionell/Standardisierung/GND/gnd.html}%
}
The indexed provenance data makes it possible, for example,
to query and reconstruct the books owned or held by a single person or corporation,
to query the whereabouts of relevant copies,
or to retrace the distribution of all indexed copies of a work.

Nowadays, provenance entries are recorded in electronic catalogues.
German university and research libraries typically do not 
maintain their own individual catalogue; rather they are provided with a central
catalogue by the library network in which they participate.%
\footnote{%
  There are six library networks \emph{(Bibliotheksverbünde)} for scientific libraries
  in Germany: \url{https://de.wikipedia.org/wiki/Bibliotheksverbund\#Deutschland}%
}
These central catalogues are equipped with an underlying database
and standardised data formats for internal representation and data export.
For example, the networks GBV and SWB maintain and use the common catalogue (database)
\emph{K10plus},%
\footnote{\url{https://www.bszgbv.de/services/k10plus/}}
which internally uses the data format PICA%
\footnote{\url{https://format.gbv.de/pica}}
and allows for exports in the data formats
MARC 21, MAB2, and Pica+.%
\footnote{\url{https://wiki.k10plus.de/display/K10PLUS/Exportformate}}
%  \footnote{\url{https://de.wikipedia.org/wiki/Machine-Readable_Cataloging}}%
%  \footnote{\url{https://www.dnb.de/DE/Professionell/Metadatendienste/Exportformate/MARC21/marc21_node.html}}
Despite the use of a uniform data format,
there are several possibilities to record provenance entries.
As Hakelberg \autocite*[Chapter~4]{Hakelberg2016} explains,
libraries even within the same network often use diverse representations
for the same type of provenance entry, and the differences are considerable:
for example, some GBV libraries record their provenance entries
in data fields on the bibliographic level,
while others use the level on the exemplar level.
These deviations lead to large differences in the presentation
of the holdings in the online catalogue,
which hinders the retrieval of relevant copies and historical holdings.

Data about persons and corporations are recorded
in Germany- or worldwide authority files and further databases,
such as the previously mentioned GND,
the \emph{WorldCat},%
\footnote{\url{https://www.worldcat.org/identities/}}
databases of projects such as ISNI%
\footnote{\url{https://isni.org/}}
and VIAF%
\footnote{\url{https://viaf.org/}}
or in Wikidata.%
\footnote{\url{https://www.wikidata.org/wiki/Wikidata:Main_Page}}
These data sources usually support standardized data formats for data export,
such as MARC~21 and RDF via several interfaces in the case of GND.
Data about a person contain, among others, the name and alternative name forms
(which can be manifold in the case of, e.g., scholars of previous centuries),
places of birth, death, and work,
as well as relationships to corporations and other persons
(e.g., coauthors and students).
The extent of a dataset of the same person can differ between data sources,
which is witnessed, for example, by the entries on the scholar
Georg Joachim Rheticus in ISNI, VIAF, WorldCat, GND, and Wikidata.%
\footnote{%
  This can be verified by following,
  at the end of the Wikipedia page on Rheticus 
  in the table „Authority Control“ \url{https://en.wikipedia.org/wiki/Georg_Joachim_Rheticus\#External_links}\,,
  the links to ISNI, VIAF, WorldCat und GND,
  and inspecting the Wikidata data set
  \url{https://www.wikidata.org/wiki/Q93588}\,.
}
Hence, the state of data on persons and corporation is heterogeneous as well,
and depending on the concrete individual, it may be necessary
to consult several data sources and combine the obtained data.

Given this diversity and heterogeneity of the existing data sources,
it is currently difficult to impossible to retrieve provenance relationships
that are not restricted to bibliographic items and their owners
but which also involve various relationships between works, copies, and (multiple) owners.
For example, one could be interested in answers to queries of the following form.
%
\begin{enumerate}
  \item[\exaquery{1}]
    Who read\todo[color=red!30]{read $\neq$ own; make clear what is meant}
    work $W$, in which manifestation and in which year?
  \item[\exaquery{2}]
    Which copies of work $W$ were passed from one of its owners to a collaborator (or a student)?
  \item[\exaquery{3}]
    What are the relationships between the recipients of work $W$
    (or of manifestation $M$ of $W$ or of copy $C$ of $W$, respectively)?
\end{enumerate}
%
In these examples, we use variables $W,M,C$ to refer to arbitrary works,
manifestations, or copies. Therefore, \exaquery{1}--\exaquery{3} are in fact
\emph{query patterns}, each of which represents a set of possible queries
that can be obtained by assigning concrete objects to the variables.
We will elaborate on this thought in Chapter~\ref{chap:prototype_queries}.

Query Pattern~\exaquery{1} addresses works as well as their manifestations
(e.g., editions of the same work in various languages).
An answer to a query of this type would allow it to trace the reception
of the same work over several eras. For example, Duchess Luise Dorothea of Saxe-Gotha-Altenburg
read French editions of English works.\todo{give more concrete example}\footnote{Dietrich Hakelberg, personal communication}

Query Patterns~\exaquery{2} and~\exaquery{3}
aim at highlighting the network
that spans between the recipients of a work.
For example, one of the two copies of Nicolaus Copernicus's
main work \emph{De revolutionibus orbium coelestium} \autocite{Kopernikus1543}
that are now held by the Gotha Research Library of the University of Erfurt
have been owned, or at least read, by several scholars
from the circle around the author,
some of which were in the teacher-student relationship.
This information can be concluded
from the accounts of Gingerich~\autocite[p.\,69]{Gingerich2002}
and Salatowsky and Lotze~\autocite[p.\,142]{Salatowsky2015},
but it can also be obtained by looking up the entry and its owners in the electronic catalogue of the library,
following the links to the GND entries of the owners,
and inspecting the relationships between the owners in GND;
see Chapter~\ref{chap:prototype_queries} for a more detailed derivation.
%\footnote{\url{https://opac.uni-erfurt.de/LNG=EN/DB=1/XMLPRS=N/PPN?PPN=567506266}, field \enquote{Provenance(s)}}

An important difference between Patterns~\exaquery{2} and~\exaquery{3} is the following.
While \exaquery{2} fixes a relationship between two persons (\enquote{collaborator} or \enquote{student})
and asks for works in whose context this relation occurs,
\exaquery{3} does not fix a particular relationship but asks for the entire context of
the work or a manifestation or copy thereof.

When attempting to answer queries conforming to patterns such as \exaquery{1}--\exaquery{3},
it does not suffice to consult a single data source such as
a catalogue, authority file, or knowledge base.
Instead, it is necessary to consult several data sources
and combine the information found. This process is highly laborious,
given not just the number of data sources but also their diverse
data formats. Therefore, automated support is essential.
This is one of the reasons for Hakelberg
to raise the following question \autocite[p.\,46, translated from German]{Hakelberg2016}:
\enquote{How can historical provenance relationships be formulated and represented
in a machine-readable way?}

In order to implement suitable tools,
it is necessary to analyse available data sources, data models, and data integration techniques,
to develop an abstract model of data sources and possible queries,
and to devise a method for obtaining answers in this abstract framework.

\todo[inline,color=red!30,caption={}]{%
  \parbox{.99\linewidth}{%
    Further thoughts on prototype queries:
    %
    \begin{itemize}
      \item
        While the difference between persons and institutions may be negligible
        from a modelling point of view, it is highly relevant concerning the
        amount of query answers and their handling.
      \item
        Copies with traces of ownerships that do not name the owner
        can be particularly interesting to study in the context of missing holdings.
        Some libraries record such traces (e.g., HAAB Weimar), but of course not exhaustively so.
        Could this be a prototype query that can be incorporated into the model,
        or should it be discussed under possible extensions?
    \end{itemize}
  }
}

% -----------------------------------------------------------------
\section{Aim and Research Question}
\label{sec:research_questions}

In this thesis, we pursue the goal of facilitating
the automated retrieval of provenance relationships.
More precisely,
we want to develop a method for answering provenance queries that refer to bibliographic entities, people, and corporations
as well as the relationships between those. This method should consult standard data sources such as 
library catalogues, authority files, and knowledge bases. The method should furthermore be implementable as a software tool
that supports the user in formulating their queries, answering them, and exploring the data that supports the query answers.
In the long term, we envisage that such a tool will support provenance research
by prospectively retrieving potentially interesting constellations.

This goal leads to the following
central research question for this thesis.
%
\begin{quote}
  \begin{itshape}
    How can provenance relationships be modelled and automatically retrieved?
  \end{itshape}
\end{quote}
%  
This question implies several subordinate questions:
%
\begin{enumerate}
  \item[\subquestion{1}]
    \emph{Which data sources are available for answering provenance queries?}
  \item[\subquestion{2}]
    \emph{What are the overlap and differences between the contents of these data sources?}
    \todo{Restrict or omit later!}
  \item[\subquestion{3}]
    \emph{Which techniques and tools are available for the integration
    of data from heterogeneous sources?}
  \item[\subquestion{4}]
    \emph{Based on the structure of the identified data sources,
    how data sources, queries, and answers be modelled in an abstract framework?}
  \item[\subquestion{5}]
    \emph{What is a suitable method for retrieving provenance relationships
    in that framework?}
\end{enumerate}


% -----------------------------------------------------------------
\section{Methods}
\label{sec:methods}

\todo[inline,color=red!30]{Connect questions from library science with methods from maths/theor. CS $\leadsto$ benefit. Name methods for each SQ?}


% -----------------------------------------------------------------
\section{Outline}
\label{sec:outline}

\begin{itemize}
  \item
    review of related work: Chapter~\ref{chap:rel_work}
  \item
    discussion of prototype queries: Chapter~\ref{chap:prototype_queries}
  \item
    analysis of available data sources and techniques: Chapter~\ref{chap:analysis}
  \item
    generic model of data sources, queries, and answers: Chapter~\ref{chap:modelling}
  \item
    method for answering queries: Chapter~\ref{chap:retrieval}
  \item
    conclusion and outlook: Chapter~\ref{chap:conclusion}
  \item
    name methods used?
\end{itemize}


