% !TeX spellcheck = en_GB
% =================================================================
\chapter{Introduction}
\label{chap:intro}

% -----------------------------------------------------------------
\section{Setting}
\label{sec:setting}

In research on the historical holdings of university and research libraries,
the origins of book copies\todo{really restrict to books?} are of central importance.
The origin of an item is also called its \emph{provenance} 
and comprises the people or corporations that owned this item over time.
Particularly interesting provenances are those related to the change of
owners when book copies are passed on or distributed \cite[p.\,2]{Hakelberg2016}.
Provenances can be reconstructed by marks of ownership
such as stamps, bookplates (ex libris), or handwritten signatures:
with the help of these features, it is possible to retrace
the ``history'' of a book copy
or the extent of library holdings that have been scattered in the meantime \cite[p.\,2]{Hakelberg2016}.


\todo[inline]{TODO: continue import from \emph{Exposé}}


% -----------------------------------------------------------------
\section{Research Question(s) and Aim of This Thesis}
\label{sec:research_questions}

\todo[inline]{TODO: import from \emph{Exposé}}

We want to develop a method for answering provenance queries that refer to bibliographic entities, people, and corporations
as well as the relationships between those. This method should consult standard data sources such as 
library catalogues, authority files, and knowledge bases. The method should furthermore be implementable as a software tool
that supports the user in formulating their queries, answering them, and exploring the data that supports the query answers.
In the long term, we envisage that such a tool will support provenance research
by prospectively retrieving potentially interesting constellations.

% -----------------------------------------------------------------
\section{Outline}
\label{sec:outline}

\begin{itemize}
  \item
    review of related work: Chapter~\ref{chap:rel_work}
  \item
    discussion of prototype queries: Chapter~\ref{chap:prototype_queries}
  \item
    analysis of available data sources and techniques: Chapter~\ref{chap:analysis}
  \item
    generic model of data sources, queries, and answers: Chapter~\ref{chap:modelling}
  \item
    method for answering queries: Chapter~\ref{chap:retrieval}
  \item
    conclusion and outlook: Chapter~\ref{chap:conclusion}
\end{itemize}


