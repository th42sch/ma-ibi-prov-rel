% !TeX spellcheck = en_GB
% =================================================================
\chapter{Introduction}
\label{chap:intro}

% -----------------------------------------------------------------
\section{Background}
\label{sec:background}

Provenance research is concerned with the origins and ownership history
of cultural objects. Its main objective is the reconstruction of
\enquote{object biographies} in the historical context.
Application areas of provenance research include
the study of private and public collections,
the detection of forgery,
and the identification and restitution of loot.
Since the release of the
\emph{Washington Conference Principles on Nazi-Confiscated Art} \autocite{WashingtonPrinciples}
in 1998,
provenance research has received increased attention.%
\footnote{%
  This paragraph is a brief summary of the introductory chapter in \citeauthor{Zuschlag2022}'s
  Introduction to Provenance Research \autocite*[Chapter 1]{Zuschlag2022}.%
}

Regarding the holdings of university and research libraries,
particularly interesting provenances are those related to the change of
owners when a book copy is passed on or distributed \autocite[p.\,2]{Hakelberg2016}.
Provenances can be reconstructed by marks of ownership
such as stamps, bookplates (ex libris), or handwritten signatures:
with the help of these features, it is possible to retrace
the \enquote{history} of a book item
or the extent of library holdings that have been scattered in the meantime \autocite[p.\,2]{Hakelberg2016}.

In order to enable provenance research,
libraries index the provenances of their historical holdings
and make them available to users via their catalogues.
A provenance entry refers to a person or coporate body
and to a feature that indicates ownership.
In order for owners to be identified unequivocally,
they are often given via a reference to authority files such as the
Integrated Authority File (GND) of the German National Library \autocite{GND}.
%\footnote{%
%  \url{https://www.dnb.de/EN/Professionell/Standardisierung/GND/gnd.html}%
%}
The indexed provenance data makes it possible, for example,
to query and reconstruct the items owned or held by a single person or corporate body,
to query the whereabouts of relevant items,
or to retrace the distribution of all indexed exemplars derived from a given work.

Nowadays, provenance entries are recorded in electronic catalogues.
German university and research libraries typically do not
maintain their own individual catalogue; rather they are provided with a central
catalogue by the library network in which they participate.%
\footnote{%
  There are six library networks \emph{(Bibliotheksverbünde)} for scientific libraries
  in Germany %\url{https://de.wikipedia.org/wiki/Bibliotheksverbund\#Deutschland}%
  \autocite{WikiBibliotheksverbundDeutschland}.
}
These central catalogues are equipped with an underlying database
and standardised data formats for internal representation and data export.
For example, the networks GBV and SWB maintain and use the common catalogue (database)
\emph{K10plus} \autocite{K10plus},
%\footnote{\url{https://www.bszgbv.de/services/k10plus/}}
which internally uses the data format PICA \autocite{PICA}
%\footnote{\url{https://format.gbv.de/pica}}
and allows for exports in the data formats
MARC 21, MAB2, and Pica+ \autocite{K10plusExportformate}.
%\footnote{\url{https://wiki.k10plus.de/display/K10PLUS/Exportformate}}
%  \footnote{\url{https://de.wikipedia.org/wiki/Machine-Readable_Cataloging}}%
%  \footnote{\url{https://www.dnb.de/DE/Professionell/Metadatendienste/Exportformate/MARC21/marc21_node.html}}
Despite the use of a uniform data format,
there are several possibilities to record provenance entries.
As Hakelberg \autocite*[Chapter~4]{Hakelberg2016} explains,
libraries even within the same network often use diverse representations
for the same type of provenance entry, and the differences are considerable:
for example, some GBV libraries record their provenance entries
in data fields on the bibliographic level,
while others use the level on the exemplar level.
These deviations lead to large differences in the presentation
of the holdings in the online catalogue,
which hinders the retrieval of relevant items and historical holdings.

Data about persons and corporate bodies are recorded
in Germany- or worldwide authority files and further databases,
such as the previously mentioned GND,
the \emph{WorldCat} \autocite{WorldCat},
%\footnote{\url{https://www.worldcat.org/identities/}}
databases of projects such as ISNI \autocite{ISNI}
%\footnote{\url{https://isni.org/}}
and VIAF \autocite{VIAF}
%\footnote{\url{https://viaf.org/}}
or in Wikidata \autocite{Wikidata}.
%\footnote{\url{https://www.wikidata.org/wiki/Wikidata:Main_Page}}
These data sources usually support standardized data formats for data export,
such as MARC~21 and RDF via several interfaces in the case of GND.
Data about a person contain, among others, the name and alternative name forms
(which can be manifold in the case of, e.g., scholars of previous centuries),
places of birth, death, and work,
as well as relationships to corporate bodies and other persons
(e.g., coauthors and students).
The extent of a dataset of the same person can differ between data sources,
which is witnessed, for example, by the entries on the scholar
Georg Joachim Rheticus in ISNI, VIAF, WorldCat, GND, and Wikidata.%
\footnote{%
  This can be verified by following,
  at the end of the Wikipedia page on Rheticus
  in the table \enquote{Authority Control}
  \autocite{WikiRheticusExternalLinks},
  %\url{https://en.wikipedia.org/wiki/Georg_Joachim_Rheticus\#External_links}\,,
  the links to ISNI, VIAF, WorldCat und GND,
  and inspecting the Wikidata data set
  \autocite{WikidataRheticus}.
  %\url{https://www.wikidata.org/wiki/Q93588}\,.
}
Hence, the state of data on persons and corporate body is heterogeneous as well,
and depending on the concrete individual, it may be necessary
to consult several data sources and combine the obtained data.

Given this diversity and heterogeneity of the existing data sources,
it is currently difficult to retrieve provenance relationships
that are not restricted to bibliographic objects and their owners
but which also involve various relationships between works, exemplars, and (multiple) owners.
For example, cultural scientists are interested in answers to queries of the following form.%
\footnote{from personal communication with Dietrich Hakelberg, Research Library Gotha of the University of Erfurt, and Jo\"elle Weis, University of Trier}
%
\begin{enumerate}
  \item[\exaquery{1}]
    Who read %\todo[color=red!30]{read $\neq$ own; make clear what is meant}
    work $X$, in which manifestation and in which year?
  \item[\exaquery{2}]
    Which exemplars%
    \footnote{%
      Conforming to the FRBR model \autocite{FRBR1998},
      the precise wording should be \enquote{examplars of manifestations of expressions of $X$}
      but, here and in the following,
      we omit the intermediate entities for brevity when no misunderstanding is expected.%
    }
    of work $X$
    were passed from one of its owners to a student?
  \item[\exaquery{3}]
    What are the relationships between the recipients of manifestation $Y$ of work $X$?
%    (or of manifestation $M$ of $W$ or of exemplar $C$ of $W$, respectively)?
  \item[\exaquery{4}]
    Which exemplars from a collection $X$ were passed on by its owner to a family member?
  \item[\exaquery{5}]
    Which exemplars from the holdings of library $X$ were acquired from bookseller $Y$
    between 1933 and 1945?
\end{enumerate}
%
In these examples, we use variables $X,Y$ as placeholders for arbitrary works,
manifestations, collections, etc. Therefore, \exaquery{1}--\exaquery{5} are actually
query \emph{patterns}, each of which represents a set of possible queries
that can be obtained by instantiating the variables with concrete objects.
For the introductory purposes of this chapter,
we continue to use the placeholders and refer to \exaquery{1}--\exaquery{5}
simply as queries. We will get back this distinction in Chapter~\ref{chap:prototype_queries}.

Query~\exaquery{1} addresses works as well as their manifestations
(e.g., editions of the same work in various languages).
Answering this query would help trace the reception
of the same work over several eras.
For example, Duchess Luise Dorothea of Saxe-Gotha-Altenburg
read French editions of English works by John Milton and Alexander Pope.%
\footnote{See, for example, the provenance entries of the respective exemplars in the Research Library Gotha \autocite{OPACLuiseDorotheaMiltonPope}.}
Obviously, there is a difference between the action \enquote{read}
used in this query and the relationship \enquote{owned} represented by
provenance (entries). We neglect this difference for the moment
and will get back to it in Chapter~\ref{chap:prototype_queries}.

Queries~\exaquery{2} and~\exaquery{3}
aim at highlighting the network
that spans between the recipients of a work.
For example, one of the two exemplars of Nicolaus Copernicus's
main work \emph{De revolutionibus orbium coelestium} \autocite{Kopernikus1543}
that are now held by the Gotha Research Library of the University of Erfurt
have been owned, by several scholars
from the circle around the author,
some of which were in the teacher-student relationship.
This information can be concluded
from the accounts of Gingerich~\autocite[p.\,69]{Gingerich2002}
and Salatowsky and Lotze~\autocite[p.\,142]{Salatowsky2015},
but it can also be obtained by looking up the entry and its owners in the electronic catalogue of the library,
following the links to the GND entries of the owners,
and inspecting the relationships between the owners in GND;
see Chapter~\ref{chap:prototype_queries} for a more detailed derivation.
%\footnote{\url{https://opac.uni-erfurt.de/LNG=EN/DB=1/XMLPRS=N/PPN?PPN=567506266}, field \enquote{Provenance(s)}}

An important difference between Queries~\exaquery{2} and~\exaquery{3} is the following.
While \exaquery{2} fixes a relationship between two persons (\enquote{collaborator} or \enquote{student})
and asks for works in whose context this relation occurs,
\exaquery{3} does not fix a particular relationship but asks for the entire context of
the work or a manifestation or exemplar thereof.

Query \exaquery{4} is very similar in structure to \exaquery{2}.
Query \exaquery{5} is important in the context of research on Nazi loot.

\todo[inline]{Discuss SNA/HNA and general significance of relationships; see intro report SoNAR AP~2}

When attempting to answer queries such as \exaquery{1}--\exaquery{4},
it does not suffice to consult a single data source such as
a catalogue, authority file, or knowledge base.
Instead, it is necessary to consult several data sources
and combine the information found. This process is highly laborious,
given not just the number of data sources but also their diverse
data formats. Therefore, automated support is essential.
This is one of the reasons for Hakelberg
to raise the following question \autocite[p.\,46, translated from German]{Hakelberg2016}:
\enquote{How can historical provenance relationships be formulated and represented
in a machine-readable way?}

In order to implement suitable tools,
it is necessary to analyse available data sources, data models, and data integration techniques,
to develop an abstract model of data sources and possible queries,
and to devise a method for obtaining answers in this abstract framework.

\todo[think,inline]{Further thoughts on example queries:}
%
\begin{itemize}
  \item
    While the difference between persons and institutions may be negligible
    from a modelling point of view, it is highly relevant concerning the
    amount of query answers and their handling.
\end{itemize}

% -----------------------------------------------------------------
\section{Aim and Research Question}
\label{sec:research_questions}

In this thesis, we pursue the goal of facilitating
the automated retrieval of provenance relationships.
More precisely,
we want to develop a method for answering provenance queries that refer to bibliographic entities, people, and coporate bodies
as well as the relationships between those. This method should consult standard data sources such as
library catalogues, authority files, and knowledge bases. The method should furthermore be implementable as a software tool
that supports the user in formulating their queries, answering them, and exploring the data that supports the query answers.
In the long term, we envisage that such a tool will support provenance research
by prospectively retrieving potentially interesting constellations.

This goal leads to the following
central research question for this thesis.
%
\begin{quote}
  \begin{itshape}
    How can provenance relationships be modelled and automatically retrieved?
  \end{itshape}
\end{quote}
%
This question implies several subordinate questions:
%
\begin{enumerate}
  \item[\subquestion{1}]
    \emph{Which data sources are available for answering provenance queries?}
%  \item[\subquestion{2}]
%    \emph{What are the overlap and differences between the contents of these data sources?}
%    \todo{Restrict or omit later!}
  \item[\subquestion{2}]
    \emph{Which techniques and tools are available for the integration
    of data from heterogeneous sources?}
  \item[\subquestion{3}]
    \emph{Based on the structure of the identified data sources,
    how data sources, queries, and answers be modelled in an abstract framework?}
  \item[\subquestion{4}]
    \emph{What is a suitable method for retrieving provenance relationships
    in that framework?}
\end{enumerate}


% -----------------------------------------------------------------
\section{Methods and Outline}
\label{sec:methods}

In order to answer our research questions, we will proceed as follows.

In Chapter~\ref{chap:rel_work}, we review existing work 
and connect it with our research questions.\todo[defer]{be more precise}
In Chapter~\ref{chap:prototype_queries}, we revisit
the exemplary queries from Section~\ref{sec:background},
demonstrate a manual attempt at answering them,
and discuss the expected quality of query answers and difficulties with this manual process.
This discussion is used as a point of reference for the subsequent considerations.
\todo[defer]{possibly adapt this paragraph later}

In order to answer Questions~\subquestion{1} and~\subquestion{2}, we review
data sources, data models, and data integration techniques in Chapter~\ref{chap:analysis}.
Based on the results of this analysis, we develop an abstract model of
data sources, queries, and answers in Chapter~\ref{chap:modelling}.
This mathematical model subsumes the previous examples
while being vastly more general: it gives a formal description of how to
build admissible queries, without restricting their contents
(i.e., the specific names, attributes, concepts, and relationships used)
or their complexity. Using this model, we hope to provide a means
for library science, which is the origin of the research question,
to benefit from established methods from mathematics and computer science.

Based on our model and the insights from the preceding analysis,
we develop in Chapter~\ref{chap:retrieval}
a method for answering queries that can serve as the basis of a software tool
as indicated in the previous section.
Finally, we draw conclusions in Chapter~\ref{chap:conclusion}.

