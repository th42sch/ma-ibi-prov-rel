% !TeX spellcheck = en_GB
% =================================================================
\chapter{Analysis of Available Data Sources and Techniques}
\label{chap:analysis}

\dots

% -----------------------------------------------------------------
\section{Data Sources}
\label{sec:data_sources}

\begin{itemize}
  \item
    library catalogues: OPAC, network catalogues (K10plus etc.), DNB,
    \mybold{KVK} (\enquote{federated search}: shallower than in single catalogues,
    but (sufficient) information on all copies!);
    further catalogues?
  \item
    authority files: GND (persons, corporate bodies, places, relationships, what else?
  \item
    KBs: Wikidata, what else?
  \item
    more from project SoNAR (IDH), see report on WP 2 (in particular details on p.23f.).
    Get back to the discussion on GND and further data sources in Section~\ref{sec:HNA+SoNAR}.
  \item
    see also the list by \autocite{Menzel2020}:

    \blockquote{%
      \begin{itemize}
        \item
          The Integrated Authority File (GND) represents and describes 8,295,047 entities (people, corporations, conferences, geographical areas, technical terms, and works);
        \item
          The German National Library (DNB) provides descriptions of bibliographic resources. The dataset has 19,926,573 records of books, magazines, newspapers, sheet music, music recordings, audio books etc.;
        \item
          The German Union Catalogue of Serials (ZDB) describes newspapers, magazines, serial titles, yearbooks, etc. and contains 1,908,334 records;
        \item
          The Kalliope Union Catalog (KPE) is a collection of personal papers, manuscripts, and publishers’ archives, which consists of 26,752 records;
        \item
          The Newspaper Information System (ZeFYS) represents 2,596,641 digitized pages of historical newspapers and full texts;
        \item
          The Exile Press represents German-language exile journals between 1933 and 1945 and consists of 5,336 digitized pages.
      \end{itemize}
    }
  \item
    Proveana, Kulturgut-DB (recommended by Jo\"elle Weis)
  \item 
    criteria for analysis of data sources: see, e.g., SoNAR grant proposal (appendices) 
  \item
    further criteria: arity of relationships, data formats, temporal data?
  \item
    also clarify: does the structure of these data sources support the model to be developed?
  \item
    focus the following considerations on a narrow choice of these data sources;
    the general approach to be developed should be largely independent on that concrete choice
  \item 
    in-depth analysis of data sources (e.g., overlap and differences) is worthwhile
    but must be deferred to future work
  \item
    DHa's remarks on GND vocabulary and richness of GND data:
    %
    \begin{itemize}
      \item
        \emph{Individualisierungsregeln}: relationships to other people are given in order to uniquely
        identify the current person; there is no obligation to record relationships
        (and GND does not claim to be an encyclopedia)
      \item
        relations are not normed:
        e.g., field \enquote{Funktionsbezeichnung}: \$4 code \enquote{Eigenschaft der Beziehung},
        specified by \$v \enquote{Freitext}
      \item
        professions are recorded in 678 \$b \enquote{biographische Anmerkung(en)} --
        historically in free text,
        more recently via link to the GND authority data for the respective profession (i.e., more standardised)
      \item
        Tp1 datasets: field 65 (6J?) \enquote{GND Systematik: eingeschr. Tätigkeit} (?),
        query level 1 datasets?
      \item
        $\leadsto$ \mybold{look/read all this up!}
    \end{itemize}
\end{itemize}

% -----------------------------------------------------------------
\section{Data Models}
\label{sec:data_models}

\begin{itemize}
  \item
    FRBR et al.
  \item 
    \mybold{OPAC or K10plus data model as ER diagram? Binary vs. \boldmath$n$-ary relations?
    $\leadsto$ check literature (possibly also papers on the FRBR model)!}
\end{itemize}

\dots

% -----------------------------------------------------------------
\section{Data Integration and Further Techniques}
\label{sec:data_integration}

\begin{itemize}
  \item 
    ETL (from grant proposal SoNAR, p.~8, AP1-1)
  \item
    upper-level ontologies?
  \item 
    ontologies on research
  \item
    FRBRoo?
  \item
    RDF, SPARQL, N-Quads?
\end{itemize}


\dots

% -----------------------------------------------------------------
\section{Implications on Modelling}
\label{sec:implications_on_modelling}

\dots

\begin{itemize}
  \item
    \mybold{constants plus unary and binary relationships suffice (?)}
  \item 
    online vs.\ offline phase?
    %
    \begin{itemize}
      \item
        online: data source graph is implicit; data sources are queried \enquote{on the fly}
      \item
        offline: data source graph is generated explicitly (using data integration techniques)
        and updated in fixed intervals
    \end{itemize}
    %
\end{itemize}


