% !TeX spellcheck = en_GB
% =================================================================
\chapter{Analysis of Available Data Sources and Techniques}
\label{chap:analysis}

\dots

% -----------------------------------------------------------------
\section{Data Sources}
\label{sec:data_sources}

\begin{itemize}
  \item
    library catalogues: OPAC, network catalogues (K10plus etc.), DNB,
    \mybold{KVK} (\enquote{federated search}: shallower than in single catalogues,
    but (sufficient) information on all copies!);
    further catalogues?
  \item
    authority files: GND, what else?
  \item
    KBs: Wikidata, what else?
  \item
    focus the following considerations on a narrow choice of these data sources;
    the general approach to be developed should be largely independent on that concrete choice
  \item 
    in-depth analysis of data sources (e.g., overlap and differences) is worthwhile
    but must be deferred to future work
\end{itemize}

% -----------------------------------------------------------------
\section{Data Models}
\label{sec:data_models}

\begin{itemize}
  \item
    FRBR et al.
  \item 
    \mybold{OPAC or K10plus data model as ER diagram? Binary vs. \boldmath$n$-ary relations?
    $\leadsto$ check literature (possibly also papers on the FRBR model)!}
\end{itemize}

\dots

% -----------------------------------------------------------------
\section{Data Integration and Further Techniques}
\label{sec:data_integration}

\begin{itemize}
  \item
    upper-level ontologies?
  \item 
    ontologies on research
  \item
    FRBRoo?
  \item
    RDF, SPARQL, N-Quads?
\end{itemize}


\dots

% -----------------------------------------------------------------
\section{Implications on Modelling}
\label{sec:implications_on_modelling}

\dots

\begin{itemize}
  \item
    \mybold{constants plus unary and binary relationships suffice (?)}
\end{itemize}


