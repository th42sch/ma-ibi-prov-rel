% !TeX spellcheck = en_GB
% =================================================================
\chapter{Analysis of Available Data Sources and Techniques}
\label{chap:analysis}

\dots


% -----------------------------------------------------------------
\section{Data Sources}
\label{sec:data_sources}

We have searched the literature and the Web for data sources that contain information
relevant for provenance research, i.e., works, expressions, manifestations, exemplars,
persons, ownership, social relationships, and more. 
We decided to put a slight focus on data sources from the German-speaking area, 
aiming at a selection of data sources that is manageable in the context of a master's thesis,
and influenced by the selection in the \gls{SoNAR} project (see Section~\ref{sec:HNA+SoNAR}).
In the following, we will first give an overview of these data sources
and then provide more information about some of them,
including their scope and the technical infrastructure provided.
This list is not exhaustive and needs to be extended
as soon as our model is implemented in a concrete tool in future work.

% - - - - - - - - - - - - - - - - - - - - - - - -
\subsection{Collection of Data Sources}

We have identified four categories of relevant data sources:
library catalogues, special databases for provenance research, authority files, and knowledge bases.
We next describe, for each of these categories, the relevant information that is
contained in the respective data sources, and we list examples.

\subsubsection{Library Catalogues}

Library catalogues contain the following information relevant for provenance research.
%
\begin{itemize}
  \item
    bibliographic entities (works, manifestations, expressions, exemplars) \todo[quick]{check for manifestations, expressions (RDA rules, LG 3.2)}
  \item
    relations between these entities (e.g., \term{isManifestationOf})
  \item
    attributes for these entities (e.g., year of publication)
  \item
    people and corporate bodies and relations such as authorship
  \item
    provenance entries (including, e.g., owners, the ownership relation, and further attributes such as the year of ownership)
  \item
    (implicitly) the current ownership
  \item
    (ideally) references to entries in authority files for all these types of entities
\end{itemize}
%
Examples of relevant library catalogues include:
%
\begin{itemize}
  \item
    catalogues of national libraries, which often aim at collecting literature exhaustively
    and which are most likely to make metadata available in interoperable formats;
    in particular: DNB, LC, British Library \todo[quick]{take care of abbreviations, also below}
  \item
    meta-catalogues that enable a federated search in several catalogues;
    in particular: WorldCat and the Karlsruhe Virtual Catalogue (KVK)
  \item
    catalogues of library networks;
    in particular: K10plus, B3KAT, hebis-Verbundkatalog, hbz-Verbundkatalog
  \item
    catalogues of single libraries, if not covered by any of the above
  \item
    special catalogues;
    in particular: ZDB, KPE, ZeFYS, ExilePress (which have been used in the \gls{SoNAR} project) \todo{explain; relate to SoNAR or loot research}
\end{itemize}


\subsubsection{Special Databases for Provenance Research}

Special databases that have been created to support provenance research contain detailed and specific information on
the provenances of the recorded items.
Examples include Proveana and the Lost Art Database, both by the German Lost Art Foundation. \todo[quick]{links, more info}

\subsubsection{Authority Files}

Authority files contain the following information relevant for provenance research.
%
\begin{itemize}
  \item
    entities such as persons, corporate bodies, places, works
  \item
    relations such as social relations, family relations, professional relations
  \item
    attributes for the entities such as their profession
\end{itemize}
%
They are usually subject to a strict quality control.
Examples include the \gls{GND} in the German-speaking area (which has been used extensively in \gls{SoNAR}),
the \gls{LCNAF} for North America,
and \gls{ISNI} and \gls{VIAF} worldwide.

\subsubsection{Knowledge Bases}

Finally, open knowledge bases contain, among others, similar data relevant for provenance research as authority files.
In contrast to the latter, they are edited by the community; hence, they tend to be larger but apply
a less strict quality control. A prominent example is Wikidata, which has also been suggested in the context of the \gls{SoNAR} project.

% - - - - - - - - - - - - - - - - - - - - - - - -
\subsection{Analysis of Data Sources}

We now provide a review of the following data sources from the previous list:
DNB, KVK, K10plus, ZDB, KPE, Proveana, \gls{GND}, Wikidata. \todo{Europeana?}
The main goal of this analysis is to provide an exemplary overview of the features and specifics
of these data sources, and to inform the model that we will develop in the subsequent chapter.
In order to fulfil this purpose, it is not necessary to review all data sources listed above
exhaustively.

We collect the following information for each data source.
%
\begin{itemize}
  \item
    \emph{Scope:}~
    thematic focus; number of datasets; criteria for data collection; coverage ***explain both!***
  \item
    \emph{Technical infrastructure:}~
    data format; data model; interfaces
  \item
    \emph{Data quality (from the \gls{SoNAR} grant proposal \autocite[p.\,19ff.]{SchneiderKempf2018}):}~
    use of persistent identifiers and URIs; adherence to standards, e.g., for time and data specification
  \item
    \emph{criteria from our discussion of \gls{SoNAR} (Section~\ref{sec:insights_from_SoNAR}):}~
    recording of temporal attributes or relations; recording of data provenance
  \item
    further features specific to the respective data source
\end{itemize}


\dots

\par\bigskip
\todo[inline]{Take the following points into account ::}

\begin{itemize}
  \item
    \mybold{KVK} (\enquote{federated search}: shallower than in single catalogues,
    but (sufficient) information on all copies!);
    further catalogues?
  \item
    more from \gls{SoNAR} , see report on WP 2 (in particular details on p.23f.).
    Get back to the discussion on \gls{GND} and further data sources in Section~\ref{sec:HNA+SoNAR}.
  \item
    see also the list by \autocite{Menzel2020}:

    \blockquote{%
      \begin{itemize}
        \item
          The Integrated Authority File (GND) represents and describes 8,295,047 entities (people, corporations, conferences, geographical areas, technical terms, and works);
        \item
          The German National Library (DNB) provides descriptions of bibliographic resources. The dataset has 19,926,573 records of books, magazines, newspapers, sheet music, music recordings, audio books etc.;
        \item
          The German Union Catalogue of Serials (ZDB) describes newspapers, magazines, serial titles, yearbooks, etc. and contains 1,908,334 records;
        \item
          The Kalliope Union Catalog (KPE) is a collection of personal papers, manuscripts, and publishers’ archives, which consists of 26,752 records;
        \item
          The Newspaper Information System (ZeFYS) represents 2,596,641 digitized pages of historical newspapers and full texts;
        \item
          The Exile Press represents German-language exile journals between 1933 and 1945 and consists of 5,336 digitized pages.
      \end{itemize}
    }
  \item
    Proveana, Kulturgut-DB (recommended by Jo\"elle Weis)
  \item 
    criteria for analysis of data sources: see, e.g., \gls{SoNAR} grant proposal (appendices) 
  \item
    further criteria: arity of relationships, data formats, temporal data?
  \item
    also clarify: does the structure of these data sources support the model to be developed?
  \item
    focus the following considerations on a narrow choice of these data sources;
    the general approach to be developed should be largely independent on that concrete choice
  \item 
    in-depth analysis of data sources (e.g., overlap and differences) is worthwhile
    but must be deferred to future work
  \item
    DHa's remarks on GND vocabulary and richness of GND data:
    %
    \begin{itemize}
      \item
        \emph{Individualisierungsregeln}: relationships to other people are given in order to uniquely
        identify the current person; there is no obligation to record relationships
        (and GND does not claim to be an encyclopedia)
      \item
        relations are not normed:
        e.g., field \enquote{Funktionsbezeichnung}: \$4 code \enquote{Eigenschaft der Beziehung},
        specified by \$v \enquote{Freitext}
      \item
        professions are recorded in 678 \$b \enquote{biographische Anmerkung(en)} --
        historically in free text,
        more recently via link to the GND authority data for the respective profession (i.e., more standardised)
      \item
        Tp1 datasets: field 65 (6J?) \enquote{GND Systematik: eingeschr. Tätigkeit} (?),
        query level 1 datasets?
      \item
        $\leadsto$ \mybold{look/read all this up!}
    \end{itemize}
\end{itemize}

% -----------------------------------------------------------------
\section{Data Models}
\label{sec:data_models}

\begin{itemize}
  \item
    FRBR et al.
  \item 
    \mybold{OPAC or K10plus data model as ER diagram? Binary vs. \boldmath$n$-ary relations?
    $\leadsto$ check literature (possibly also papers on the FRBR model)!}
\end{itemize}

\dots

% -----------------------------------------------------------------
\section{Data Integration and Further Techniques}
\label{sec:data_integration}

\begin{itemize}
  \item 
    ETL (from grant proposal \gls{SoNAR}, p.~8, AP1-1)
  \item
    upper-level ontologies?
  \item 
    ontologies on research
  \item
    FRBRoo?
  \item
    RDF, SPARQL, N-Quads?
\end{itemize}


\dots

% -----------------------------------------------------------------
\section{Implications on Modelling}
\label{sec:implications_on_modelling}

\dots

\begin{itemize}
  \item
    \mybold{constants plus unary and binary relationships suffice (?)}
  \item 
    online vs.\ offline phase?
    %
    \begin{itemize}
      \item
        online: data source graph is implicit; data sources are queried \enquote{on the fly}
      \item
        offline: data source graph is generated explicitly (using data integration techniques)
        and updated in fixed intervals
    \end{itemize}
    %
\end{itemize}


