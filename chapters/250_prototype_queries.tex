% !TeX spellcheck = en_GB
% =================================================================
\chapter{Prototype Queries}
\label{chap:prototype_queries}

As a first step towards delineating the type of queries that should be covered by our approach,
we examine the query patterns introduced in Section~\ref{sec:background} more closely.
Those will serve as a point of reference for the analysis of the available data sources
in Chapter~\ref{chap:analysis},
and they will be generalised by the abstract framework developed in Chapter~\ref{chap:modelling}.
That framework will provide a rigorous definition of the queries that can be formulated
within it.
These query patterns from Section~\ref{sec:background} have been identified
as important examples by personal communication
with provenance researchers\todo{ask more experts},%
\footnote{Dietrich Hakelberg, Research Library Gotha of the University of Erfurt}
which justifies the choice of considering them as \emph{prototypical}.

Obviously, the question arises whether these prototypical query patterns are representative
for the range of queries that provenance researchers are interested in asking.
Answering that question would require systematic analysis of queries relevant to or useful for researchers.
Such an analysis would need to comprise an extensive interview study
based on very generic questions of a predominantly open-ended nature,
requiring a labour-intensive evaluation.
Given that this thesis focusses on the technical prerequisites and realisation,
such a study is clearly outside its scope. However, since the general framework that we will develop
is informed by the available data sources
and designed to cover a wide range of possible queries,
it is reasonable to assume that tools developed on its basis will be helpful for provenance researchers.
In subsequent work, when our method will hopefully have been implemented in a prototype tool,
the extent to which researchers' needs are served can be determined by means of a more focussed user study
with more specific questions, which in turn can inform possible extensions of the framework.

Let us now examine the prototypical query patterns more closely.
%
\begin{enumerate}
  \item[\exaquery{1}]
    Who read work $W$, in which manifestation and in which year?
  \item[\exaquery{2}]
    Which exemplars of work $W$ were passed from one of its owners to a collaborator (or a student)?
  \item[\exaquery{3}]
    What are the relationships between the recipients of work $W$
    (or of manifestation $M$ of $W$ or of exemplar $C$ of $W$, respectively)?
\end{enumerate}
%
In order instantiate one of them, let us fix work $W$ to be the seminal work \emph{De revolutionibus orbium coelestium}
(short: \emph{De revolutionibus}; English translation: \emph{On the Revolutions of the Heavenly Spheres}) by the astronomer Nicolaus Copernicus (1473–1543) \autocite{Kopernikus1543}.
We now consider the following concrete query, which is an instance of \exaquery{2}:
%
\begin{enumerate}
  \item[\exaquery{2$'$}]
%    Which exemplars of \emph{De revolutionibus} were owned by scientists who passed them on to a student?
    Which exemplars of \emph{De revolutionibus} were owned by some scientist who passed them on to a student?
\end{enumerate}
%
Let us first assume that a researcher wants to answer this query.
One obvious way to proceed would be as follows: first, our researcher finds exemplars of \emph{De revolutionibus} 
in online catalogues of libraries and library networks. For each such exemplar, they then inspect the provenance entries
that name owners who were people (not corporations). Finally, our researcher will have to find those names in databases such as
authority files or Wikidata and, for each entry, explore the specified profession (\enquote{scientist})
and relationships to other people (\enquote{student}).

For example, the online catalogue (OPAC) of the Gotha Research Library of the University of Erfurt (Forschungsbibliothek Gotha) lists two printed exemplars
of \emph{De revolutionibus} \autocite{OPACDeRev}.
%\footnote{%
%  \url{https://opac.uni-erfurt.de/DB=1/CMD?ACT=SRCHA&IKT=1016&SRT=YOP&TRM=tit+de+revolutionibus+and+per+kopernikus+and+jah+15**+and+bbg+a*}%
%}
One of those bears the signature \sig{Druck~4°~00466}, and its provenance entries name the following previous owners  \autocite{OPACDeRevPPN}:
%\footnote{%
%  \url{https://opac.uni-erfurt.de/LNG=EN/DB=1/XMLPRS=N/PPN?PPN=567506266}%
%}
%
\begin{itemize}
  \item
    Hieronymus Tilesius (1529–1566): autograph and date 1551
  \item
    NN: note, date 1553, name scraped out
  \item
    Johann Hommel (1518–1562), autograph
  \item
    Valentin Thau (1531–1575), note (greek proverb, possibly not denoting ownership)
  \item
    Ernest II, Duke of Saxe-Gotha-Altenburg (1745–1804): stamp/seal, initial
  \item
    Ducal Library, Gotha (a predecessor organisation of Gotha Research Library): stamp marking a duplicate
  \item
    Ernestine Gymnasium, Gotha: stamp
  \item
    Landesbibliothek Gotha: stamp
\end{itemize}
%
Our researcher can immediately decide that they can ignore the second entry (no name given) and the last three entries (corporations).
For the remaining four entries, our researcher follows the links given in the OPAC to the Integrated Authority File (GND) of the German National Library \autocite{DNBCatalogue}.
%\footnote{%
%  \url{https://katalog.dnb.de/EN/home.html?v=plist}%
%}
On inspection of these entries, it turns out that Ernest II was a regent and very probably not a scientist,
and that the other three people---Tilesius, Hommel, and Thau---had professions such as theologian,
mathematician, and astronomer, which qualifies them as scientists. Furthermore, Hommel's entry
contains a reference to Thau with the relationship \enquote{has student}
(and Thau's entry contains the inverse reference to Hommel).
From this reference, our researcher can conclude that two scientists in the teacher-student relation
have both possessed the exemplar. Since the data available does implies neither that Hommel passed the exemplar (directly) to Thau
nor that Thau was really an owner,
our researcher can formulate hypotheses based on the retrieved data and start an in-depth research.

The manual process that we have just described is cumbersome, laborious, and prone to errors and omissions for several reasons:
The search in library catalogues for exemplars of works and their provenances requires expert skills.
Catalogues with potential matches need to be selected manually,
and each catalogue needs to be queried individually, using its own search functionality and syntax. \todo{give examples of commonalities (OPAC) and differences (discovery vs. OPAC)?}
The traversal through all retrieved exemplars and the pursuit of each potential relevant provenance entry per exemplar 
multiplies the amount of manual work necessary.
Finally, it is not clear what an effective and efficient way to \enquote{explore} relationships would be:
while it is easy to find direct relationships such as \enquote{person $P_1$ is student of person $P_2$} in the view for a person's entry
in databases such as GND or Wikidata, there are relationships that cannot be discovered easily by hand,
e.g., \enquote{$P_1$ and $P_2$ are students of the same scholar}.

It should be clear from this example that the process of answering queries such as \exaquery{2$'$}
would strongly benefit from automated support, which could help reduce the amount of manual work, integrate heterogeneous data sources,
incorporate background knowledge (e.g., every mathematician is a scientist),
and discover relationships between entities that are not necessarily direct.
It is also clear that, due to the incompleteness of data, query answers obtained by manual or automated search
are always tentative in two complementary senses: First, if relevant provenance entries or relationships between people are missing
in the data sources, then neither a human nor a machine will be able to retrieve the respective answers. Hence, the sets of query answers obtained
are not necessarily complete.
Second, since the relationship \enquote{person $P_1$ passed a book on to another person $P_2$} is generally omitted from
provenance entries, a human or machine can only retrieve possible constellations such as the ones
given above. Hence, not all query answers need to be correct, but all answers can be considered
as informed hypotheses that initiate further research.

\todo[inline]{!! GND doesn't give \enquote{scientist} for, e.g., Thau -- it uses more specific relations such as \enquote{mathematician}, \enquote{astronomer}, \enquote{lawyer}. $\Rightarrow$ we need query answering with taxonomies/ontologies !!}

%
\begin{itemize}
  \item
    more instances of \exaquery{1} or \exaquery{3}?
  \item
    explain how \exaquery{1}--\exaquery{3} differ from each other (extend remarks from Section~\ref{sec:background}?)
\end{itemize}
