% !TeX spellcheck = en_GB
% =================================================================
\chapter{Conclusion}
\label{chap:conclusion}

% -----------------------------------------------------------------
\section{Summary of the Results}
\label{sec:summary}

In this thesis, we have examined the problem of modelling and automated retrieval of provenance relationships
that require information which is usually distributed over several data sources.
We have provided a review of the related literature,
a case study with example queries, and a review of available standards, data sources,
and techniques for data exchange and integration.
As the main contribution, we have developed an abstract and general model for queries,
data sources, and query answers, and we have devised a method for implementing this model
in an answer retrieval system, together with a discussion of various design choices.

In order to return to the initial research question and its subquestions,
we list them here once more.
\begin{mdframed}[
  linewidth=1pt,
  linecolor=black!50,
%  innertopmargin=-3pt,
  innerleftmargin=0pt,innerrightmargin=0pt,
  leftline=false,rightline=false
]
  \begin{enumerate}
    \item[\RQ\phantom{\mybold{1}}]
  %    \begin{mdframed}[roundcorner=10pt]
        \mybold{How can provenance relationships be modelled and automatically retrieved?}
  %    \end{mdframed}
  \end{enumerate}
\end{mdframed}
%
This question implies several subordinate questions:
%
\begin{mdframed}[
  linewidth=1pt,
  linecolor=black!50,
%  innertopmargin=-3pt,
  innerleftmargin=0pt,innerrightmargin=0pt,
  leftline=false,rightline=false
]
  \begin{enumerate}
    \item[\subquestion{1}]
      \emph{What is the state of research on infrastructures for the automated retrieval
      of provenance relationships? Which approach(es) is/are most closely related?}
    \item[\subquestion{2}]
      \emph{What are the general challenges for answering queries such as \exaquery{1}--\exaquery{9},
      and what are the specific challenges for an automated approach?}
    \item[\subquestion{3}]
      \emph{Which data sources, standards, data formats, and further tools
      are available for answering provenance queries using multiple, heterogeneous data sources?}
    \item[\subquestion{4}]
      \emph{Based on the structure of the identified data sources,
      how can data sources, queries, and query answers be modelled in an abstract framework?}
    \item[\subquestion{5}]
      \emph{What is a suitable method for retrieving provenance relationships
      in that framework?}
  \end{enumerate}
\end{mdframed}
%
We now summarise the insights obtained in this thesis relating to each of these questions.

\paragraph{\RQ[1] (Chapter~\ref{chap:context})}

Our literature review has shown that
social and historical network analysis are generic methods closely related to the goal of this thesis;
the general context of social and historical research produces research questions similar to
those underlying provenance research.
The SoNAR project is a rich source of relevant insights concerning the support for historical research
by research data infrastructures.
While network analysis aims at answering questions of a \enquote{global} nature,
our scenario covers more \enquote{local} questions.

Furthermore, our review has also shown that
linked (open) data and data integration techniques have been used extensively in the library domain,
and that the state of provenance indexing varies greatly between, and sometimes even within,
institutions and library networks.

\paragraph{\RQ[2] (Chapter~\ref{chap:prototype_queries})}

The general challenges for answering queries such as~\exaquery{1}--\exaquery{9}
include several sources for missing and spurious answers.
A prominent reason is the mismatch between natural-language terms for concepts or relations
and the terms that are 



Possible remedies include 

\paragraph{\RQ[3] (Chapter~\ref{chap:analysis})}

\dots

\paragraph{\RQ[4] (Chapter~\ref{chap:modelling})}

\dots

\paragraph{\RQ[5] (Chapter~\ref{chap:method})}

\dots

\paragraph{\RQ}

\dots


% -----------------------------------------------------------------
\section{Outlook}
\label{sec:outlook}

\dots


\todo[inline]{finish conclusion ::}


\begin{itemize}
  \item
    get back to initial research question and its subordinate questions
  \item
    Acknowledgements? See comments.
    % Herr Jäschke (Betreuung, wertvolles Feedback, Begutachtung)
    % Herr Rüter (dito)
    % Herr Hakelberg (Diskussionen, in denen u.a. die Grundidee zur Arbeit entstand)
    % Frau Petras (Hinweis auf Berichte SoNAR usw.)
    % Nadine, Frau Weis, Frau Scheibe (wertvolle Diskussionen und Feedback)
    % ((Jakob Voß (Information zu LOD-Export) ??))
\end{itemize}
