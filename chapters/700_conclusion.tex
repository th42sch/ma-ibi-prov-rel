% !TeX spellcheck = en_GB
% =================================================================
\chapter{Conclusion}
\label{chap:conclusion}

In this thesis, we have examined the problem of modelling and automated retrieval of provenance relationships
that require information which is usually distributed over several data sources.
We have provided a review of the related literature,
a case study with example queries, and a review of available standards, data sources,
and techniques for data exchange and integration.
As the main contribution, we have developed an abstract and general model for queries,
data sources, and query answers, and we have devised a method for implementing this model
in an answer retrieval system, together with a discussion of various design choices.

In order to return to the initial research question and its subquestions,
we list them here once more.
\begin{mdframed}[
  linewidth=1pt,
  linecolor=black!50,
%  innertopmargin=-3pt,
  innerleftmargin=0pt,innerrightmargin=0pt,
  leftline=false,rightline=false
]
  \begin{enumerate}
    \item[\RQ\phantom{\mybold{1}}]
  %    \begin{mdframed}[roundcorner=10pt]
        \mybold{How can provenance relationships be modelled and automatically retrieved?}
  %    \end{mdframed}
  \end{enumerate}
\end{mdframed}
%
This question implies several subordinate questions:
%
\begin{mdframed}[
  linewidth=1pt,
  linecolor=black!50,
%  innertopmargin=-3pt,
  innerleftmargin=0pt,innerrightmargin=0pt,
  leftline=false,rightline=false
]
  \begin{enumerate}
    \item[\subquestion{1}]
      \emph{Which data sources are available for answering provenance queries?}
  %  \item[\subquestion{2}]
  %    \emph{What are the overlap and differences between the contents of these data sources?}
  %    \todo{Restrict or omit later!}
    \item[\subquestion{2}]
      \emph{Which techniques and tools are available for integrating
      data from heterogeneous sources?}
    \item[\subquestion{3}]
      \emph{Based on the structure of the identified data sources,
      how can data sources, queries, and query answers be modelled in an abstract framework?}
    \item[\subquestion{4}]
      \emph{What is a suitable method for retrieving provenance relationships
      in that framework?}
  \end{enumerate}
\end{mdframed}



\dots

\todo[inline]{finish conclusion ::}


\begin{itemize}
  \item
    get back to initial research question and its subordinate questions
  \item
    Acknowledgements? See comments.
    % Herr Jäschke (Betreuung, wertvolles Feedback, Begutachtung)
    % Herr Rüter (dito)
    % Herr Hakelberg (Diskussionen, in denen u.a. die Grundidee zur Arbeit entstand)
    % Frau Petras (Hinweis auf Berichte SoNAR usw.)
    % Nadine, Frau Weis, Frau Scheibe (wertvolle Diskussionen und Feedback)
    % ((Jakob Voß (Information zu LOD-Export) ??))
\end{itemize}
