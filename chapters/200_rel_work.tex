% !TeX spellcheck = en_GB
% =================================================================
\chapter{Context}
\label{chap:rel_work}
\label{chap:context}

This thesis aims at providing support for provenance research, which is part of historical research.
As we have seen in Chapter~\ref{chap:intro}, networks that represent
people and their relationships play a central role.
Therefore, we begin our exploration of the scientific context of our work
with the topic of historical (social) network analysis.
We will collect insights from a recent research project
aimed at providing a research infrastructure for historical network analysis
and apply them to our research questions.

It has also become clear in Chapter~\ref{chap:intro} that the relevant data is distributed 
over a multitude of heterogeneous data sources.
Therefore we will review the literature on linked data, data integration,
and data provenance.

Furthermore/finally \dots

\todo[defer,inline]{\enquote{announce} knowledge graphs, state of provenance indexing, further research}

% -------------------------------------------------------------------
\section{Network Analysis and the SoNAR Project}
\label{sec:HNA+SoNAR}


% - - - - - - - - - - - - - - - 
\subsection{Network Analysis}

According to Jansen \autocite*{Jansen2003},
the notion of a \emph{network} is a central tool for the analysis
of modern societies in sociology, political science, and economics.
In these disciplines, networks of political actors, companies, or researchers,
among many others, are the subject of study.
Additionally, networks play an important role
in organisational psychology, biology, and web science \autocite{WikiSNAGerman,WikiNetworkAnalysis}.
%As we will see in the remainder of this section,
%networks and their analysis are essential tools
%in the context of the approach that we are going to develop.
%Therefore, the next 
The following
paragraph briefly summarises the main constituents of network analysis
as described by Jansen \autocite*{Jansen2003}.

\emph{Network analysis}
defines networks and provides a statistical toolset for describing and analysing them.
A network is a graph, which consists of nodes and edges.
Nodes represent actors, events or objects (e.g., people, companies, institutions, countries),
and edges represent relations between those (in the case of social networks, e.g.,
friendship, collaboration, or family relationships).
Networks are defined via certain modelling decisions
such as the commitment to a set of actors and relations that are of relevance,
or the decision whether relations are directed or undirected
and whether they are dichotomous (an edge is present or not)
or weighted by values representing frequencies or extent.
The tools provided by network analysis include
various metrics that apply to single nodes, pairs of nodes, paths in the network,
or the whole network; those often come in several variants.
Examples are connectivity, in-/outdegree, density,
size, cohesion, multiplexity, reachability, and path distance.
Since a graph can conveniently be represented by its adjacency matrix,
methods from matrix algebra are part of the toolset.

In applications where social structures are the object of study,
the term \emph{social network analysis (SNA)} is used,
and it predominantly refers to the \emph{method} of investigating
social interactions \autocite{Otte2002}.

In the context of historical research,
the notion of \emph{historical network analysis (HNA)}
has emerged recently; it focusses on the reconstruction of
historical networks \autocite{Menzel2020}.
A distinguishing feature of HNA is its retrospective view,
i.e., it is used to analyse historical data extracted
from available sources \autocite{Fangerau2022}.
Menzel et al.\ \autocite*{Menzel2020} name
\enquote{a lack of awareness with regard to the availability of suitable research data}
as a limiting factor of HNA;
in particular, a large amount of data is distributed over heterogeneous data sources.

% - - - - - - - - - - - - - - - 
\subsection{SoNAR (IDH)}

The project \enquote{SoNAR (IDH),
Interfaces to Data for Historical Social Network Analysis
and Research} \autocite{Bludau2020,Menzel2020,SoNAR},
which was funded by the German Research Foundation (DFG)
from 2019 to 2021,
developed approaches to building 
\enquote{an advanced research technology environment
supporting Historical Network Analysis and related research} \autocite{SoNAR}.
In the long-term vision, that environment is expected to
integrate data from a variety of existing repositories,
thus providing researchers with an extensive,
standardised, and interregional infrastructure for answering research questions
using methods from HNA.
According to the project proposal and the final reports \autocite{SoNARreports},
the project participants undertook
a systematic analysis of processing and managing the source data
for the purposes of HNA,
designed a model of a structured data analysis for HNA based on SNA methods,
developed visualization approaches and interfaces,
evaluated all components for a scientific usage,
and developed a concept for implementation and operation.

One of the project's four work packages (WPs)
addresses the development of the research technology environment
and its evaluation against real research questions.
In the following paragraphs,
we summarise the insights described in the final report on this WP \autocite{Fangerau2022}
that are relevant for the research goals of this thesis.

The report emphasises relationships and their evolution over time
as central aspects of historical research and acknowledges the increased 
importance of methods of HNA. In this context, the authors note that visualisation plays an important role
as a means for restructuring information and thus contributes to the progression of knowledge.
They also address the \emph{hermeneutic circle} \autocite{Malpas2015}
as a general approach to answering historical research questions:
this term refers to an iterative, circular work process
where the initial question guides the work with the source material
and is, in turn, readjusted based on the answers obtained.

In order to evaluate the research technology,
the project participants developed two research scenarios
with several research questions.
The report names four concrete questions that were considered central.
The nature of these questions is very general and \enquote{global}
in the sense that they refer to a large part of a network:
for example, they ask for the point in time when a scientific area became a separate discipline,
or for the role of academic and familial links in the course of a given scientist's career.

In this context, the report distinguishes two approaches to developing research questions
in HNA. One of these is the explorative approach,
where suitable questions are developed based on the available data.
According to the authors, this approach includes serendipity,
i.e., the hope that an inspection of the data helps identify unexpected
phenomena that contribute to the shaping of the research question.

The report uses the term \emph{data source} for original documents, media, and artefacts
from which \enquote{network-compatible} (i.e., essentially relational) data can be obtained;
data sources can be identified via \emph{repositories} of various kinds,
including library catalogues, archive portals, databases, and more.
The report highlights the advantages of using data from authority files
such as the GND (see Section~\ref{sec:background}), which are standardised,
subject to quality control, and essential for disambiguating personal names.
Moreover, authority files are freely accessible and provide information
on the provenance of their data.
%According to the authors,
%\enquote{the use of authority data from GND in SoNAR is a unique feature
%of the research technology environment and has, in principle, proved its worth}.

Concerning the repositories used,
the authors report the general problem of missing or erroneous data,
which leads to distorted answers to the questions, and which was not solved
systematically in the project. In particular, the GND is focused on German library holdings
and thus contains a disproportionately high amount of German-speaking persons.
The data is biased towards people who have published at all,
towards elite research, towards men (in particular among authors before the 1950s),
and against certain disciplines, such as economy.
The authors conclude that these restrictions significantly affect the answers to
historical research questions. As possible remedies, they include 
further repositories, such as 
the German Union Catalogue of Serials (ZDB),
the German National Library (DNB),
the Kalliope Union Catalogue (KPE),
and, tentatively, other authority files (in particular, Wikidata and VIAF);
we will get back to these in Section~\ref{sec:data_sources}.
However, the evaluation showed that even the sum of these repositories
does not provide enough data for a differentiated temporal
analysis; in particular, biographical data of the authors 
dominate, but those cover long time spans and do not provide sufficient evidence
for or against dynamic relationships such as teacher-student relationships.

%\enquote{allow a temporal differentiation over longer periods of time} (translated).

Concerning the selection of data, the report distinguishes between
direct and indirect information on relationships.
For example, if one is looking for support for the hypothesis of a social relationship
between two persons $A$ and $B$, then a family relationship between $A$ and $B$ explicit in the data
supports the hypothesis directly, while matching biographical dates for $A$ and $B$
are an indirect indicator. In the latter case, the relationship
explicit in the data (matching biographical dates) acts as a substitute
(\enquote{Stellvertreterinformation}) for the relationship under consideration.
The authors also address a special kind of indirect relationship,
called intellectual, which mostly involves a third person
or event, such as the co-citation of $A$'s and $B$'s texts by someone else.

A main constituent of the report is a catalogue of requirements
to HNA and the research technology. This catalogue consists of 61
requirements that reflect the specific needs of researchers.
The following of these requirements are relevant for our purposes
(descriptions translated from the original German text and slightly rephrased and/or shortened):
%
\begin{description}
  \item[R003]
    Researcher wants all information on relationships contained in the metadata of the resources
    to be shown as derived social relationships in the data model.
  \item[R004]
    Researcher wants all information on non-social
    relationships (see above) to be distinguished from direct social relationships.
  \item[R005]
    Researcher wants to connect non-social and non-explicit social relationships
    with further conditions (e.g., overlapping lifespans).
  \item[R006]
    Researcher wants to group people with comparable attributes (e.g., overlapping lifespans).
  \item[R016]
    Researcher wants a list of people connected to a given person.
  \item[R017]
    Researcher wants a list of corporate bodies connected to a given person.
  \item[R018]
    Researcher wants a list of people connected to a given person via some corporate body.
  \item[R029]
    Researcher wants, given a node, a description of the \enquote{potentially available attributes}
    and the provenance of the data.
  \item[R030]
    [dito, but with edges instead of nodes]
  \item[R043]
    [Researcher wants] a filter on attributes, e.g. biographical data, in order to
    restrict relationships to adulthood.
  \item[R061]
    Researcher wants to know which kinds of relationships are contained in the dataset.
\end{description}

% - - - - - - - - - - - - - - - 
\subsection{Graph Technologies for the Analysis of HNAs Using Heterogeneous Data Sources}

In the paper of the same name, \textcite{Menzel2020} report on
work in the SoNAR project concerning the 
\enquote{creation and operation of research infrastructure
for [HNA] based on heterogeneous data sources from cultural heritage institutions}.
In particular, the authors present insights on modelling and transformation
of heterogeneous data sources and on the design of visualisation for HNAs.
We summarise their insights on data selection and processing.

In the described approach, the authors integrate data from 6 heterogeneous repositories
into a large uniform graph that is stored in a graph database
and managed by the highly performant graph database management system Neo4j \autocite{Neo4j}.
The original repositories comprise an authority file (GND \autocite{GND}),
federated library catalogues (DNB, ZDB, KPE \autocite{DNBCatalogue,ZDB,KPE}),
and full texts (ZEFYS, Exile Press \autocite{ZEFYS,ExilePress});
altogether they contain some 30 million records.
The data model is restricted to 9 entity types (e.g., \term{Resource} or \term{PersonName})
and 9 very generic relation types (e.g., \term{RelationToPersonName}, \term{RelationToTopicTerm}).
There is also a distinction between explicit and implicit relations:
the former are present in the data, and the latter have to be inferred automatically via
certain \enquote{guidelines} and marked as such.

Since some of the above data sources include full texts,
it is necessary to link names of people, corporate bodies, or geographical entities
to entities in authority files. The use of the underlying technique,
\emph{named entity linking}, is discussed further by \textcite{Menzel2021}
in the context of SoNAR,
and by \textcite{Meiners2022} independently of SoNAR.

Among the challenges associated specifically with processing data,
\textcite{Menzel2020} name
the sheer size of the combined graph, which causes performance issues,
and the fact that the normalisation led to errors and inconsistencies.
More generic challenges include the integration of domain knowledge
and the adaptation of a more performant graph database engine (GraphDB),
which requires extensive remodelling.

\todo[think,inline]{Some more comments on NEL and choice of data sources from \autocite{Menzel2021, Meiners2022}? see Exposé}

% - - - - - - - - - - - - - - - 
\subsection{Insights Relevant for This Thesis}

We now put the insights from the SoNAR project described in the previous two subections
into relation with the research goals pursued in this thesis.
First, the analysis of historical networks is a broad concept, and SoNAR supports a setting
that is much more general than just provenance research.
Thus, the setting that we want to support in this thesis is a specific section of HNA and subsumed 
by the SoNAR setting.

The central role of relationships and of temporal aspects in historical research
have to be reflected in the model and method that we are going to develop.
Furthermore, the hermeneutic approach to historical research
and the explorative approach to answering general research questions 
need to be supported by our method. In fact, the repeated exploration of data
is exactly what we hope to support by enabling researchers to ask specific
queries over a combination of data sources and use the answers to obtain
\enquote{local} views of the whole network and thus inform the next steps in their research process.

\emph{Locality} is an important feature that distinguishes our approach from the SoNAR setting:
The exemplary research questions from the SoNAR report appear to be rather global
in the sense that they focus on wider areas in the network and that their analysis requires
\enquote{global} techniques such as graph metrics and statistical methods.
In contrast, we aim at answering more local queries
that are part of a larger research question and which 
focus on a single item/actor and its neighbourhood in the network
(e.g., owners of a book or students of a scholar, see the exemplary questions
in Section~\ref{sec:background}). Answering such queries
requires the retrieval of certain \enquote{candidates} from that neighbourhood
which provide the necessary evidence and which then can be used
to inform another iteration within the explorative approach.
In the light of these considerations, 
we should not adopt the a-priori restriction
to a fixed set of entity types or (generic) relation types;
instead, it should be to use arbitrary concepts and relations
from the data sources in queries and answers.
Hence, we will focus on the described local perspective
and leave the accommodation of global techniques for future work.
In a similar spirit, we will not study visualisation;
however, our envisaged tool can be used \emph{along with} visualisation tools
and provide entry points to a large network.

The construction of a uniform data source \emph{(graph)} that integrates all the available data
from the heterogeneous repositories appears to be an integral part of SoNAR,
according to \citeauthor*{Menzel2020}'s \autocite*{Menzel2020} discussion.
The obvious advantages include the interaction with a single database,
the use of a single data model, autonomous hosting
independently of the single repositories, and thus full control over the data.
Furthermore, inconsistencies
between the repositories can be resolved a-priori (at creation time),
and the created database is independent on possible changes in the data models of the repositories.
However, the static nature of the uniform database also has disadvantages:
a data model has to be fixed upfront, and ultimately needs to be adapted when
the models of the repositories change or new repositories are added;
regular updates are necessary when the contents of the repositories changes;
finally, as we have learnt above,
a large graph database uses a large amount of memory and requires a highly performant
database system.

As a result of this discussion,
we want our approach to exploit the benefits of a dynamic solution.
More precisely, while our abstract model will centre around a single data source (graph)
that represents the combination of the distributed repositories,
our method will abstain from constructing that graph explicitly and, instead,
answer queries \enquote{in place} over the original repositories.
This way, our approach will not depend on hosting capacities,
and it will always have direct access to the current content of the repositories.
On the downside, our approach will depend on external web services provided by the repositories,
and it will be sensitive to changes in their data models.
Our dynamic approach also requires that inconsistencies are resolved
a posteriori, i.e., every time a query answer is retrieved.
As a final advantage, the dynamic approach is flexible
in the sense that it can be applied to a static integrated data source as well,
thus benefiting from the advantages of the static approach.

A feature that we miss in the reports and publications of SoNAR is a rigorous definition of admissible queries
that specifies exactly which queries are in the scope and which are not.
We aim at providing such a rigorous definition for our setting
which, at the same time, should be as general as possible. We consider this rigorous definition
a main feature of our approach.

Data sources in the sense used in the SoNAR project report,
i.e., original documents, media, and further artefacts
are not part of our setting, as we do not consider the part of the research process
that consults those data sources. Our work focuses on the level of
what the SoNAR authors refer to as repositories,
and we will relate our choice of repositories in Section~\ref{sec:data_sources}
to \citeauthor*{Menzel2020}'s \autocite*{Menzel2020} selection.
In this thesis, we continue to use the term \enquote{data source} for what
SoNAR refers to as a repository,
and we speak of \emph{original data sources} to refer to original documents etc.

In order to allow researchers to refer to the original data sources,
it becomes clear that \emph{data provenance} plays a crucial role:
the answer to a specific query in our approach needs to refer to
the original data sources that provide evidence for the information contained in that answer.
This reference can point to original documents (SoNAR: data sources),
and/or to data sets in repositories. For a more detailed discussion of data provenance,
see Section~\ref{sec:data_provenance}.

The observed relevance of indirect or substitute information in the context of SoNAR 
seems important to keep in mind for our work as well: 
It has to be noted that not all information that one might be looking for is recorded
in the data. For example, there is no record of who \emph{read} a book,
and ownership has to be used as a substitute for readership although evidence
of ownership is only a necessary condition for readership and not a sufficient one---%
although one might argue that, in the case of, say, a scientist, ownership is very likely
to indicate readership.
Therefore, answers to questions
about readership retrieved on the grounds of recorded ownership require further interpretation and investigation
by the researcher who asked the question in the first place.
The example from the SoNAR report that uses \enquote{having the same biographical dates}
as a substitute for a social relationship is very similar, but in addition
the fact that two persons have the same biographical dates is not a relationship that is
given explicitly in the data but requires a certain amount of reasoning on several attributes of the entries
for the two persons.
This general observation has to be taken into account by our approach.

Finally, the insights on advantages and disadvantages of the repositories used in SoNAR
will affect our discussion of data sources in Section~\ref{sec:data_sources}.

% - - - - - - - - - - - - - - - 
\subsection{Further Initiatives Providing Research Infrastructures}

\textcite{Menzel2020} mention several projects that connect
decentralised and heterogeneous bibliographic data sources
and/or extract historical networks within the social sciences and humanities.
We briefly summarise some of these projects.

DARIAH-DE \autocite{DariahDE} is an association that develops
a digital research infrastructure for the arts and humanities
following the example of large digital research infrastructures for the natural sciences.
It comprises of 16 partner institutions from the academic sector
and is part of the European network DARIAH-EU \autocite{DariahEU}.
DARIAH-DE's services for research include
the development and hosting of services for data analysis and visualisation
\autocite{WikiDariahDE}.

Culturegraph \autocite{Vorndran2018,Culturegraph} is a service offered by the
German National Library (DNB) which aggregates bibliographic data 
from library networks in German-speaking countries,
from the DNB itself, and from the British National Library.
According to \textcite{Vorndran2018}, Culturegraph makes over 160 million data sets
available \enquote{for data analyses, evaluation of connections and statistical analyses}.
This is achieved, among other things,
by enrichment via external data sources such as GND \autocite{GND}
and ORCID \autocite{ORCID}, thus allowing for the identification
and disambiguation of persons.

\textcite{Menzel2020} further note \enquote{an increase in joint research projects
that are focused on the extraction of historical networks within the social sciences and the humanities.}
These projects include \emph{Six Degrees of Francis Beacon} \autocite{Warren2016,SixDegreesFB},
\emph{histoGraph} \autocite{Novak2014,histograph},
\emph{Issues with Europe} \autocite{IssuesWithEurope},
\emph{APIS} \autocite{Gruber2017,APIS},
and \emph{Gesellschaftliche Wissensproduktion in der Aufklärung} \autocite{Purschwitz2018}.
Judging from \citeauthor*{Menzel2020}'s \autocite*{Menzel2020} summary,
these projects put an emphasis on statistical methods,
visualisation, and/or exploration.


% -------------------------------------------------------------------
\section{Linked Data and Data Integration}
\label{sec:linked_data+integration}

% - - - - - - - - - - - - - - - 
\subsection{Basic Concepts}

\emph{Data Integration} refers to the problem of making a set of autonomous and heterogeneous data sources
uniformly accessible \autocite[p.6]{Doan2012}.
The current landscape of data sources includes highly structured data
represented and accessed using classical database techniques,
as well as more loosely structured data accessible via (Semantic) Web techniques.
Given this vast and diverse landscape, data integration is a complex problem,
and various techniques have been developed for this purpose.
The textbook by \textcite{Doan2012} provides a
comprehensive introduction to data integration.

In the remainder of this section,
we restrict our focus on those aspects of data integration
that are most relevant in the context of bibliographic data sources on the Web.

\emph{Linked data} \autocite{W3CLinkedData,Domingue2011} refers to data published on the World Wide Web that is structured and connected with other data.
Based on a small and uniform set of simple technologies
%---essentially HTTP URIs and the Resource Description Framework (RDF)---,
the linked data paradigm provides applications with access to \enquote{a global, unbounded dataspace} \autocite{Domingue2011}.
In the context of the Semantic Web \autocite{BernersLee2001,Marshall2003},
linked data is one of several developments aimed at making data on the Web more machine-understandable.

The main technologies underlying linked data are the following.
%
\begin{itemize}
  \item
    \emph{Uniform Resource Identifiers (URIs)} \autocite{RFC3986}:
    
    \par
    A URI is a string that is assigned to a physical or logical resource
    in order to identify that resource uniquely. URLs (Uniform Resource Locators)
    are a special case of URIs which additionally ensure that a resource
    can be located in a network and that information can be retrieved from it,
    where \enquote{network} includes, but is not restricted to, the World Wide Web (WWW).
    URIs are an integral part of RDF and the Web Ontology Language OWL (see below).
  \item
    the \emph{Hypertext Transfer Protocol (HTTP)} \autocite{HTTP}:
    
    \par
    HTTP is the fundamental protocol for data transfer and communication underlying the WWW.
  \item
    the \emph{Resource Description Framework (RDF)} \autocite{RDFPrimer}:
    
    \par
    RDF is \enquote{is a framework for expressing information about resources. Resources can be anything, including documents, people, physical objects, and abstract concepts}
    \autocite{RDFPrimer}.
    Its main ingredients are \emph{resources} and binary \emph{properties} for linking resources.
    RDF statements are \emph{triples} of the format \enquote{subject--predicate--object}
    consisting of a resource, a property, and a resource (or literal).
    Resources and properties are described using URIs.
    RDF thus provides a simple and flexible data model that is particularly useful for the publication
    and linking of data on the Web.
\end{itemize}
%
In the context of the Semantic Web, RDF and associated technologies
are important tools for the definition and use of ontologies.
For a definition of the term \enquote{ontology} in this setting, we cite \textcite{Horrocks2011}:
\enquote{A major feature of the Semantic Web is the ability to provide definitions for objects and types of objects that are accessible and manipulable from within the Semantic Web. In Computer Science, a collection of these sorts of definitions about a particular domain is called an ontology, although philosophers may (and probably will) have a different understanding of what constitutes an ontology.}

The following technologies associated with RDF and ontologies
are of interest for the purposes of this thesis:
%
\begin{itemize}
  \item
    \emph{RDF Schema (RDFS)} \autocite{RDFS}:
    
    \par
    RDFS is a specific RDF vocabulary which can be used for modelling
    and thus serves as a very basic ontology language.
    It provides terms for modelling, among others, classes, properties, and domain and range restrictions.
  \item
    the \emph{Web Ontology Language OWL} \autocite{OWLPrimer}:
    
    \par
    OWL is the ontology language recommended by the W3C.
    It is based on knowledge representation languages from the description logic (DL) family \autocite{Baader2017}.
    DLs have a well-defined syntax and model-theoretic semantics, which allows
    for unambiguous statements about the domain of interest
    and for reasoning (inference), i.e., the discovery of information that is implicit in the set of statements
    that constitute the ontology.
    The members of the DL family vary regarding their expressive power and,
    closely related, regarding the computational complexity of their basic reasoning problems.
    OWL 2, which is the current version of OWL,
    is based on an expressive DL where reasoning is still decidable and algorithms have been implemented
    that perform well on a wide range of real-world ontologies.
    Besides these knowledge representation languages,
    OWL includes infrastructure for interaction with ontologies and interoperability
    within the Web, such as Internationalized Resource Identifiers (IRIs),
    XML schema datatypes, import mechanisms, and many more.
  \item
    \emph{SPARQL} (Simple Protocol And RDF Query Language) \autocite{SPARQL}:
    
    \par
    SPARQL is a W3C-recommended query language for the Semantic Web.
    More precisely, it is a language for querying RDF graphs via pattern matching,
    whose syntax is based on the ISO standard SQL \autocite{SQL} for managing and querying relational databases.
    However, SPARQL is \enquote{far more powerful than SQL, since it is designed for the \emph{open}, \emph{decentralized}, and \emph{fluid} Web} \autocite{DellaValle2011}.
    SPARQL also provides a communication protocol for the interaction between clients and \emph{endpoints};
    answers are returned in RDF or XML.
    Additionally, SPARQL exploits inference mechanisms,
    i.e., answers may contain facts that are not explicitly stated in the original RDF graph
    but are obtained involving certain sets of inference rules, based on OWL or other standards.
\end{itemize}

\emph{Linked open data (LOD)} is linked data that is also \emph{open data},
i.e., accessible and shareable by everyone \autocite{WikiLinkedData}.

\todo[inline]{Also mention ontology-based data integration? See \url{https://en.wikipedia.org/wiki/Ontology-based_data_integration} and references therein} 



% - - - - - - - - - - - - - - - 
\subsection{Linked Data in the Library Domain}

The state of LOD in the library domain in 2013 is summarised by \textcite{Pohl2013}
in their introduction to the edited volume \enquote{(Open) Linked Data in Libraries} \autocite{Danowski2013}
as follows (translated from German and rephrased). Libraries and related organisations
began to experiment with linked data as early as 2008.
This was followed by linked-data initiatives by important players such as
OCLC, the Library of Congress (LC), and the DNB;
in particular, the LC started the initiative \enquote{Bibliographic Framework for the Digital Age} in 2011,
declaring the renunciation of the library-specific standards MARC21 and Z39.50,
and announcing the development of an infrastructure based on the Resource Description Framework (RDF) as a basic data model. The advantages of LOD for the library domain include
retrievability of, e.g., catalogue data by search engines,
mechanisms for permanently linking, e.g., hit lists and hit views,
the use of a much more flexible data model,
interoperability ensured by the use of open web standards,
and reusability via open licences.
A prominent example of this development is the provision of authority data as LOD,
which can be shared an reused on the Web:
e.g., authority data on person names from the GND has been used by Wikipedia since 2005
in order to provide links to further reading in Wikipedia articles, see also \autocite{Hengel2005}.
In addition to the GND, further providers have made authority data available as LOD,
among them LC and VIAF.

The state of L(O)D in libraries in 2013 was extensively surveyed in the special issue
\enquote{Linked Data, Semantic Web and Libraries} of the \emph{Journal of Library Metadata} \autocite{JLM13_2-3}.
In her preface to this issue, \textcite{Bair2013} refers to the challenges to linked data implementations
in general and in the library domain that were reported in previous work
\autocite{Bizer2009,Byrne2010,Gonzalez2011,Alemu2012},
and she emphasises that these challenges remain.
The special issue contains a survey on the perception of linked data in libraries,
as well as
\enquote{seven case studies of the experimental efforts of LAM institutions to use linked data to increase access to their collections and user services, plus four others that aim to increase awareness of and educate on key topics.}
The challenges highlighted in these articles fall into five areas,
among them data and schema mapping and interoperability, and data quality and trust.

Further work on L(O)D in the library domain is reported in the following publications.

\textcite{Burrows2021} describe an LOD model that links
three large manuscript databases
in the \emph{Mapping Manuscript Migrations (MMM)} project.
MMM aims at \enquote{providing large-scale analysis and visualizations of the history and provenance of medieval and Renaissance manuscripts.}

\textcite{LigiaTriques2022} give an overview of the data collection and integration technology
in the Digital Public Library of America (DPLA), with a focus on
interoperability and challenges for integrated data access.

\textcite{Freire2019} study metadata aggregation in the context of \emph{Europeana} \autocite{Isaac2012,Petras2017},
a web portal of the European Union that provides unified access to 
more than 56 million digital objects
from the collections of over 4000 cultural heritage institutions in Europe \autocite{Europeana}.
\citeauthor{Freire2019}'s case study involves two Dutch institutions as data provider and aggregator,
and aims at improved discoverability of the data.
The main challenge was the transition from traditional data models to flexible semantic data models.
The article presents the results of a requirement analysis,
the workflow that was developed, and its implementation.

\textcite{Ullah2018} review the current state of L(O)D in cataloguing
in the context of the trending transfer of bibliographic metadata
towards L(O)D by major libraries.
Their review provides an extensive survey of the recent literature.
The main findings include
the observations that L(O)D is becoming
\enquote{the mainstream trend in library cataloging especially in the major libraries and research projects of the world}
and that bibliographic metadata is becoming increasingly meaningful and reusable of through the emergence of Linked Open Vocabularies.

\textcite{Hauser2014} gives an overview of the linked data service
of the German National Library (DNB),
which started in 2010 with the publication of GND authority data as linked data.
Since 2012, the DNB has been using and maintaining its own GND ontology \autocite{GNDOntology}
in order to bridge the gap from the original MARC-based data model to a flexible, open data model.
In particular, the GND ontology provides a vocabulary for describing entities such as persons, corporate bodies, and places,
thus allowing to disambiguate and link these entities.

% - - - - - - - - - - - - - - - 
\subsection{Some Recent Developments}

\textcite{Boumechaal2023} developed a framework for transforming queries formulated in natural language (NL)
to SPARQL queries in order to \enquote{query linked and heterogeneous semantic data on the web}.
In contrast to most of the previous work, their approach focuses on complex queries
involving negation, numbers, superlatives, and comparative adjectives.
The aspect of translation from NL to SPARQL (or any other formal query language)
is important in the context of historical research, where the use of a formal query language
would be a barrier for most users. This aspect is, however, not in the scope of this thesis
and should be considered in future work.


% - - - - - - - - - - - - - - - 
\subsection{Data Provenance}
\label{sec:data_provenance}

*** \autocite{Eckert2012,Eckert2013}
*** PoDI Chapter 14?
*** more from Data Provenance folder

\dots


% -------------------------------------------------------------------
\section{Knowledge Graphs}
\label{sec:KGs}

\dots


% -------------------------------------------------------------------
\section{Further Relevant Work}
\label{sec:further}


*** state of provenance indexing (with authority data): \autocite{Hakelberg2016}


\dots

