% !TeX spellcheck = en_GB
% =================================================================
\chapter{Context}
\label{chap:rel_work}
\label{chap:context}

\todo[inline]{TODO: import from \emph{Exposé} and extend}

\begin{itemize}
  \item
    state of provenance indexing (with authority data): \autocite{Hakelberg2016}
  \item
    data provenance: \autocite{Eckert2012}
  \item
    Named Entity Linking and Recognition: \autocite{Menzel2021, Meiners2022}, see Exposé
  \item 
    Historical Network Analysis: \autocite{Menzel2020}, see Exposé
  \item
    more from project SoNAR (IDH): Abstract 2020 and references therein; more recent publications?
\end{itemize}

% -------------------------------------------------------------------
\section{Network Analysis and the SoNAR Project}
\label{sec:HNA+SoNAR}

*** Network analysis, SNA, HNA:
* define \enquote{network}
* methods of SNA
* importance of HNA for historical research

The project \enquote{SoNAR (IDH),
Interfaces to Data for Historical Social Network Analysis
and Research} \autocite{Bludau2020,Menzel2020,SoNAR},
which was funded by the German Research Foundation (DFG)
from 2019 to 2021,
developed approaches to building 
\enquote{an advanced research technology environment
supporting Historical Network Analysis and related research} \autocite{SoNAR}.
In the long-term vision, that environment is expected to
integrate data from a variety of existing repositories,
thus providing researchers with an extensive,
standardised, and interregional infrastructure for answering research questions
using methods from HNA.
According to the project proposal and the final reports \autocite{SoNARreports},
the project participants undertook
a systematic analysis of processing and managing the source data
for the purposes of HNA,
designed a model of a structured data analysis for HNA based on SNA methods,
developed visualization approaches and interfaces,
evaluated all components for a scientific usage,
and developed a concept for implementation and operation.

One of the project's four work packages (WPs)
addresses the development of the research technology environment
and its evaluation against real research questions.
In the following paragraphs,
we summarise the insights described in the final report on this WP \autocite{Fangerau2022}
that are relevant for the topic of this thesis.

The report emphasises relationships and their evolution over time
as central aspects of historical research and acknowledges the increased 
importance of methods of HNA. In this context, the authors note that visualisation plays an important role
as a means for restructuring information and thus contributes to the progression of knowledge.
They also address the \emph{hermeneutic circle} \autocite{Malpas2015}
as a general approach to answering historical research questions:
this term refers to an iterative, circular work process
where the initial question guides the work with the source material
and is, in turn, readjusted based on the answers obtained.

In order to evaluate the research technology,
the project participants developed two research scenarios
with several research questions.
The report names four concrete questions that were considered central.
The nature of these questions is very general and \enquote{global}
in the sense that they refer to a large part of a network:
for example, they ask for the point in time when a scientific area became a separate discipline,
or for the role of academic and familial links in the course of a given scientist's career.

In this context, the report distinguishes two approaches to developing research questions
in HNA. One of these is the explorative approach,
where suitable questions are developed based on the available data.
According to the authors, this approach includes serendipity,
i.e., the hope that an inspection of the data helps identify unexpected
phenomena that contribute to the shaping of the research question.

The report uses the term \emph{data source} for original documents, media, and artefacts
from which \enquote{network-compatible} (i.e., essentially relational) data can be obtained;
data sources can be identified via \emph{repositories} of various kinds,
including library catalogues, archive portals, databases, and many more.
The report highlights the advantages of using data from authority files
such as the GND (see Section~\ref{sec:background}), which are standardised,
subject to quality control, and essential for disambiguating personal names.
Moreover, authority files are freely accessible and provide information
on the provenance of their data. According to the authors,
\enquote{the use of authority data from GND in SoNAR is a unique feature
of the research technology environment and has, in principle, proved its worth}.

Concerning the repositories used,
the authors report the general problem of missing or erroneous data,
which leads to distorted answers to the questions, and which could not been solved
systematically. In particular, the GND is focussed on German library holdings
and thus contains a disproportionately high amount of German-speaking persons.
The data is biased towards people who have published at all,
towards elite research, towards men (in particular among authors before the 1950s),
and against certain disciplines, such as economy.
The authors conclude that these restrictions significantly effect the answers to
historical research questions. As possible remedies, they include 
further repositories, such as 
the German Union Catalogue of Serials (ZDB),
the German National Library (DNB),
the Kalliope Union Catalogue (KPE),
and, tentatively, other authority files (in particular, Wikidata and VIAF);
we will get back to these in Section~\ref{sec:data_sources}.
However, the evaluation showed that even the sum of these repositories
does not provide enough data for a differentiated temporal
analysis; in particular, biographical data of the authors 
dominate, but those cover long time spans and do not provide sufficient evidence
for or against dynamic relationships such as teacher-student relationships.

%\enquote{allow a temporal differentiation over longer periods of time} (translated).

Concerning the selection of data, the report distinguishes between
direct and indirect information on relationships.
For example, if one is looking for support for the hypothesis of a social relationship
between two persons $A$ and $B$, then a family relationship between $A$ and $B$ explicit in the data
supports the hypothesis directly, while matching biographical dates for $A$ and $B$
are an indirect indicator. In the latter case, the relationship
explicit in the data (matching biographical dates) acts as a substitute
(\enquote{Stellvertreterinformation}) for the relationship under consideration.
The authors also address a special kind of indirect relationship,
called intellectual, which mostly involves a third person
or event, such as the co-citation of $A$'s and $B$'s texts by someone else.

A main constituent of the report is a catalogue of requirements
to HNA and the research technology. This catalogue consists of 61
requirements that reflect the specific needs of researchers.
The following of these requirements are relevant for our purposes
(descriptions translated from the original German text and slightly rephrased and/or shortened):
%
\begin{description}
  \item[R003]
    Researcher wants all information on relationships contained in the metadata of the resources
    to be shown as derived social relationships in the data model.
  \item[R004]
    Researcher wants all information on non-social
    relationships (see above) to be distinguished from direct social relationships.
  \item[R005]
    Researcher wants to connect non-social and non-explicit social relationships
    with further conditions (e.g., overlapping lifespans).
  \item[R006]
    Researcher wants to group people with comparable attributes (e.g., overlapping lifespans).
  \item[R016]
    Researcher wants a list of people connected to a given person.
  \item[R017]
    Researcher wants a list of corporate bodies connected to a given person.
  \item[R018]
    Researcher wants a list of people connected to a given person via some corporate body.
  \item[R029]
    Researcher wants, given a node, a description of the \enquote{potentially available attributes}
    and the provenance of the data.
  \item[R030]
    [dito, but with edges instead of nodes]
  \item[R043]
    [Researcher wants] a filter on attributes, e.g. biographical data, in order to
    restrict relationships to adulthood.
  \item[R061]
    Researcher wants to know which kinds of relationships are contained in the dataset.
\end{description}

We now discuss how these insights from the SoNAR project relate to, and inform, our work on the topic of this thesis.
First, the analysis of historical networks is a broad concept, and SoNAR supports a setting
that is much more general than just provenance research.
Thus, the setting that we want to support in this thesis is a specific section of HNA and subsumed 
by the SoNAR setting.

The central role of relationships and of temporal aspects in historical research
have to be reflected in the model and method that we are going to develop.
Furthermore, the hermeneutic approach to historical research
and the explorative approach to answering general research questions 
need to be supported by our method. In fact, the repeated exploration of data
is exactly what we hope to support by enabling researchers to ask specific
queries over a combination of data sources and use the answers to obtain
\enquote{local} views of the whole network and thus inform the next steps in their research process.

Locality is a feature that distinguishes our approach from the SoNAR setting:
while the exemplary research questions for which SoNAR caters are mostly of a global nature,
our exemplary queries are more local in the sense that they focus on
a single item, person, or corporate body and its neighbourhood in the network
(e.g., owners of a book or students of a scholar). 
In order to fit the scope of a master's thesis,
we consider it prudent to restrict this work to a local setting,
and leave global features (including, among others, network metrics)
for future work. The same holds for visualisation, which is an important component
of SoNAR.

A feature that we miss in the reports and publications of SoNAR is a rigorous definition of admissible queries
that specifies exactly which queries are in the scope and which are not.
We aim at providing such a rigorous definition for our setting
which, at the same time, should be as general as possible. We consider this rigorous definition
a main feature of our approach.

Data sources in the sense used by SoNAR, i.e., original documents, media, and further artefacts
are not part of our setting, as we do not consider the part of the research process
that consults those data sources. Our approach focusses on the level of
what the SoNAR authors refer to as repositories.
In order to allow researchers to refer to the original data sources,
it becomes clear that \emph{data provenance} plays a crucial role:
the answer to a specific query in our setting needs to refer to
the original data sources that provide evidence for the information contained in that answer.
This reference can point to original documents (SoNAR: data sources),
and/or to data sets in repositories. For a more detailed discussion of data provenance,
see Section~\ref{sec:data_provenance}.
In this thesis, we continue to use the term \enquote{data source} for what
SoNAR refers to as a repository.

Another observation in the context of SoNAR seems important to keep in mind
when designing our method: the relevance of indirect or substitute information.
It has to be noted that not all information that one might be looking for is recorded
in the data. For example, there is no record of who \emph{read} a book,
and ownership has to be used as a substitute for readership although evidence
of ownership is only a necessary condition for readership and not a sufficient one---%
although one might argue that, in the case of, say, a scientist, ownership is very likely
to indicate readership.
Therefore, answers to questions
about readership retrieved on the grounds of recorded ownership require further interpretation and investigation
by the researcher who asked the question in the first place.
The example from the SoNAR report that uses \enquote{having the same biographical dates}
as a substitute for a social relationship is very similar, but in addition
the fact that two persons have the same biographical dates is not a relationship that is
given explicitly in the data but requires a certain amount of reasoning on several attributes of the entries
for the two persons.
This general observation has to be taken into account by our method.


\todo[inline]{In general, look for a uniform word to refer to our approach and use it from the start!}

*** Insights for our topic:
* SoNAR: 
  Pro authority files, con GND, pro additional repositories/Wikidata/VIAF, GND \enquote{Alleinstellungsmerkmal}; 
  Some requirements can be addressed directly with our queries, e.g., R006!
  R016 seems to require the \enquote{$\top$-role} in queries!
  Edge weights;
  E.g. R029, 30, 61: explorative (for every node/edge the available attributes) vs. ??? (ask query knowing those attributes)



\dots

% -------------------------------------------------------------------
\section{Linked Data and Data Integration}
\label{sec:linked_data+integration}

\dots


% -------------------------------------------------------------------
\section{Data Provenance}
\label{sec:data_provenance}

\dots


% -------------------------------------------------------------------
\section{Knowledge Graphs}
\label{sec:KGs}

\dots


% -------------------------------------------------------------------
\section{Further Relevant Work}
\label{sec:further}

\dots


