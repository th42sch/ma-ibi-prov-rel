% !TeX spellcheck = en_GB
% =================================================================
\chapter{Context}
\label{chap:rel_work}
\label{chap:context}

\todo[inline]{TODO: import from \emph{Exposé} and extend}

\begin{itemize}
  \item
    state of provenance indexing (with authority data): \autocite{Hakelberg2016}
  \item
    data provenance: \autocite{Eckert2012}
  \item
    Named Entity Linking and Recognition: \autocite{Menzel2021, Meiners2022}, see Exposé
  \item 
    Historical Network Analysis: \autocite{Menzel2020}, see Exposé
  \item
    more from project SoNAR (IDH): Abstract 2020 and references therein; more recent publications?
\end{itemize}

% -------------------------------------------------------------------
\section{Network Analysis and the SoNAR Project}
\label{sec:HNA+SoNAR}

*** Network analysis, SNA, HNA:
* methods of SNA
* importance of HNA for historical research

The project \enquote{SoNAR (IDH),
Interfaces to Data for Historical Social Network Analysis
and Research} \autocite{Bludau2020,Menzel2020,SoNAR},
which was funded by the German Research Foundation (DFG)
from 2019 to 2021,
developed approaches to building 
\enquote{an advanced research technology environment
supporting Historical Network Analysis and related research} \autocite{SoNAR}.
In the long-term vision, that environment is expected to
integrate data from a variety of existing repositories,
thus providing researchers with an extensive,
standardised, and interregional infrastructure for answering research questions
using methods from HNA.
According to the project proposal and the final reports \autocite{SoNARreports},
the project participants undertook
a systematic analysis of processing and managing the source data
for the purposes of HNA,
designed a model of a structured data analysis for HNA based on SNA methods,
developed visualization approaches and interfaces,
evaluated all components for a scientific usage,
and developed a concept for implementation and operation.

On of the project's four work packages (WPs)
addresses the development of the research technology environment
and its evaluation against real research questions.


the one most relevant for the 

The project was divided into four work packages (WPs).
The WP m,ost re



\dots

% -------------------------------------------------------------------
\section{Linked Data and Data Integration}
\label{sec:linked_data+integration}

\dots


% -------------------------------------------------------------------
\section{Data Provenance}
\label{sec:data_provenance}

\dots


% -------------------------------------------------------------------
\section{Knowledge Graphs}
\label{sec:KGs}

\dots


% -------------------------------------------------------------------
\section{Further Relevant Work}
\label{sec:further}

\dots


