% !TeX spellcheck = en_GB
% =================================================================
\chapter{Context}
\label{chap:rel_work}
\label{chap:context}

\todo[inline]{TODO: import from \emph{Exposé} and extend}

\begin{itemize}
  \item
    state of provenance indexing (with authority data): \autocite{Hakelberg2016}
  \item
    data provenance: \autocite{Eckert2012}
  \item
    Named Entity Linking and Recognition: \autocite{Menzel2021, Meiners2022}, see Exposé
  \item 
    Historical Network Analysis: \autocite{Menzel2020}, see Exposé
  \item
    more from project SoNAR (IDH): Abstract 2020 and references therein; more recent publications?
\end{itemize}

% -------------------------------------------------------------------
\section{Network Analysis and the SoNAR Project}
\label{sec:HNA+SoNAR}

*** Network analysis, SNA, HNA:
* methods of SNA
* importance of HNA for historical research

The project \enquote{SoNAR (IDH),
Interfaces to Data for Historical Social Network Analysis
and Research} \autocite{Bludau2020,Menzel2020,SoNAR},
which was funded by the German Research Foundation (DFG)
from 2019 to 2021,
developed approaches to building 
\enquote{an advanced research technology environment
supporting Historical Network Analysis and related research} \autocite{SoNAR}.
In the long-term vision, that environment is expected to
integrate data from a variety of existing repositories,
thus providing researchers with an extensive,
standardised, and interregional infrastructure for answering research questions
using methods from HNA.
According to the project proposal and the final reports \autocite{SoNARreports},
the project participants undertook
a systematic analysis of processing and managing the source data
for the purposes of HNA,
designed a model of a structured data analysis for HNA based on SNA methods,
developed visualization approaches and interfaces,
evaluated all components for a scientific usage,
and developed a concept for implementation and operation.

One of the project's four work packages (WPs)
addresses the development of the research technology environment
and its evaluation against real research questions.
In the following paragraphs,
we summarise the insights described in the final report on this WP \autocite{Fangerau2022}
that are relevant for the topic of this thesis.

The report emphasises relationships and their evolution over time
as central aspects of historical research and acknowledges the increased 
importance of methods of HNA. In this context, the authors note that visualisation plays an important role
as a means for restructuring information and thus contributes to the progression of knowledge.
They also address the \emph{hermeneutic circle} \autocite{Malpas2015}
as a general approach to answering historical research questions:
this term refers to an iterative, circular work process
where the initial question guides the work with the source material
and is, in turn, readjusted based on the answers obtained.

The report identifies two approaches to developing research questions
in HNA. One of these is the explorative approach,
where suitable questions are developed based on the available data.
According to the authors, this approach includes serendipity,
i.e., the hope that an inspection of the data helps identify unexpected
phenomena that contribute to the shaping of the research question.

The report uses the term \enquote{data source} for original documents, media and artefacts
from which network-compatible (i.e., essentially relational) data can be obtained;
data sources can be identified via repositories of various kinds,
including library catalogues, archive portals, databases, and many more.

*** Elaborate on choice of repositories (from 3rd part of the report)

Concerning the selection of data, the report distinguishes between
direct and indirect information on relationships.
For example, if one is looking for support for the hypothesis of a social relationship
between two persons $A$ and $B$, then a family relationship between $A$ and $B$ explicit in the data
supports the hypothesis directly, while matching biographical dates for $A$ and $B$
are an indirect indicator. In the latter case, the relationship
explicit in the data (matching biographical dates) acts as a substitute
(\enquote{Stellvertreterinformation}) for the relationship under consideration.
The authors also address a special kind of indirect relationship,
called intellectual, which mostly involves a third person
or event, such as the co-citation of $A$'s and $B$'s texts by someone else.

The report highlights the advantages of using data from authority files
such as the GND (see Section~\ref{sec:background}), which are standardised,
subject to quality control, and essential for disambiguating personal names.
Moreover, authority files are freely accessible and provide information
on the provenance of their data.

A main constituent of the report is a catalogue of requirements
to HNA and the research technology. This catalogue consists of 61
requirements that reflect the specific needs of researchers.
The requirements relevant for our purposes are the following
(description translated from the original German text and slightly rephrased/shortened):
%
\begin{description}
  \item[R003]
    Researchers want all information on relationships contained in the metadata of the resources
    to be shown as derived social relationships in the data model.
  \item[R004]
    Researchers want all information on non-social
    relationships (see above) to be distinguished from direct social relationships.
  \item[R005]
    Researchers want to connect non-social and non-explicit social relationships
    with further conditions (e.g., overlapping lifespans).
  \item[R006]
    Researchers want to group people with comparable attributes (e.g., overlapping lifespans).
  \item[R016]
    Researchers want a list of people connected to a given person.
  \item[R017]
    Researchers want a list of corporate bodies connected to a given person.
  \item[R018]
    Researchers want a list of people connected to a given person via some corporate body.
  \item[...]
    \todo[inline]{Continue checking with R020!}
\end{description}



*** Catalogue of requirements (R001--062); address R003 and possibly others!



*** Insights for our topic:
* delineation from pure (H)NA -- focus on more \enquote{local} questions, but (H)NA is possible
* SoNAR: relationships, temporal aspects, hermeneutic circle; visualisation out of scope, but enabled!
  Explorative approach; our notion of a data source (abstract from original documents, focus on machine-readable data in a standardised format).
  Indirect information (substitute); even \enquote{having the same biographical data} already involves some interpretation of, or reasoning on, the pure data.
  Authority files; data provenance;
  Some requirements can be addressed directly with our queries, e.g., R006!
  R016 seems to require the \enquote{$\top$-role} in queries!



\dots

% -------------------------------------------------------------------
\section{Linked Data and Data Integration}
\label{sec:linked_data+integration}

\dots


% -------------------------------------------------------------------
\section{Data Provenance}
\label{sec:data_provenance}

\dots


% -------------------------------------------------------------------
\section{Knowledge Graphs}
\label{sec:KGs}

\dots


% -------------------------------------------------------------------
\section{Further Relevant Work}
\label{sec:further}

\dots


