% !TeX spellcheck = en_GB
% =================================================================
\chapter{Automated Retrieval of Provenance Relationships}
\label{chap:retrieval}

\begin{itemize}
  \item
    develop method for answering queries in the model just defined
  \item
    data source $=$ union of all available data sources (see §\ref{subsec:data_sources})
  \item
    online vs.\ off{}line scenario:
    %
    \begin{itemize}
      \item
        off{}line: 
        
        data source graph is generated explicitly (using data integration techniques, e.g., LD)
        and updated in fixed intervals
                
        at runtime: provenance query is formulated as a \gls{SPARQL} query and posed against the graph
        
        learn from SoNAR insights
        
        reduction to SQL? use DB systems
      \item
        online (preferred?):
        
        data source graph is implicit; data sources are queried \enquote{on the fly}
        (this is the preferred scenario here; see delineation from SoNAR in §\ref{sec:HNA+SoNAR})
        
        data sources need to be defined and
        interactions with them programmed in advance
        
        at runtime: decompose query into single \gls{SPARQL} (sub)queries and pose them against several data sources,
        potentially iteratively;
        compose final answer from the partial answers
        $\leadsto$ demonstrate this for example query/ies!
      \item
        don't commit to online scenario (check mentions of \enquote{method} in text and rephrase if necessary)
      \item
        discuss pros and cons; connect to discussion in §\ref{sec:HNA+SoNAR}; reuse:
        
        More precisely, while our abstract model will centre around a single data source (graph)
        that represents the combination of the distributed repositories,
        our method will abstain from constructing that graph explicitly and, instead,
        answer queries \enquote{in place} over the original repositories.
        This way, our approach will not depend on hosting capacities,
        and it will always have direct access to the current content of the repositories.
        On the downside, our approach will depend on external web services provided by the repositories,
        and it will be sensitive to changes in their data models.
        Our dynamic approach also requires that inconsistencies are resolved
        a posteriori, i.e., every time a query answer is retrieved.
        As a final advantage, the dynamic approach is flexible
        in the sense that it can be applied to a static integrated data source as well,
        thus benefiting from the advantages of the static approach.
  \end{itemize}
    %
  \item
    get back to Section~\ref{sec:quality_of_answers}:
    %
    \begin{itemize}
      \item
        missing vs.\ spurious answers
      \item
        overapproximation
      \item
        \emph{reasoning}: ontologies and related technology -- also explain why
        missing knowledge cannot be added upfront, e.g., like this:
        
        Clearly, it would not be advisable to attempt adding all implicit knowledge to the data
        because that would massively inflate the data,
        as most terms have several superordinate concepts or relations and,
        furthermore, implicit knowledge is not restricted to taxonomic knowledge.
      \item
        \emph{hypothesising}
    \end{itemize}
    %
  \item
    get back to SoNAR discussion (Section~\ref{subsec:insights_from_SoNAR})
  \item
    get back to substitute information (from SoNAR and Section~\ref{sec:quality_of_answers})
  \item
    indirect relationships
\end{itemize}
