% !TeX spellcheck = en_GB
% =================================================================
%\chapter{Automated Retrieval of Provenance Relationships}
\chapter{A Method for Retrieving Provenance Relationships}
\label{chap:retrieval}
\label{chap:method}

In the previous chapter, 
we have developed a model for answering provenance queries.
It is the vision of this thesis that, ultimately,
this model is implemented 
in a retrieval system
that answers provenance queries conforming to our model.
The primary user group for that system consists of provenance researchers
and historians who want to investigate, for example, the history of
items and collections or the social networks of historically important figures.

In this chapter, we design a method that serves
as a the general basis for a retrieval system.
This method abstracts away from specific details such as 
programming paradigms, languages, libraries, or data structures;
instead, it represents a high-level specification
for a detailed future implementation.

In the discussion of the insights from the \gls{SoNAR} project
in Section~\ref{subsec:insights_from_SoNAR},
we have suggested a distinction between a dynamic and a static setting.
The method that we are going to develop should be flexible
and support both settings.
For this purpose, we make this distinction more explicit.
The \emph{static setting} resembles the setting used in the SoNAR project:
a large uniform data source \emph{(graph)} is constructed
from the various identified data sources \emph{(repositories)},
using a uniform data model and that integrates the heterogeneous data models
of the repositories in some way. This graph is 
used as the sole source for computing query answers.
In contrast, the \emph{dynamic setting} foregoes
the explicit construction of a uniform graph and instead
regards a collection of remote repositories as a representation of that
graph. In order to answer a query, the system identifies
suitable repositories, computes partial answers using the repositories' interfaces,
and composes the final answer from the partial answers.

Our method will consist of two main phases, the second of which is divided
into three subphases.
We describe these phases in Section~\ref{sec:phases}.
In Section~\ref{sec:dynamic_vs_static}, we take a closer look at the two settings.
Finally, we will discuss several aspects of the system's interaction with its users
in Section~\ref{sec:user_interaction}.

% -------------------------------------------------------------------
\section{The Main Phases}
\label{sec:phases}

\newlength{\mainphases}\settowidth{\mainphases}{\fns \textbf{Configuration Phase}~}%
\newlength{\subphases}\settowidth{\subphases}{\fns Answer Presentation~}%
%\begin{figure}[ht]
\begin{wrapfigure}[16]{o}{5.2cm}
  \centering
  \begin{tikzpicture}[
    >=Latex,
    every node/.style={on grid,rectangle,rounded corners=1mm,draw=black,fill=lightgreen,thick,inner sep=1.5mm},
    every edge/.style={draw=black,thick}
  ]
    
    % the 3 subphases of Runtime
    \begin{scope}[
      every node/.append style={text width=\subphases,align=center}
    ]
      \node                    (QF) {\fns\mystrut Query Formulation};
      \node [below=10mm of QF] (QP) {\fns\mystrut Query Processing};
      \node [below=10mm of QP] (AP) {\fns\mystrut Answer Presentation};
    \end{scope}

    % frames for the 2 main phases
    \begin{scope}[
      every node/.append style={fill=none}
    ]      
      \node [fit={($(QF.north west) + (-0.3, 1.0)$) ($(AP.south east) + (0.7, -0.55)$)},inner sep=0mm] (RPF) {};
      \node [fit={($(QF.north west) + (-0.3, 1.5)$) ($(QF.north east) + (0.7, 2.1)$)},inner sep=0mm]  (PPF) {};
    \end{scope}
      
    % labels for the 2 main phases
    \begin{scope}[
      every node/.append style={draw=none,fill=none}
    ]      
      
      \node [right=0mm of PPF.west]                     (PP) {\fns\mystrut\textbf{Configuration Phase}};
      \node [below right=0mm and 0mm of RPF.north west] (RP) {\fns\mystrut\textbf{Runtime Phase}};
    \end{scope}

    \begin{scope}[%
      every node/.style={draw=none,fill=none,inner sep=.2mm}
    ]
      
      \path[->]
        % the 3 main Group-1 relations
        (QF) edge (QP)
        (QP) edge (AP)
        (PPF) edge (RPF)
      ;
      
      \draw[->, rounded corners=2.8mm, draw=black, thick]
        (AP.south) |- ($(AP.south east) + (0.4, -0.3)$) -- ($(QF.north east) + (0.4, 0.45)$) -| (QF.north);
    \end{scope}
  \end{tikzpicture}
  
  \caption{Main phases of the retrieval method}
  \label{fig:method_phases}
\end{wrapfigure}
%\end{figure}


Our method consists of a \emph{Configuration Phase}, which
needs to be carried out before users can interact with the system,
and a \emph{Runtime Phase}, where the actual user interaction takes place.
The latter is divided into three subphases as shown in Figure~\ref{fig:method_phases},
which are traversed consecutively.
In the \emph{Query Formulation} phase, the system supports the user 
in constructing their query.
In the \emph{Query Processing} phase, the system evaluates the query
and computes answers.
In the \emph{Answer Presentation} phase, the system allows the user to
navigate through the computed answers.
Given the significance of an explorative approach to historical research
(see Section~\ref{subsec:SoNAR_reports}),
it is very likely that the answers presented by the system will 
result in the user wanting to reformulate their original query.
In fact, query answers can be seen as one means of support
for formulating a query in the first place.
For this reason, we prefer to understand the Runtime Phase as an iterative process,
which we indicate by the looping arrow in Figure~\ref{fig:method_phases}.

We now describe these phases in more detail.

% - - - - - - - - - - - - - - - - - - - - - - - - - - - - - - - - -
\subsection{Configuration Phase}

The Configuration Phase consists of the steps that need to be taken
in order for the system to be able to interact with the user in the runtime phase.
The first step is the selection of data sources that should be incorporated.
This choice can 

\dots

% - - - - - - - - - - - - - - - - - - - - - - - - - - - - - - - - -
\subsection{Runtime Phase}

The Runtime Phase is the phase in which the system receives queries from the user,
computes answers, and presents them to the user. It is thus divided into
three subphases.

\dots




% -------------------------------------------------------------------
\section{Dynamic versus Static Setting}
\label{sec:dynamic_vs_static}

\dots

% -------------------------------------------------------------------
\section{User Interaction Aspects}
\label{sec:user_interaction}

\dots



\begin{itemize}
  \item
    data source $=$ union of all available data sources (see §\ref{subsec:data_sources})
  \item
    distinguish \emph{dynamic} and \emph{static} method -- continue discussion
    from Sections~\ref{subsec:insights_from_SoNAR} and~\ref{sec:computational_properties}
  \item
    for each of the two methods, develop/discuss:
    %
    \begin{itemize}
      \item
        preparatory steps
      \item
        QA algorithm (pseudocode?) and possibilities for implementation
      \item
        advantages (see below and Section~\ref{subsec:insights_from_SoNAR})
      \item
        disadvantages (dito)
      \item
        caveats
    \end{itemize}
    %
  \item
    dynamic vs.\ static method:
    %
    \begin{itemize}
      \item
        static: 
        
        data source graph is generated explicitly (using data integration techniques, e.g., LD)
        and updated in fixed intervals
                
        at runtime: provenance query is formulated as a \gls{SPARQL} query and posed against the graph
        
        learn from SoNAR insights
        
        reduction to SQL? use DB systems
      \item
        dynamic (preferred??):
        
        data source graph is implicit; data sources are queried \enquote{on the fly}
%        (this is the preferred scenario here; see delineation from SoNAR in §\ref{sec:HNA+SoNAR})
        
        data sources need to be defined and
        interactions with them programmed in advance
        
        at runtime: decompose query into single \gls{SPARQL} (sub)queries and pose them against several data sources,
        potentially iteratively;
        compose final answer from the partial answers
        $\leadsto$ demonstrate this for example query/ies!
      \item
        don't commit to dynamic method % (check mentions of \enquote{method}/\enquote{scenario}/\enquote{online}\enquote{off{}line} in text and rephrase if necessary)
      \item
        discuss pros and cons; connect to discussion in §\ref{sec:HNA+SoNAR}; reuse:
        
        More precisely, while our abstract model will centre around a single data source (graph)
        that represents the combination of the distributed repositories,
        our method will abstain from constructing that graph explicitly and, instead,
        answer queries \enquote{in place} over the original repositories.
        This way, our approach will not depend on hosting capacities,
        and it will always have direct access to the current content of the repositories.
        On the downside, our approach will depend on external web services provided by the repositories,
        and it will be sensitive to changes in their data models.
        Our dynamic approach also requires that inconsistencies are resolved
        a posteriori, i.e., every time a query answer is retrieved.
        As a final advantage, the dynamic approach is flexible
        in the sense that it can be applied to a static integrated data source as well,
        thus benefiting from the advantages of the static approach.
  \end{itemize}
    %
  \item
    get back to Section~\ref{sec:quality_of_answers}:
    %
    \begin{itemize}
      \item
        missing vs.\ spurious answers
      \item
        overapproximation
      \item
        \emph{reasoning}: ontologies and related technology -- also explain why
        missing knowledge cannot be added upfront, e.g., like this:
        
        Clearly, it would not be advisable to attempt adding all implicit knowledge to the data
        because that would massively inflate the data,
        as most terms have several superordinate concepts or relations and,
        furthermore, implicit knowledge is not restricted to taxonomic knowledge.
      \item
        \emph{hypothesising}
    \end{itemize}
    %
  \item
    get back to SoNAR discussion (Section~\ref{subsec:insights_from_SoNAR}) (?)
  \item
    get back to substitute information (from SoNAR and Section~\ref{sec:quality_of_answers})
    \todo[inline]{move discussion of substitute information from §\ref{subsec:further_information} to §\ref{chap:retrieval}?}
  \item
    indirect relationships
  \item
    ternary relations: quads instead of triples (in \gls{RDF} speak); add attributes to the graph model? (Markus Krötzsch's work?)
  \item
    \textbf{discuss interaction with the user!}
\end{itemize}
