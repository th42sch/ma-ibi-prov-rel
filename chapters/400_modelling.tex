% !TeX spellcheck = en_GB
% =================================================================
\chapter{Modelling Provenance Relationships}
\label{chap:modelling}

In this chapter, we develop a generic approach to modelling provenance relationships.
More precisely, we need to model three central notions: queries that a user may want to ask,
data sources that are to be consulted in order to answer a query,
and answers given to a query.
In order to obtain a generic approach, we aim at providing rigorous definitions
for these central concepts, and we seek intensional rather than extensional definitions.
In particular, those definitions should not depend on concrete example queries or data sources
such as the ones discussed in Chapters~\ref{chap:prototype_queries} and~\ref{chap:analysis};
neither should they depend on concrete objects, concepts, or relationships
(such as \enquote{Copernicus}, \enquote{exemplar}, or \enquote{student}).
Instead, we will develop an abstract model that formalises
the notions of a query, data source, and answer.
This model can then be instantiated with a multitude of concrete queries and data sources,
and it specifies how future implementations need to proceed in order to
provide answers as specified by the definitions in the model.

As a basis for our abstract model, we choose standard concepts and techniques
from graph theory.
The concept of a graph is widely used in computer science and discrete mathematics;
see standard textbooks \autocite[e.g.,][]{Diestel2012}.
Graphs and graph techniques are widely applied in various areas such as 
computer science, linguistics, physics and chemistry,
social sciences, and biology \autocite{WikiGraphTheoryApplications}.
In particular, they are used to represent large knowledge bases \autocite[e.g.,][]{Ehrlinger2016}
and underlie the Resource Description Framework (RDF) ..., which is a standard of the
World Wide Web Consortium (W3C) ... 
Furthermore, the basic definition of a graph is conceptually simple,
while graph theory provides a plethora of well-understood methods,
some of which lend themselves to computationally well-behaved algorithms for query answering. \todo{explain; citation}

\todo[inline]{Delineate from classical database theory (text snippets commented out)? Take care that this argument is a logical consequence of the requirement analysis in the previous chapter(s)!}

%An obvious choice would be to base our abstract model on database theory,
%where the notions of a database, a query, and a query answer are well-defined based on rigorous mathematical concepts;
%see, e.g., the standard introduction ... \todo{citation}.
%This framework is very general, well-established, and implemented in database management systems
%that scale well to large databases \todo{citation}.
%However, .....
%...
%We want to model objects and concepts (e.g., ...) as well as relationships
%between objects (e.g., ...).  $\leadsto$ constants, unary and binary relations
%...
%With the choice of graphs, we commit ourselves to a restricted view of a data source:
%graphs can only represent unary and binary relations via nodes and edges
%while, in general, a database may have relations of arbitrary arity.
%However, we do not consider this a significant restriction in the context of our purpose
%because we only want to represent relations that are relevant for provenance research,
%and those are predominantly unary or binary. \todo{strengthen argument, give examples, consult literature}
%
%%
%\begin{itemize}
%  \item
%    mathematically complex
%  \item
%    hard to visualise
%  \item
%    relations of arity $\geqslant 2$ seem overkill, given the relations used in OPACs, GND, Wikidata
%\end{itemize}
%%
%\todo[inline]{We first need an analysis of available data sources and of requirements; only then can we justify the choice of framework.}

In the following sections, we introduce the basic notion of a graph used in our model,
demonstrate how to use it for modelling data sources, queries, and answers,
discuss decision procedures related to query answering,
and comment on possible extensions in the context of our application.
Although the definitions are standard and self-sufficient from a mathematical point of view,
we provide explanations and illustrations for the sake of readers with little or no background in
mathematics.

% -------------------------------------------------------------------
\section{Labelled Directed Graphs}
\label{sec:labelled_digraphs}

%\begin{figure}[ht]
\begin{wrapfigure}{o}{5cm}
  \centering
  \begin{tikzpicture}[
    >=Latex,
    every node/.style={on grid,circle,draw=black,fill=lightblue,thick,inner sep=1.5mm},
    every edge/.style={draw=black,thick}
  ]
    \node                       (nw)                         {};
    \node [right=12mm of nw]    (ne)                         {};
    \node [below= 8mm of nw]    (sw)                         {};
    \node [right=12mm of sw]    (se)                         {};
    \node [draw=none,fill=none] (n)     at ($(nw)!0.5!(ne)$) {};
    \node [above= 6mm of n]     (ridge)                      {};
      
    \path[->]
      (sw)    edge (nw)
      (nw)    edge (ridge)
      (ridge) edge (ne)
      (ne)    edge (se)
      (se)    edge (sw)
      (sw)    edge (ne)
      (ne)    edge (nw)
      (nw)    edge (se)
    ;
        
  \end{tikzpicture}
  \caption{A directed graph}
  \label{fig:example_graph_abstract}
%\end{figure}
\end{wrapfigure}

The basic components of a graph are nodes and edges. Edges link nodes, and they can be directed
or undirected. Graphs are easy to visualise; nodes are typically represented as circles
or rectangles, and edges as arrows (directed) or lines (undirected).
Figure~\ref{fig:example_graph_abstract} gives an abstract example of a directed graph.

For our purposes,
nodes represent entities such as works, expressions, manifestations, items,
persons, or corporate bodies.
Edges represent relationships between those entities, which are typically directed:
e.g., \term{has\_owner} points \emph{from} an item
\emph{to} a person or corporate body,
whereas \term{is\_owner\_of} points into the opposite direction.
Therefore we use directed graphs.
Symmetric relationships, such as \term{collaborates\_with},
can be represented via two edges, one for each direction.

Furthermore, we want to assign a unique name to each node of a graph
and one or several labels to each node and each edge:
The name of a node specifies the \emph{object} that is represented by that node.
The labels of a node specify the \emph{concepts}
of which that node is an \emph{instance}.
For example, a node representing the physicist Albert Einstein
may be labelled, among others, with the concepts \term{Person}, \term{Scientist},
and \term{Physicist}.
The labels of an edge specify the \emph{relations} of which the pair of nodes
represented by that edge is an instance.
For example, if a person $p_1$ has a student $p_2$ and, in later life, 
collaborates with $p_2$, then this can be represented via an edge from $p_1$ to $p_2$
with the label $\{\term{has\_student},\term{collaborates\_with}\}$
(and/or an edge from $b$ to $p$ with the label $\{\term{is\_student\_of},\term{collaborates\_with}\}$).
These considerations lead us to a straightforward extension
of the notion of a directed graph:
a \emph{labelled directed graph}.

\todo[inline]{Refer to the literature for labelled graphs? Refer to description logic terminology regarding the terms \enquote{object}, \enquote{concept}, \enquote{relation}, \emph{instance}, etc.? Distinguish clearly between \enquote{relation} and \enquote{relationship}.}

In order to visualise a labelled directed graph,
node names are written into the respective node,
and node and edge labels are written next to the node or edge.
Multiple labels of the same node or edge are delimited with commas.
An example is given in Figure~\ref{fig:example_graph}.
The shown graph represents a part of the data described in Section~\ref{sec:manual_answering}
concerning an exemplar of Copernicus' \emph{De revolutionibus} at the
Gotha Research Library \emph{(FB Gotha)}.
It contains a node for the work (labelled with the FRBR entity \term{Work}),
a node for the exemplar (labelled with the FRBR entity \term{Item}),
and nodes for the author and two of the owners (labelled with their professions according to their GND entries).
The ten edges represent the FRBR relationships between work and item,
the creator relationship between work and author,
the owner relationship between exemplar and two owners,
student and collaboration relationships between the owners,
and the converses of these relationships.
For the sake of simplicity, the graph deviates from the FRBR model \autocite{FRBR1998}
by omitting the FRBR entities \enquote{Expression} and \enquote{Manifestation}
that should occur between the nodes labelled \enquote{Work} and \enquote{Item}.

\newcommand{\tikzexagraph}[1][]{%
  \node [#1]                                (work1)   {\fns\mystrut\term{De\_revolutionibus}};
  \node [below=21mm of work1]               (item1)   {\fns\tikztabtwo{\term{FB\_Gotha\_}}{\term{Druck\_4°\_00466}}};
  \node [above right=4mm and 50mm of work1] (person1) {\fns\tikztabtwo{\term{Nicolaus\_}}{\term{Copernicus}}};
  \node [above right=9.5mm and 50mm of item1] (person2) {\fns\tikztabtwo{\term{Johann\_}}{\term{Hommel}}};
  \node [below right=9.5mm and 50mm of item1] (person3) {\fns\tikztabtwo{\term{Valentin\_}}{\term{Thau}}};
  
  \begin{scope}[%
    every node/.style={draw=none,fill=none,inner sep=.2mm}
  ]
    \path[->]
      (work1)   edge[bend right=10] node[pos=.4,left=1mm]      {\fns\tikztabtwo[r]{\term{has\_}}{\term{exemplar}}} (item1)
      (item1)   edge[bend right=10] node[pos=.8,right=1mm]     {\fns\term{is\_exemplar\_of}}      (work1)
      (work1)   edge[bend left=4]   node[pos=.5,sloped, above] {\fns\strut\term{has\_creator}}    (person1)
      (person1) edge[bend left=4]   node[pos=.5,sloped, below] {\fns\strut\term{is\_creator\_of}} (work1)
      (item1)   edge[bend left=14]  node[pos=.5,sloped, above] {\fns\strut\term{has\_owner}}      (person2)
      (person2) edge[bend right=6]  node[pos=.5,sloped, below] {\fns\strut\term{is\_owner\_of}}   (item1)
      (item1)   edge[bend right=6]  node[pos=.5,sloped, above] {\fns\strut\term{has\_owner}}      (person3)
      (person3) edge[bend left=14]  node[pos=.5,sloped, below] {\fns\strut\term{is\_owner\_of}}   (item1)
      (person2) edge[bend right=10] node[pos=.46,left=1mm]     {\fns\tikztabtwo[r]{\term{has\_student,}}{\term{collaborates\_with}}} (person3)
      (person3) edge[bend right=10] node[pos=.54,right=1mm]    {\fns\tikztabtwo{\term{is\_student\_of,}}{\term{collaborates\_with}}} (person2)
    ;
      
    \node[above=.5mm of work1]   () {\fns\term{Work}};
    \node[below=.5mm of item1]   () {\fns\term{Item}};
    \node[right=.5mm of person1] () {\fns\tikztabtwo{\term{Person,}}{\term{Scientist}}};
    \node[right=.5mm of person2] () {\fns\tikztabtwo{\term{Person,}}{\term{Mathematician}}};
    \node[right=.5mm of person3] () {\fns\tikztabtwo{\term{Person,}}{\term{Astronomer}}};
    
  \end{scope}      
}
%
\begin{figure}[ht]
  \centering
  \begin{tikzpicture}[
    >=Latex,
    every node/.style={on grid,rectangle,rounded corners=1mm,draw=black,fill=lightblue,thick,inner sep=1.5mm},
    every edge/.style={draw=black,thick}
  ]
    \tikzexagraph
  \end{tikzpicture}
  
  \caption{A labelled directed graph that represents data
    concerning an exemplar of Copernicus' \emph{De revolutionibus} and some of its owners}
  \label{fig:example_graph}
\end{figure}

As we will see in the following, labelled directed graphs can be used in our setting
to represent (combinations of) data sources as well as queries.
They allow us to draw on standard notions from graph theory and query answering
in order to define admissible query answers and to devise methods for obtaining those.

%In a nutshell, a labelled directed graph consists of a set of nodes, a set of directed edges between
%the nodes, a function that names nodes with objects,
%and a function that labels the nodes (edges) with concepts (relations)
%of which the nodes (edges) are instances.
%In our setting, these four abstract components have the following meaning:

The above explanations can be cast into a rigorous mathematical definition,
which uses sets to represent nodes, a binary relation over the set of nodes
to represent edges, and functions over the nodes and edges to represent
names and labels. In order for the range of those functions to be well-defined,
the definition of a labelled directed graph is relative to a namespace which
contains all the names of objects, concepts and relations that are relevant.
The contents of this namespace is arbitrary and may consist,
for example, of all the names found in the relevant data sources.
The following definition introduces the notions of a namespace and a
labelled directed graph.
%
%\begin{definition}
%  Let $R$ be a set of \emph{relation names}.
%  A \emph{directed edge-labelled graph over $R$} is a triple $G = (V,E,\Lmc)$,
%  where
%  %
%  \begin{itemize}
%    \item
%    $V$ is a set, whose members are called or \emph{nodes};\footnote{%
%      In classical graph theory, nodes are called \emph{vertices}; thus the set of
%      nodes of a graph is denoted by $V$. We adopt the denotation $V$ for conformity
%      and the more modern term \enquote{node} for brevity.%
%    }      
%    \item 
%    $E \subseteq V \times V$ is a set of pairs of nodes, whose members are called \emph{edges};
%    \item
%    $\Lmc : E \to 2^R$ is a function that assigns to each edge a non-empty set of relation names,
%    called the \emph{labels} of that edge; we call \Lmc a \emph{labelling function}.
%  \end{itemize}
%\end{definition}
%
\begin{definition}
  \label{def:ld_graph}
  Let $\namespace=(\NO,\NC,\NR)$ be a \emph{namespace} consisting of a set \NO of \emph{object names}, a set \NC of \emph{concept names}, and a set \NR of \emph{relation names}.
  A \emph{labelled directed graph over $\namespace$} is a triple $G = (V,E,\Nmc,\Lmc)$
  where
  %
  \begin{itemize}
    \item
      $V$ is a set, whose members are called \emph{nodes};\footnote{%
        In classical graph theory, nodes are called \emph{vertices}; thus the set of
        nodes of a graph is denoted by $V$. We adopt the denotation $V$ for conformity
        and the more modern term \enquote{node} for brevity.%
      }      
    \item 
      $E \subseteq V \times V$ is a set of pairs of nodes, whose members are called \emph{edges};
    \item
      $\Nmc : V \to \NO$ is an injective function that assigns
      to each node a unique object (called the node's \emph{name});
    \item
      $\Lmc : V \cup E \to \NV \cup 2^{\NR}$ is a function that assigns 
      to each node a set of concept names (called the node's \emph{labels}) and
      to each edge a non-empty set of relation names (called the edge's \emph{labels});
      we call \Lmc a \emph{labelling function}.
  \end{itemize}
\end{definition}
%
Definition~1 formalises the following commitments regarding names and labels.
%%
%\begin{itemize}
%  \item
%    every node has a unique name and no two nodes have the same name (the latter being ensured by injectivity);
%  \item
%    a node can have an arbitrary number of labels, including no label (in case the node belongs to no concept);
%  \item
%    an edge can have an arbitrary number of labels, but that number must not be zero --
%    the effect of an edge having no labels can be achieved by simply omitting the edge.
%\end{itemize}
%
(1) Every node has a unique name, and no two nodes have the same name (the latter being ensured by injectivity).
(2) A node can have an arbitrary number of labels, including no label (in case the node belongs to no concept).
(3) An edge can have an arbitrary number of labels, but that number must not be zero --
the effect of an edge having no labels can be achieved by simply omitting that edge.

In order to illustrate the components of Definition~\ref{def:ld_graph},
we refer to the graph depicted in Figure~\ref{fig:example_graph}:
$V$ consists of five nodes, and $E$ of ten edges
(each single arrow constitutes an edge since the direction matters).
%Let $v_1,v_2$ denote the nodes on the left and $v_3,v_4,v_5$
%denote the nodes on the right (both from top to bottom).
There are, among others, the following node names and labels:
%
\begin{itemize}
  \item
    The node at the top left has the name \term{De\_revolutionibus}
    and the single label \term{Work}.
  \item 
    The node at the bottom right has the two labels \term{Person} and \term{Astronomer}.
  \item 
    The edge from the node named \term{Johann\_Hommel} to the node named \term{Valentin\_Thau}
    has the two labels \term{has\_student} and \term{collaborates\_with},
    and the edge pointing into the converse direction has the labels
    \term{is\_student\_of} and \term{collaborates\_with}.
\end{itemize}
%
%%
%\begin{equation*}
%  \Nmc(v_1) = \term{De\_revolutionibus}
%  \qquad
%  \Lmc(v_1) = \{\term{Work}\}
%  \qquad
%  \Lmc(v_5) = \{\term{Person},\term{Astronomer}\}
%\end{equation*}
%%
%Additionally, two of the ten edges have the following labels:
%%
%\begin{alignat*}{2}
%  e_1 & = (v_4,v_5) & \qquad \Lmc(e_1) & = \{\term{has\_student},\term{collaborates\_with}\} \\
%  e_2 & = (v_5,v_4) &        \Lmc(e_2) & = \{\term{is\_student\_of},\term{collaborates\_with}\} \\
%\end{alignat*}

\todo[inline,color=red!30]{Continue here: Give more explanation for people not familiar with graph theory; make steps that seem logical more explicit; name advantages of using graph theory as an instrument (modelling power, abstraction, available methods, well-behaved algorithms, ready-to-use implementations); span bridge from concrete example queries to GT; possibly give links to basic (video) tutorials in GT (ask Leif?)}
 
% ------------------------------------------------------------------
\section{Modelling Data Sources, Queries, and Answers}
\label{sec:modelling}

We now model a data source (e.g., library catalogue, authority file, or other database)
using the exact notion of a graph that we have introduced above.
%
\begin{definition}
%  \scalebox{0.965}[1]{A \emph{data source over the namespace $\namespace=(\NO,\NC,\NR)$} is a labelled directed graph
%  over \namespace.}
  A \emph{data source over the namespace $\namespace=(\NO,\NC,\NR)$} is a labelled directed graph
  over \namespace.
\end{definition}
%
%With this definition, we obviously commit ourselves to a restricted view of a data source:
%graphs can only represent unary and binary relations via nodes and edges
%while, in general, a database may have relations of arbitrary arity.
%However, we do not consider this a significant restriction in the context of our purpose
%because we only want to represent relations that are relevant for provenance research,
%and those are predominantly unary or binary. \todo{strengthen argument, give examples, consult literature}
%
In order to model queries with the same notion of graphs, we need to introduce
two sets of variables that serve as distinct node names:
For example, consider the following variant of the example query \exaquery{2} from Chapter~\ref{chap:prototype_queries}:
%
\begin{enumerate}
  \item[\exaquery{2$'$}]
%    Which exemplars of work $W$ were passed from one of its owners to a collaborator of theirs?
    Which exemplars of \emph{De revolutionibus} were owned by some scientist who passed them on to a student?
\end{enumerate}
%
Query~\exaquery{2$'$}
should be modelled by the graph shown in Figure~\ref{fig:graph_for_exa_query2'}.

\newcommand{\tikzexaquery}{%
  \node                                        (derev) {\fns\mystrut$\term{De\_revolutionibus}$};
  \node [ansvar,below=14mm of work1]           (x)     {\fns\mystrut$x$};
  \node [anovar,above right=6.4mm and 24mm of x] (y)     {\fns\mystrut$y$};
  \node [anovar,below right=6.4mm and 24mm of x] (z)     {\fns\mystrut$z$};
  
  \begin{scope}[%
    every node/.style={draw=none,fill=none,inner sep=.2mm}
  ]
    \path[->]
      (derev) edge node[left=1mm]           {\fns\tikztabtwo[r]{\term{has\_}}{\term{exemplar}}} (x)
      (x)     edge node[sloped, above=.6mm] {\fns\term{has\_owner}}         (y)
      (x)     edge node[sloped, below]      {\fns\strut\term{has\_owner}}   (z)
      (y)     edge node[right=1mm]          {\fns\tikztabtwo[r]{\term{has\_}}{\term{student}}} (z)      
    ;
      
    \node[above=.5mm of derev] () {\fns\term{Work}};
    \node[left=.5mm of x]      () {\fns\term{Item}};
    \node[above=.5mm of y]     () {\fns\term{Scientist}};
    
  \end{scope}
}
%
\begin{figure}[ht]
  \centering
  \begin{tikzpicture}[
    >=Latex,
    every node/.style={on grid,rectangle,rounded corners=1mm,draw=black,fill=lightblue,thick,inner sep=1.5mm},
    every edge/.style={draw=black,thick}
  ]
    \tikzexaquery
  \end{tikzpicture}
  
  \caption{A graph representing example query \exaquery{2$'$}}
  \label{fig:graph_for_exa_query2'}
\end{figure}

The nodes of this graph fall into three groups:
%
\begin{enumerate}[(1)]
  \item
    the node named \tikzinlinenode{\term{De\_revolutionibus}} represents that work;
  \item
    node \tikzinlinenode[ansvar]{\mystrut$x$} represents an exemplar of this work that satisfies the conditions stated in~\exaquery{2$'$}
    and whose name is to be found;
  \item
    nodes \tikzinlinenode[anovar]{\mystrut$y$} and \tikzinlinenode[anovar]{\mystrut$z$}
    represent the two owners (scientist and their student) whose names are not known.
\end{enumerate}
%
Consequently, node name $x$ serves as a placeholder for the answer to the query,
and names $y,z$ are placeholders for further objects that \enquote{witness} the answer.
We call $x$ the \emph{answer variable} and $y,z$ the \emph{anonymous variables}
of the query.

From now on, we fix two sets \VARANS and \VARANON
of \emph{answer variables} and \emph{anonymous variables}, respectively,
and we require that these two sets are disjoint with each other
and with any set \NO of object names.
In particular, the namespace of data sources must not contain any variables,
in contrast to the namespace of queries.
These considerations lead to the following definition of a query.

\begin{definition}
  A \emph{query over the namespace $(\NO,\NC,\NR)$} is a labelled directed graph
  over $(\NO \uplus \VARANS \uplus \VARANON, \NC, \NR)$.
\end{definition}
%
\todo[inline,caption={}]{%
  Possibly comment on:
  %
  \begin{itemize}
    \item
      the special case of Boolean queries?
    \item
      specific requirements for modelling \exaquery{1} and \exaquery{3}: 
      \begin{itemize}
        \item 
          \exaquery{1} seems to require answer variables representing sets (the owners)
          and an appropriate extension of the definition of a homomorphism;
        \item 
          the same holds for \exaquery{3}; additionally the answer should include the relationships
          between the images of the answer variables (the relationships between the owners),
          i.e., some sort of spanned subgraph
      \end{itemize}
  \end{itemize}
}
%
Answers to queries will be defined via homomorphisms that map graphs representing queries
to graphs representing data sources.
%
\begin{definition}
  Let $\namespace=(\NO,\NC,\NR)$ be a namespace, $G = (V,E,\Nmc,\Lmc)$ a query over \namespace,
  and $G' = (V',E',\Nmc',\Lmc')$ a data source over \namespace.
  A \emph{homomorphism from $G$ to $G'$} is a map $h : V \to V'$ that satisfies the following properties.
  %
  \begin{enumerate}
    \item[\hmph{1}]
      $\Nmc(v) = \Nmc'(h(v))$ for every node $v \in V$ with $\Nmc(v) \in \NO$.
    \item[\hmph{2}]
      $\Lmc(v) \subseteq \Lmc'(h(v))$ for every node $v \in V$.
    \item[\hmph{3}]
      $\Lmc(v_1, v_2) \subseteq \Lmc'(h(v_1), h(v_2))$
      for every edge $(v_1,v_2) \in E$.
  \end{enumerate}
  %
  If $h$ is a homomorphism from $G$ to $G'$, we write $h : G \to G'$.
  If there is some homomorphism from $G$ to $G'$, we write $G \lesssim G'$.
\end{definition}
%
Property~\hmph{1} ensures that a homomorphism maps each node in $G$ that is named with an object
to that node in $G'$ which is named with the same object.
Nodes named with variables in $G$ can be mapped to arbitrary nodes in $G'$.
Properties~\hmph{2} and~\hmph{3} ensure that homomorphisms preserve node and edge labels;
the $\subseteq$-relation allows the image of a node (or edge) under $h$ to have more labels
than the node (or edge) itself.

Figure~\ref{fig:example_hmph} shows a homomorphism $h$ (dashed lines)
from the query depicted in Figure~\ref{fig:graph_for_exa_query2'}
to the graph from Figure~\ref{fig:example_graph}.

\begin{figure}[ht]
  \centering
  \begin{tikzpicture}[
    >=Latex,
    every node/.style={on grid,rectangle,rounded corners=1mm,draw=black,fill=lightblue,thick,inner sep=1.5mm},
    every edge/.style={draw=black,thick}
  ]
    \tikzexaquery
    \tikzexagraph[right=64mm of derev]
    
    \begin{scope}[%
      every node/.style={draw=none,fill=none,inner sep=.2mm},
      every edge/.style={densely dashed,draw=black!70,thick}
    ]
      \path[->]
        (derev) edge[out= 20,in=160]               node[below=.6mm]         {\fns$h$} (work1)
        (x)     edge[out=270,in=200]               node[above=.4mm]         {\fns$h$} (item1)
        (y)     edge[out= 30,in=150,looseness=.19] node[above=.4mm,pos=.25] {\fns$h$} (person2)
        (z)     edge[out=340,in=200,looseness=.8]  node[above=.4mm,pos=.60] {\fns$h$} (person3)
      ;
      
    \end{scope}
  \end{tikzpicture}
  
  \caption{An example homomorphism}
  \label{fig:example_hmph}
\end{figure}

\todo[inline,caption={}]{
  \parbox{.99\linewidth}{%
    Comment on restrictions made:
    %
    \begin{itemize}
      \item
      no temporal distinctions (\enquote{is/was student}) or \enquote{was owner in year $n$}. 
      \item[$(\leadsto)$]
      no attributes for relationships (citation)
      \item[$\leadsto$]
      keep the presentation digestible for readers with little or no background in maths and CS
      (i.e., a large proportion of librarians and historians).
      \item
      possible extensions in Section~\ref{sec:possible_extensions}
    \end{itemize}
  }%
}



% ------------------------------------------------------------------
\section{Decision Problems}
\label{sec:decision_problems}

\todo[inline]{TODO: formulate problems, discuss (data!) complexity}

% ------------------------------------------------------------------
\section{Possible Extensions}
\label{sec:possible_extensions}

\todo[inline]{TODO; discuss more options? E.g., relations of arbitrary arity, data provenance, concrete values (\enquote{year of publication})}

\subsection*{Attributes on Relationships}

\begin{itemize}
  \item
    sketch idea: e.g., add year to relationship \term{has\_owner} -- example: \enquote{passed on to} requires descending year numbers \emph{and} no successor with intermediate year number
  \item
    explain difficulties: more complex formal machinery (def.\ of graphs, queries, and matches)
  \item
    argue for limited use: with or without the use of additional attributes, results may contain false positives due to incomplete data $\leadsto$ manual inspection is necessary anyway
  \item
    conclusion: attributes are not covered here
\end{itemize}

