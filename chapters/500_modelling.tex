% !TeX spellcheck = en_GB
% =================================================================
\chapter{A General Model of Provenance Relationships}
\label{chap:modelling}

In this chapter, we develop a generic approach to modelling provenance relationships.
More precisely, we need to model three central notions: queries that a user may want to ask,
data sources that are to be consulted in order to answer a query,
and answers given to a query.
In order to obtain a generic approach, we aim at providing rigorous definitions
for these central concepts, and we seek intensional rather than extensional definitions.
In particular, those definitions should not depend on specific example queries or data sources
such as the ones discussed in Chapters~\ref{chap:prototype_queries} and~\ref{chap:analysis};
neither should they depend on concrete objects, concepts, or relations
(such as \term{Copernicus}, \term{exemplar}, or \term{student}).
Instead, we will develop an abstract model that formalises
the notions of a query, data source, and answer.
This model can then be instantiated with a multitude of specific queries and data sources,
and it constitutes the basis of the method for finding answers
that we will develop in the next chapter.

As a basis for our abstract model, we choose standard concepts and techniques
from graph theory.
The concept of a graph is widely used in computer science and discrete mathematics;
see standard textbooks \autocite[e.g.,][]{Diestel2012}.
Graphs and graph techniques are widely applied in various areas such as 
computer science, linguistics, physics and chemistry,
social sciences, and biology \autocite{WikiGraphTheoryApplications}.
They are also the foundation of social networks \autocite{Galety2022}
and therefore highly relevant for historical research, e.g.,
in the \gls{SoNAR} project, as we have seen in Section~\ref{sec:HNA+SoNAR}.
Graphs
are used to represent large knowledge bases \autocite[e.g.,][]{Ehrlinger2016}
and are a fundamental ingredient of \gls{RDF}.
Furthermore, the basic definition of a graph is conceptually simple,
and graph theory provides a plethora of well-understood concepts
and algorithms. By utilising graph theory, our application scenario
can benefit from these concepts, algorithms \autocite{Diestel2012,Even2012},
and implementations \autocite{PythonGraphLibraries,JGraphT}.

The main idea of our abstract model is the following.
Both queries and data sources are represented as graphs
(typically a rather small graph for the query and a very large graph for the data source).
Answers to the query are those parts of the graph that have the same structure
as the query. In more formal words, answers are found using \emph{pattern matching} techniques,
which are well-known from querying \gls{RDF} graphs via \gls{SPARQL} \autocite{DellaValle2011}
and from database and graph theory \autocite{Abiteboul1995,Diestel2012}.

%\todo[defer,inline]{Delineate from classical database theory (text snippets commented out)? This argument %should be a logical consequence of the requirement analysis in the previous chapter(s)!}
%
%An obvious choice would be to base our abstract model on database theory,
%where the notions of a database, a query, and a query answer are well-defined based on rigorous mathematical concepts;
%see, e.g., the standard introduction ... \todo{citation}.
%This framework is very general, well-established, and implemented in database management systems
%that scale well to large databases \todo{citation}.
%However, .....
%...
%We want to model objects and concepts (e.g., ...) as well as relationships
%between objects (e.g., ...).  $\leadsto$ constants, unary and binary relations
%...
%With the choice of graphs, we commit ourselves to a restricted view of a data source:
%graphs can only represent unary and binary relations via nodes and edges
%while, in general, a database may have relations of arbitrary arity.
%However, we do not consider this a significant restriction in the context of our purpose
%because we only want to represent relations that are relevant for provenance research,
%and those are predominantly unary or binary. \todo{strengthen argument, give examples, consult literature}
%
%%
%\begin{itemize}
%  \item
%    mathematically complex
%  \item
%    hard to visualise
%  \item
%    relations of arity $\geqslant 2$ seem overkill, given the relations used in OPACs, GND, Wikidata
%\end{itemize}
%%
%\todo[inline]{We first need an analysis of available data sources and of requirements; only then can we justify the choice of framework.}

In the following sections, we introduce our model
by defining 
the basic terminology (Section~\ref{sec:terminology}),
the specific notion of a graph (Section~\ref{sec:labelled_digraphs}),
and the abstract notions of data sources, queries, and answers
(Section~\ref{sec:modelling}).
We also discuss decision procedures related to query answering (Section~\ref{sec:decision_problems}).
Up to that point, the model remains conceptually simple
and is based on elementary and self-contained mathematical definitions.
We provide additional explanations and illustrations
for the sake of readers with little or no background in mathematics.
%
Finally, in Section~\ref{sec:modelling_discussion}, we will discuss
the scope and limitations of our basic model based on the exemplary query patterns
\exaquery{1}--\exaquery{9}, and sketch possible extensions.

% -------------------------------------------------------------------
\section{Basic Notions}
\label{sec:terminology}

We start by introducing the basic terminology that we are going to use
in the following, which consists of the usual terms from conceptual
modelling \autocite{Brodie1984}.

\paragraph{Objects}
An \emph{object} is a specific entity,
for example, the work \enquote{Graph Theory} by Reinhard Diestel,
its expression in English,
its manifestation as the 4th edition by Springer,
a specific copy, or the person Reinhard Diestel.

\paragraph{Concepts}
A \emph{concept} is a class of objects,
such as the entities \term{Work}, \term{Expression}, etc.\
from the \gls{FRBR} model (see Section~\ref{sec:bib_standards}).

We determine that names for concepts and objects start with an uppercase letter.

\paragraph{Relations}
An \emph{$n$-ary relation} is a set of $n$-tuples of objects;
for example, the binary relation \term{creator}
consists of pairs of objects including $(\term{Graph\_Theory},\term{Reinhard\_Diestel})$
if we assume for the sake of simplicity that \term{Graph\_Theory}
is the unambiguous name of said work. 

Constants and concepts are in fact nullary and unary relations,
but it is more intuitive to use those separate terms,
As we have already argued in Section~\ref{sec:implications_on_modelling},
we will hardly need to deal with relations of arity beyond 2;
therefore we will mostly discuss binary relations
and refer to them as \emph{relations} when no confusion may arise.

To distinguish relations from concepts and objects,
we determine that their names start with a lowercase letter.

\paragraph{Converses of relations}
The \emph{converse} of a relation $R$ is the relation obtained from $R$
by swapping the components in each pair,
e.g., the converse \term{creator\_of} of \term{creator}
contains the pair $(\term{Reinhard\_Diestel},\term{Graph\_Theory})$.

\paragraph{Instances}
An \emph{instance} of a concept or relation
is an object or a pair of objects from that set;
e.g., \term{Reinhard\_Diestel} is an instance of \term{Person},
and $(\term{Graph\_Theory},\term{Reinhard\_Diestel})$ is an instance of
\term{creator}.

\paragraph{Relationships}
Instances of relations are called \emph{relationships}.

\paragraph{Literals}
A \emph{literal} is a fixed value, such as a year, date, or identifier.

%%
%\begin{itemize}
%  \item
%    An \emph{object} is a specific entity,
%    for example, the work \enquote{Graph Theory} by Reinhard Diestel,
%    its manifestation as the 4th edition in English,
%    a specific copy, or the person Reinhard Diestel.
%  \item
%    A \emph{concept} is a class of objects,
%    such as the entities \term{Work}, \term{Expression}, etc.\
%    from the \gls{FRBR} model (see Section~\ref{sec:bib_standards}).
%    
%    We determine that names for concepts and objects start with an uppercase letter.
%  \item
%    An \emph{$n$-ary relation} is a set of $n$-tuples of objects;
%    for example, the binary relation \term{creator}
%    consists of pair of objects including $(\term{Reinhard\_Diestel},\term{Graph\_Theory})$
%    if we assume for the sake of simplicity that \term{Graph\_Theory}
%    is the unambiguous name of said work. 
%    
%    Constants and concepts are in fact nullary and unary relations,
%    but it is more intuitive to use those separate terms,
%    As we have already argued in Section~\ref{sec:implications_on_modelling},
%    we will hardly need to deal with relations of arity beyond 2;
%    therefore we will mostly discuss binary relations
%    and refer to them as \emph{relations} when no confusion may arise.
%    
%    To distinguish relations from concepts and objects,
%    we determine that their names start with a lowercase letter.
%  \item
%    An \emph{instance} of a concept or relation
%    is an object or a pair of objects from that set;
%    e.g., \term{Reinhard\_Diestel} is an instance of \term{Person},
%    and $(\term{Reinhard\_Diestel},\term{Graph\_Theory})$ is an instance of
%    \term{creatorOf}.
%  \item
%    Instances of relations are also called \emph{relationships}.
%  \item 
%    A \emph{literal} is a fixed value, such as a year, date, or identifier.
%\end{itemize}
%

% -------------------------------------------------------------------
\section{Labelled Directed Graphs}
\label{sec:labelled_digraphs}

%\begin{figure}[ht]
\begin{wrapfigure}[6]{o}{5.2cm}
  \centering
  \vspace*{-.7\baselineskip}
  \begin{tikzpicture}[
    >=Latex,
    every node/.style={on grid,circle,draw=black,fill=lightblue,thick,inner sep=1.5mm},
    every edge/.style={draw=black,thick}
  ]
    \node                       (nw)                         {};
    \node [right=12mm of nw]    (ne)                         {};
    \node [below= 8mm of nw]    (sw)                         {};
    \node [right=12mm of sw]    (se)                         {};
    \node [draw=none,fill=none] (n)     at ($(nw)!0.5!(ne)$) {};
    \node [above= 6mm of n]     (ridge)                      {};
      
    \path[->]
      (sw)    edge (nw)
      (nw)    edge (ridge)
      (ridge) edge (ne)
      (ne)    edge (se)
      (se)    edge (sw)
      (sw)    edge (ne)
      (ne)    edge (nw)
      (nw)    edge (se)
    ;
        
  \end{tikzpicture}
  \caption{A directed graph}
  \label{fig:example_graph_abstract}
%\end{figure}
\end{wrapfigure}

Before we develop the technical definitions,
we provide the underlying intuitions.
Graphs consists of nodes and edges. Edges link nodes and can be directed
or undirected. Graphs are easy to visualise: nodes are drawn as circles
or rectangles, and edges as arrows (directed) or lines (undirected).
Figure~\ref{fig:example_graph_abstract} shows an abstract example of a directed graph.

For our purposes,
nodes represent objects or literals. 
%Objects include, e.g., works, expressions, manifestations, items,
%persons, or corporate bodies; literals include, e.g., publication years, birth years, or identifiers.
Directed edges represent relations, 
for example, 
the relation \term{creator} is represented by a set of directed edges
pointing \emph{from} nodes representing persons or corporate bodies
edges representing the relation
\emph{to} nodes representing the corresponding works;
the converse relation \term{creatorOf} is represented by the same set of edges 
with their direction reversed.
Since relations such as \term{creator} are not symmetric,
direction matters and we use \emph{directed} graphs. 
Symmetric relations such as \term{relatedEndeavour}
can be represented via pairs of edges pointing in both directions.

Furthermore, we want to assign a unique name to each node of a graph
and one or several labels to each node and each edge:
The name of a node specifies the object that is represented by that node.
The labels of a node specify the concepts
of which that node is an instance.
For example, a node representing the physicist Albert Einstein
may be labelled, among others, with the concepts \term{Person}, \term{Scientist},
and \term{Physicist}.
The labels of an edge specify the relations of which the pair of nodes
represented by that edge is an instance.
For example, if a person $P_1$ has a student $P_2$ and also 
collaborates with $P_2$, then this can be represented via an edge from $P_1$ to $P_2$
labelled with both \term{student} and \term{collaboratesWith}
(and/or an edge from $P_2$ to $P_1$ labelled with both \term{studentOf} and \term{collaborator}).
These considerations lead us to a straightforward extension
of the notion of a directed graph:
a \emph{labelled directed graph}.

In order to visualise a labelled directed graph,
node names are written into the respective node,
and node and edge labels are written next to the node or edge.
Multiple labels of the same node or edge are delimited with commas.
An example is given in Figure~\ref{fig:example_graph}.
The shown graph represents a part of the data described in Section~\ref{sec:manual_answering}
concerning an exemplar of Copernicus' \emph{De revolutionibus} at the
Gotha Research Library \emph{(FB Gotha)}.
It contains a node for the work (labelled with the \gls{FRBR} entity \term{Work}),
a node for the exemplar (labelled with the \gls{FRBR} entity \term{Item}),
and nodes for the author and two of the owners (labelled with their professions according to their \gls{GND} entries).
For the sake of this example, the owners are additionally labelled with the profession
\term{Scientist}, which is implicit in the actual data.
The ten edges represent relationships that are instances
of \gls{FRBR} relations and others between \term{Work}, \term{Item}, and \term{Person}.
For the sake of simplicity, the graph deviates from the \gls{FRBR} model \autocite{FRBR1998}
by omitting the \gls{FRBR} entities \term{Expression} and \term{Manifestation}
that should occur between the nodes labelled \term{Work} and \term{Item}.

\newlength{\derevnode}\settowidth{\derevnode}{\fns\term{De\_revolutionibus}}%
\newlength{\copernode}\settowidth{\copernode}{\fns\term{Copernicus}}%
\newcommand{\tikzexagraphMinusCopernicus}[1][]{%
  \node [text width=\derevnode,#1]                                (work1)   {\fns\mystrut\term{De\_revolutionibus}};
  \node [text width=\derevnode,below=24mm of work1]               (item1)   {\fns\tikztabtwo{\term{FB\_Gotha\_}}{\term{Druck\_4°\_00466}}};
  \node [text width=\copernode,above right=11mm and 45mm of item1] (person2) {\fns\tikztabtwo{\term{Johann\_}}{\term{Hommel}}};
  \node [text width=\copernode,below right=11mm and 45mm of item1] (person3) {\fns\tikztabtwo{\term{Valentin\_}}{\term{Thau}}};
  
  \begin{scope}[%
    every node/.style={draw=none,fill=none,inner sep=.2mm,align=left}
  ]
    \path[->]
%      (work1)   edge[bend right=10] node[pos=.4,left=1mm]      {\fns\tikztabtwo[r]{\term{has\_}}{\term{exemplar}}} (item1)
%      (item1)   edge[bend right=10] node[pos=.8,right=1mm]     {\fns\term{is\_exemplar\_of}}      (work1)
%      (work1)   edge[bend left=4]   node[pos=.5,sloped, above] {\fns\strut\term{has\_creator}}    (person1)
%      (person1) edge[bend left=4]   node[pos=.5,sloped, below] {\fns\strut\term{is\_creator\_of}} (work1)
%      (item1)   edge[bend left=14]  node[pos=.5,sloped, above] {\fns\strut\term{has\_owner}}      (person2)
%      (person2) edge[bend right=6]  node[pos=.5,sloped, below] {\fns\strut\term{is\_owner\_of}}   (item1)
%      (item1)   edge[bend right=6]  node[pos=.5,sloped, above] {\fns\strut\term{has\_owner}}      (person3)
%      (person3) edge[bend left=14]  node[pos=.5,sloped, below] {\fns\strut\term{is\_owner\_of}}   (item1)
%      (person2) edge[bend right=10] node[pos=.46,left=1mm]     {\fns\tikztabtwo[r]{\term{has\_student,}}{\term{collaborates\_with}}} (person3)
%      (person3) edge[bend right=10] node[pos=.54,right=1mm]    {\fns\tikztabtwo{\term{is\_student\_of,}}{\term{collaborates\_with}}} (person2)
      ($(work1.270)+(-0.08,0)$)   edge node[pos=.3,left=.2mm]          {\fns\term{exemplar}}      ($(item1.90)+(-0.08,0)$)
      ($(item1.90)+(0.08,0)$)     edge node[pos=.7,right=.2mm]         {\fns\term{exemplarOf}}    ($(work1.270)+(0.08,0)$)
      
      ($(item1.0)+(0,0.4)$)      edge node[pos=.5,sloped, above=.4mm] {\fns\term{owner}}   ($(person2.180)+(0,0.08)$)
      ($(person2.180)+(0,-0.08)$) edge node[pos=.5,sloped, below=.4mm] {\fns\term{ownerOf}} ($(item1.0)+(0,0.24)$)

      ($(item1.0)+(0,-0.24)$)     edge node[pos=.5,sloped, above=.4mm] {\fns\term{owner}}   ($(person3.180)+(0,0.08)$)
      ($(person3.180)+(0,-0.08)$) edge node[pos=.5,sloped, below=.4mm] {\fns\term{ownerOf}} ($(item1.0)+(0,-0.4)$)
      
      ($(person2.270)+(-0.08,0)$) edge node[pos=.46,left=.2mm]         {\fns\tikztabtwo[r]{\term{student,}}{\term{collaborator}}} ($(person3.90)+(-0.08,0)$)
      ($(person3.90)+(0.08,0)$)   edge node[pos=.54,right=.2mm]        {\fns\tikztabtwo{\term{studentOf,}}{\term{collaborator}}}  ($(person2.270)+(0.08,0)$)
    ;
      
    \node[above=.5mm of work1]   () {\fns\term{Work}};
    \node[below=.5mm of item1]   () {\fns\term{Item}};
    \node[right=.5mm of person2] () {\fns\tikztabthree{\term{Person,}}{\term{Scientist,}}{\term{Mathematician}}};
    \node[right=.5mm of person3] () {\fns\tikztabthree{\term{Person,}}{\term{Scientist,}}{\term{Astronomer}}};
    
  \end{scope}      
}%
\newcommand{\tikzexagraph}[1][]{%
  \tikzexagraphMinusCopernicus[#1]

  \node [text width=\copernode,above right=4mm and 45mm of work1] (person1) {\fns\tikztabtwo{\term{Nicolaus\_}}{\term{Copernicus}}};

  \begin{scope}[%
    every node/.style={draw=none,fill=none,inner sep=.2mm,align=left}
  ]
    \path[->]
      ($(work1.0)+(0,0.08)$)      edge node[pos=.5,sloped, above=.4mm] {\fns\term{creator}}   ($(person1.180)+(0,0.08)$)
      ($(person1.180)+(0,-0.08)$) edge node[pos=.5,sloped, below=.4mm] {\fns\term{creatorOf}} ($(work1.0)+(0,-0.08)$)
    ;
  
    \node[right=.5mm of person1] () {\fns\tikztabtwo{\term{Person,}}{\term{Astronomer}}};
  \end{scope}      

}%
%
\begin{figure}[ht]
  \centering
  \begin{tikzpicture}[
    >=Latex,
    every node/.style={on grid,rectangle,rounded corners=1mm,draw=black,fill=lightblue,thick,inner sep=1.5mm,align=center},
    every edge/.style={draw=black,thick}
  ]
    \tikzexagraph
  \end{tikzpicture}
  
  \caption{A labelled directed graph that represents data
    concerning an exemplar of Copernicus' \emph{De revolutionibus} and some of its owners}
  \label{fig:example_graph}
\end{figure}

As we will see in the following, labelled directed graphs can be used in our setting
to represent (combinations of) data sources as well as queries.
They allow us to draw on standard notions from graph theory and query answering
in order to define admissible query answers and to devise methods for obtaining those.

%In a nutshell, a labelled directed graph consists of a set of nodes, a set of directed edges between
%the nodes, a function that names nodes with objects,
%and a function that labels the nodes (edges) with concepts (relations)
%of which the nodes (edges) are instances.
%In our setting, these four abstract components have the following meaning:

The above explanations can be cast into a rigorous mathematical definition,
which uses sets to represent nodes, a binary relation over the set of nodes
to represent edges, and functions over the nodes and edges to represent
names and labels. In order for the ranges%
\footnote{The \emph{range} or \emph{codomain} of a function is the set of values that can occur as images of that function.}
of those functions to be well-defined,
we give a definition of a labelled directed graph that is relative to a \emph{namespace},
which contains all the names of objects, concepts and relations that are relevant.
This namespace is a parameter that can be freely chosen; it may consist,
for example, of all the names found in the relevant data sources.
The following definition introduces the notions of a namespace and a
labelled directed graph.
%
%\begin{definition}
%  Let $R$ be a set of \emph{relation names}.
%  A \emph{directed edge-labelled graph over $R$} is a triple $G = (V,E,\Lmc)$,
%  where
%  %
%  \begin{itemize}
%    \item
%    $V$ is a set, whose members are called or \emph{nodes};\footnote{%
%      In classical graph theory, nodes are called \emph{vertices}; thus the set of
%      nodes of a graph is denoted by $V$. We adopt the denotation $V$ for conformity
%      and the more modern term \enquote{node} for brevity.%
%    }      
%    \item 
%    $E \subseteq V \times V$ is a set of pairs of nodes, whose members are called \emph{edges};
%    \item
%    $\Lmc : E \to 2^R$ is a function that assigns to each edge a non-empty set of relation names,
%    called the \emph{labels} of that edge; we call \Lmc a \emph{labelling function}.
%  \end{itemize}
%\end{definition}
%
\begin{definition}
  \label{def:ld_graph}
  Let $\namespace=(\NO,\NC,\NR)$ be a \emph{namespace} consisting of a set \NO of \emph{object names}, a set \NC of \emph{concept names}, and a set \NR of \emph{relation names}.
  A \emph{labelled directed graph over $\namespace$} is a triple $G = (V,E,\Nmc,\Lmc)$
  where
  %
  \begin{itemize}
    \item
      $V$ is a set, whose members are called \emph{nodes};\footnote{%
        We use the standard denotation $V$ (from \enquote{vertex}) for the set of nodes.%
      }      
    \item 
      $E \subseteq V \times V$ is a set of pairs of nodes, whose members are called \emph{edges};
    \item
      $\Nmc : V \to \NO$ is an injective function that assigns
      to each node a unique object (called the node's \emph{name});
    \item
      $\Lmc : V \cup E \to \NV \cup 2^{\NR}$ is a function that assigns 
      to each node a set of concept names (called the node's \emph{labels}) and
      to each edge a non-empty set of relation names (called the edge's \emph{labels});
      we call \Lmc a \emph{labelling function}.
  \end{itemize}
\end{definition}
%
Definition~1 captures the following commitments regarding names and labels.
%%
%\begin{itemize}
%  \item
%    every node has a unique name and no two nodes have the same name (the latter being ensured by injectivity);
%  \item
%    a node can have an arbitrary number of labels, including no label (in case the node belongs to no concept);
%  \item
%    an edge can have an arbitrary number of labels, but that number must not be zero --
%    the effect of an edge having no labels can be achieved by simply omitting the edge.
%\end{itemize}
%
(1) Every node has a unique name, and no two nodes have the same name (the latter is ensured by injectivity).
(2) A node can have an arbitrary number of labels, including no label (in case the node belongs to no concept).
(3) An edge can have an arbitrary number of labels, but that number must not be zero --
the effect of an edge having no labels can be achieved by simply omitting that edge.

In order to illustrate the components of Definition~\ref{def:ld_graph},
we refer to the graph depicted in Figure~\ref{fig:example_graph}:
$V$ consists of five nodes, and $E$ of ten edges
(each single arrow constitutes an edge since the direction matters).
%Let $v_1,v_2$ denote the nodes on the left and $v_3,v_4,v_5$
%denote the nodes on the right (both from top to bottom).
There are, among others, the following node names and labels:
%
\begin{itemize}
  \item
    The node at the top left has the name \term{De\_revolutionibus}
    and the single label \term{Work}.
  \item 
    The node at the bottom right has the two labels \term{Person} and \term{Astronomer}.
  \item 
    The edge from the node named \term{Johann\_Hommel} to the node named \term{Valentin\_Thau}
    has the two labels \term{student} and \term{collaborator},
    and the edge pointing in the reverse direction has the labels
    \term{studentOf} and \term{collaborator}.
\end{itemize}
%
%%
%\begin{equation*}
%  \Nmc(v_1) = \term{De\_revolutionibus}
%  \qquad
%  \Lmc(v_1) = \{\term{Work}\}
%  \qquad
%  \Lmc(v_5) = \{\term{Person},\term{Astronomer}\}
%\end{equation*}
%%
%Additionally, two of the ten edges have the following labels:
%%
%\begin{alignat*}{2}
%  e_1 & = (v_4,v_5) & \qquad \Lmc(e_1) & = \{\term{has\_student},\term{collaborates\_with}\} \\
%  e_2 & = (v_5,v_4) &        \Lmc(e_2) & = \{\term{is\_student\_of},\term{collaborates\_with}\} \\
%\end{alignat*}

% ------------------------------------------------------------------
\section{Modelling Data Sources, Queries, and Answers}
\label{sec:modelling}

We can now use our notion of a labelled directed graph to model
data sources and queries, and to obtain a rigorous definition of a query answer.

% - - - - - - - - - - - - - - - - - - - - - - - - - - - - - - - - -
\subsection{Data Sources}
\label{subsec:data_sources}

Data sources correspond exactly to our notion of a graph.
%
\begin{definition}
  \label{def:data_source}
%  \scalebox{0.965}[1]{A \emph{data source over the namespace $\namespace=(\NO,\NC,\NR)$} is a labelled directed graph
%  over \namespace.}
  A \emph{data source over the namespace $\namespace=(\NO,\NC,\NR)$} is a labelled directed graph
  over \namespace.
\end{definition}
%
In our model, we assume that there is always a \emph{single} data source against which a query is posed and evaluated.
When the model is applied to real-word queries and data sources,
the abstract notion of a data source is instantiated by the union of
all available actual data sources (such as catalogues, authority files, knowledge bases),
including mappings between them if applicable.
We will discuss this point in more detail in Chapter~\ref{chap:retrieval}.

% - - - - - - - - - - - - - - - - - - - - - - - - - - - - - - - - -
\subsection{Queries}

In order to model queries based on graphs, we need to distinguish
two special groups of nodes that act as placeholders (1) for the object(s) after which the query asks
and (2) for further objects that are mentioned in the query without being named explicitly.
For example, consider Query \exaquery{2$'$} from Section~\ref{sec:manual_answering}:
%
\Qtwoprime
%
To model this query, we do not only need a node representing the work \emph{De revolutionibus},
but also a node representing an exemplar that satisfies the conditions stated in the query and whose name is asked for (Group~1),
and two nodes representing the owner and their student (Group~2).
Since these three individuals are not known when the query is formulated,
we need to use \emph{variables} for naming them.
Query~\exaquery{2$'$}
can now be modelled by the graph shown in Figure~\ref{fig:graph_for_exa_query2'}.

\newcommand{\tikzexaquery}[1][1]{%
  \begin{scope}[%
    opacity=#1,fill opacity=#1
  ]
    \node                                        (derev) {\fns\mystrut$\term{De\_revolutionibus}$};
    \node [ansvar,below=14mm of derev]           (x)     {\fns\mystrut$x$};
    \node [anovar,above right=6.4mm and 20mm of x] (y)     {\fns\mystrut$y$};
    \node [anovar,below right=6.4mm and 20mm of x] (z)     {\fns\mystrut$z$};
  \end{scope}
  
  \begin{scope}[%
    every node/.style={draw=none,fill=none,inner sep=.2mm},
    opacity=#1,fill opacity=#1
  ]
    \path[->]
%      (derev) edge node[left=1mm]           {\fns\tikztabtwo[r]{\term{has\_}}{\term{exemplar}}} (x)
%      (x)     edge node[sloped, above=.6mm] {\fns\term{has\_owner}}         (y)
%      (x)     edge node[sloped, below]      {\fns\strut\term{has\_owner}}   (z)
%      (y)     edge node[right=1mm]          {\fns\tikztabtwo[r]{\term{has\_}}{\term{student}}} (z)      
      (derev) edge node[left=1mm]           {\fns\term{exemplar}} (x)
      (x)     edge node[sloped, above=.6mm] {\fns\term{owner}}         (y)
      (x)     edge node[sloped, below]      {\fns\strut\term{owner}}   (z)
      (y)     edge[shorten >= -1.5pt, shorten <= -1.5pt] node[right=1mm] {\fns\term{student}} (z)      
    ;
      
    \node[above=.5mm of derev] () {\fns\term{Work}};
    \node[left=.5mm of x]      () {\fns\term{Item}};
    \node[above right=2mm and -5mm of y]     () {\fns\term{Scientist}};
    
  \end{scope}
}
%
\begin{figure}[ht]
  \centering
  \begin{tikzpicture}[
    >=Latex,
    every node/.style={on grid,rectangle,rounded corners=1mm,draw=black,fill=lightblue,thick,inner sep=1.5mm},
    every edge/.style={draw=black,thick}
  ]
    \tikzexaquery
  \end{tikzpicture}
  
  \caption{Graph representation of example query \exaquery{2$'$}}
  \label{fig:graph_for_exa_query2'}
\end{figure}

The nodes of this graph fall into three groups:
%
\begin{enumerate}[(1)]
  \item
    The node named \tikzinlinenode{\term{De\_revolutionibus}} represents that work;
  \item
    Node \tikzinlinenode[ansvar]{\mystrut$x$} falls into Group~1 as explained above;
  \item
    Nodes \tikzinlinenode[anovar]{\mystrut$y$} and \tikzinlinenode[anovar]{\mystrut$z$}
    fall into Group~2 as explained above.
\end{enumerate}
%
Node names $x,y,z$ are the variables mentioned above,
and we call $x$ the \emph{answer variable} and $y,z$ the \emph{anonymous variables}
of this query.
From now on, we fix two sets \VARANS and \VARANON
of \emph{answer variables} and \emph{anonymous variables}, respectively,
and we assume that they both contain a countably infinite number of elements
(which ensures that there is an unlimited supply of variables).
We furthermore require that these two sets are disjoint with each other
and with any set \NO of object names.
Thus, according to Definition~\ref{def:data_source},
graphs representing data sources cannot use any variables as node names.

In order to allow queries to use variables,
the following definition of a query is now immediate. 
%
\begin{definition}
  A \emph{query over the namespace $(\NO,\NC,\NR)$} is a labelled directed graph
  over $(\NO \uplus \VARANS \uplus \VARANON, \NC, \NR)$.
\end{definition}
%
The operator \enquote{$\uplus$} used in this definition
stands for \emph{disjoint union}, i.e., \enquote{$A \uplus B$}
stands for the union of the \emph{disjoint} sets $A,B$.
The wording of the definition also ensures
that we do not have to mention variables explicitly when specifying the
namespace of a query.

% - - - - - - - - - - - - - - - - - - - - - - - - - - - - - - - - -
\subsection{Query Answers}

Based on the representation of both data sources and queries as graphs,
we can now define the notion of an answer to a query 
with respect to a data source. For this purpose, it is important to realise
that, typically, a query is a small graph and a data source is a large graph,
and that finding answers means finding small parts of the large graph
that have the same structure as the small graph.
For the example query given in Figure~\ref{fig:graph_for_exa_query2'},
this means that every subgraph of the data source consisting of four nodes
with the same edges and labels should be an answer.
Regarding the previous example, the subgraphs given in
Figure~\ref{fig:example_answers}~(a,\,b) constitute answers,
while the subgraphs in Figure~\ref{fig:example_answers}~(c,\,d) do not:
Subgraph~(a) is identical to the query graph with the only exception that it
contains proper objects instead of variables in the node names;
Subgraph~(b) is the same graph as~(a) but extended with an additional edge and additional node labels;
Subgraph~(c) lacks the edge labelled \term{student} between the two owners;
Subgraph~(d) resembles the structure of the query but does not contain
the required node named \term{De\_revolutionibus}.

\newlength{\student}
\settowidth{\student}{{\fns\term{student}}}
\newlength{\Scientist}
\settowidth{\Scientist}{{\fns\term{Scientist}}}
\begin{figure}[ht]
  \centering
  \begin{tabular}{@{}c@{\hspace{10mm}}c@{}}
    \begin{tikzpicture}[
      >=Latex,
      every node/.style={on grid,rectangle,rounded corners=1mm,draw=black,fill=lightblue,thick,inner sep=1.5mm,align=center},
      every edge/.style={draw=black,thick}
    ]
      \node [text width=\derevnode]                     (derev) {\fns\mystrut$\term{De\_revolutionibus}$};
      \node [text width=\derevnode,below=16mm of work1] (x)     {\fns\tikztabtwo{\term{FB\_Gotha\_}}{\term{Druck\_4°\_00466}}};
      \node [above right=8.5mm and 37mm of x] (y)     {\fns\tikztabtwo{\term{Johann\_}}{\term{Hommel}}};
      \node [below right=8.5mm and 37mm of x] (z)     {\fns\tikztabtwo{\term{Valentin\_}}{\term{Thau}}};
  
      \begin{scope}[%
        every node/.style={draw=none,fill=none,inner sep=.2mm,align=left}
      ]
        \path[->]
%          (derev) edge node[left=1mm]           {\fns\tikztabtwo[r]{\term{has\_}}{\term{exemplar}}} (x)
%          (x)     edge node[sloped, above=.6mm] {\fns\term{has\_owner}}         (y)
%          (x)     edge node[sloped, below]      {\fns\strut\term{has\_owner}}   (z)
%          (y)     edge node[right=1mm]          {\fns\tikztabtwo[r]{\term{has\_}}{\term{student}}} (z)      
          (derev) edge node[left=1mm]           {\fns\term{exemplar}} (x)
          (x)     edge node[sloped, above=.6mm] {\fns\term{owner}}         (y)
          (x)     edge node[sloped, below]      {\fns\strut\term{owner}}   (z)
          (y)     edge node[right=1mm]          {\fns\term{student}} (z)      
        ;
          
        \node[above=.5mm of derev] () {\fns\term{Work}};
        \node[below=.5mm of x]     () {\fns\term{Item}};
        \node[above=.5mm of y]     () {\fns\term{Scientist}};
      \end{scope}
    \end{tikzpicture}
    &
    \begin{tikzpicture}[
      >=Latex,
      every node/.style={on grid,rectangle,rounded corners=1mm,draw=black,fill=lightblue,thick,inner sep=1.5mm,align=center},
      every edge/.style={draw=black,thick}
    ]
      \node [text width=\derevnode]                     (derev) {\fns\mystrut$\term{De\_revolutionibus}$};
      \node [text width=\derevnode,below=16mm of work1] (x)     {\fns\tikztabtwo{\term{FB\_Gotha\_}}{\term{Druck\_4°\_00466}}};
      \node [above right=8.5mm and 37mm of x] (y)     {\fns\tikztabtwo{\term{Johann\_}}{\term{Hommel}}};
      \node [below right=8.5mm and 37mm of x] (z)     {\fns\tikztabtwo{\term{Valentin\_}}{\term{Thau}}};
  
      \begin{scope}[%
        every node/.style={draw=none,fill=none,inner sep=.2mm,align=left}
      ]
        \path[->]
%          (derev) edge[bend right=10] node[left=1mm]         {\fns\tikztabtwo[r]{\term{has\_}}{\term{exemplar}}} (x)
%          (x)     edge[bend right=10] node[pos=.6,right=1mm] {\fns\term{is\_exemplar\_of}}                       (derev)
%          (x)     edge node[sloped, above=.6mm] {\fns\term{has\_owner}}         (y)
%          (x)     edge node[sloped, below]      {\fns\strut\term{has\_owner}}   (z)
%          (y)     edge node[right=1mm]          {\fns\tikztabtwo[r]{\term{has\_}}{\term{student}}} (z)      
          (derev) edge[bend right=10] node[left=1mm]         {\fns\term{exemplar}} (x)
          (x)     edge[bend right=10] node[pos=.6,right=1mm] {\fns\term{exemplarOf}}                       (derev)
          (x)     edge node[sloped, above=.6mm] {\fns\term{owner}}         (y)
          (x)     edge node[sloped, below]      {\fns\strut\term{owner}}   (z)
          (y)     edge node[right=1mm]          {\fns\term{student}} (z)      
        ;
          
        \node[above=.5mm of derev] () {\fns\term{Work}};
        \node[below=.5mm of x]     () {\fns\term{Item}};
        \node[right=.5mm of y]     () {\fns\tikztabtwo{\term{Person}}{\term{Scientist}}};
        \node[right=.5mm of z]     () {\fns\tikztabtwo{\term{Person}}{\term{Scientist}}};
      \end{scope}
    \end{tikzpicture}
    \\[-6pt]
    (a) & (b) \\[6pt]
    \begin{tikzpicture}[
      >=Latex,
      every node/.style={on grid,rectangle,rounded corners=1mm,draw=black,fill=lightblue,thick,inner sep=1.5mm,align=center},
      every edge/.style={draw=black,thick}
    ]
      \node [text width=\derevnode]                     (derev) {\fns\mystrut$\term{De\_revolutionibus}$};
      \node [text width=\derevnode,below=16mm of work1] (x)     {\fns\tikztabtwo{\term{FB\_Gotha\_}}{\term{Druck\_4°\_00466}}};
      \node [above right=8.5mm and 37mm of x] (y)     {\fns\tikztabtwo{\term{Johann\_}}{\term{Hommel}}};
      \node [below right=8.5mm and 37mm of x] (z)     {\fns\tikztabtwo{\term{Valentin\_}}{\term{Thau}}};
  
      \begin{scope}[%
        every node/.style={draw=none,fill=none,inner sep=.2mm,align=left}
      ]
        \path[->]
          (derev) edge node[left=1mm]           {\fns\term{exemplar}} (x)
          (x)     edge node[sloped, above=.6mm] {\fns\term{owner}}         (y)
          (x)     edge node[sloped, below]      {\fns\strut\term{owner}}   (z)
%          (y)     edge node[right=1mm]          {\fns\term{student}} (z)      
        ;
        \path[every edge/.style={draw=none}]
          (y) edge node[right=1mm] {\rule{\student}{0pt}} (z)
        ;

        \node[above=.5mm of derev] () {\fns\term{Work}};
        \node[below=.5mm of x]     () {\fns\term{Item}};
        \node[above=.5mm of y]     () {\fns\term{Scientist}};
      \end{scope}
    \end{tikzpicture}
    &
    \begin{tikzpicture}[
      >=Latex,
      every node/.style={on grid,rectangle,rounded corners=1mm,draw=black,fill=lightblue,thick,inner sep=1.5mm,align=center},
      every edge/.style={draw=black,thick}
    ]
      \node                                   (derev) {\fns\mystrut$\term{ABC}$};
      \node [below=16mm of work1]             (x)     {\fns\mystrut$\term{DEF}$};
      \node [above right=8.5mm and 37mm of x] (y)     {\fns\tikztabtwo{\term{Johann\_}}{\term{Hommel}}};
      \node [below right=8.5mm and 37mm of x] (z)     {\fns\tikztabtwo{\term{Valentin\_}}{\term{Thau}}};
  
      \begin{scope}[%
        every node/.style={draw=none,fill=none,inner sep=.2mm,align=left}
      ]
        \path[->]
%          (derev) edge node[left=1mm]           {\fns\tikztabtwo[r]{\term{has\_}}{\term{exemplar}}} (x)
%          (x)     edge node[sloped, above=.6mm] {\fns\term{has\_owner}}         (y)
%          (x)     edge node[sloped, below]      {\fns\strut\term{has\_owner}}   (z)
%          (y)     edge node[right=1mm]          {\fns\tikztabtwo[r]{\term{has\_}}{\term{student}}} (z)      
          (derev) edge node[left=1mm]           {\fns\term{exemplar}} (x)
          (x)     edge node[sloped, above=.6mm] {\fns\term{owner}}         (y)
          (x)     edge node[sloped, below]      {\fns\strut\term{owner}}   (z)
          (y)     edge node[right=1mm]          {\fns\term{student}} (z)      
        ;
          
        \node[above=.5mm of derev] () {\fns\term{Work}};
        \node[below=.5mm of x]     () {\fns\term{Item}};
        \node[right=.5mm of y]     () {\fns\term{Scientist}};
      \end{scope}
    \end{tikzpicture}
    \\[-6pt]
    (c) & (d)
  \end{tabular}
  %
  \caption{Positive (a,\,b) and negative (c,\,d) examples for query answers}
  \label{fig:example_answers}
\end{figure}

In order to identify subgraphs of a given graph that have the same structure as another given graph,
we use the standard notion of a homomorphism.
A homomorphism is a function that maps some object to another
while preserving the structure of the former.
We therefore need to define a variant of homomorphism that maps queries
to data sources. This variant is given in the following.
%
\begin{definition}
  \label{def:hmph}
  Let $\namespace=(\NO,\NC,\NR)$ be a namespace, $G = (V,E,\Nmc,\Lmc)$ a query over \namespace,
  and $G' = (V',E',\Nmc',\Lmc')$ a data source over \namespace.
  A \emph{homomorphism from $G$ to $G'$} is a map $h : V \to V'$ that satisfies the following properties.
  %
  \begin{enumerate}
    \item[\hmph{1}]
      $\Nmc(v) = \Nmc'(h(v))$ for every node $v \in V$ with $\Nmc(v) \in \NO$.
    \item[\hmph{2}]
      $\Lmc(v) \subseteq \Lmc'(h(v))$ for every node $v \in V$.
    \item[\hmph{3}]
      $\Lmc(v_1, v_2) \subseteq \Lmc'(h(v_1), h(v_2))$
      for every edge $(v_1,v_2) \in E$.
  \end{enumerate}
  %
  If $h$ is a homomorphism from $G$ to $G'$, we write $h : G \to G'$.
  If there is some homomorphism from $G$ to $G'$, we write $G \lesssim G'$.
\end{definition}
%
Property~\hmph{1} requires that a homomorphism maps each node in $G$ that is named with an object
to that node in $G'$ which is named with the same object.
Nodes named with variables in $G$ can be mapped to arbitrary nodes in $G'$.
Properties~\hmph{2} and~\hmph{3} require that homomorphisms preserve node and edge labels;
more precisely, the subset relation entails that the image of a node (or edge) under $h$
must have \emph{at least} the same labels (and may have additional labels).
Furthermore, the graph $G'$ may have additional nodes that are not among the images
of the homomorphism.

Figure~\ref{fig:example_hmph} shows a homomorphism $h$ (dashed lines)
from the query depicted in Figure~\ref{fig:graph_for_exa_query2'}
to the graph from Figure~\ref{fig:example_graph}.

\newcommand{\tikzhmph}{%
  \begin{scope}[%
    every node/.style={draw=none,fill=none,inner sep=.2mm,align=left},
    every edge/.style={densely dashed,draw=black!70,thick}
  ]
    \path[->]
      (derev) edge[out= 20,in=160]               node[below=.6mm]         {\fns$h$} (work1)
      (x)     edge[out=270,in=200]               node[above=.4mm]         {\fns$h$} (item1)
      (y)     edge[out= 30,in=150,looseness=.19] node[above=.4mm,pos=.25] {\fns$h$} (person2)
      (z)     edge[out=340,in=200,looseness=.8]  node[above=.4mm,pos=.60] {\fns$h$} (person3)
    ;
  \end{scope}
}

\begin{figure}[ht]
  \centering
  \begin{tikzpicture}[
    >=Latex,
    every node/.style={on grid,rectangle,rounded corners=1mm,draw=black,fill=lightblue,thick,inner sep=1.5mm,align=center},
    every edge/.style={draw=black,thick}
  ]
    \tikzexaquery
    \tikzexagraph[right=69mm of derev]
    \tikzhmph      
  \end{tikzpicture}
  %  
  \caption{An example homomorphism}
  \label{fig:example_hmph}
\end{figure}

We can now use homomorphisms to define the notion of an answer to a query.
%
\begin{definition}
  \label{def:query_answer}
  Let $\namespace=(\NO,\NC,\NR)$ be a namespace, $G = (V,E,\Nmc,\Lmc)$ a query over \namespace,
  and $G' = (V',E',\Nmc',\Lmc')$ a data source over \namespace.
  %
  \begin{enumerate}
    \item
      An \emph{answer to $G$ in $G'$} is a pair $(h,G'')$
      where $h$ is a homomorphism $h : G \to G'$
      and $G''$ is the subgraph of $G'$ induced by $h(V)$.
    \item
      The set of all answers to $G$ in $G'$ is called the \emph{answer set} for $G$ in $G'$
      and denoted $\ans(G,G')$.
  \end{enumerate}
\end{definition}
%
We make the following remarks concerning Definition~\ref{def:query_answer}.

%
%\begin{itemize}
%  \item
    The \emph{subgraph of $G'$ induced by $h(V)$} in Point~1 is the graph $G'' = (V'', E'', \Nmc'', \Lmc'')$
    where $V'' = h(V)$ and $E'', \Nmc'', \Lmc''$ are the restrictions of $E,\Nmc,\Lmc$ to $V''$.
    
%  \item
    The answer corresponding to the homomorphism $h$ depicted in Figure~\ref{fig:example_hmph}
    consists of $h$ and 
    the graph on the right-hand side after removal of the node \term{Nicolaus\_Copernicus} and the two adjacent edges.
    This situation is shown in Figure~\ref{fig:example_answer} with the query greyed out.
    
%  \item
    Every homomorphism determines an answer. If no homomorphism $h : G \to G'$ exists,
    then there is no answer to $G$ in $G'$, i.e., $\ans(G,G') =  \emptyset$.
    
%  \item
    From a mathematical point of view, it would suffice to simply equate answers with homomorphisms
    because $G''$ can be reconstructed from $G$ and $h$. However, we decided to make $G''$ explicit
    because we expect that this subgraph is the minimal \enquote{section} of the data that a researcher will want to inspect.
%\end{itemize}

\begin{figure}[ht]
  \centering
  \begin{tikzpicture}[
    >=Latex,
    every node/.style={on grid,rectangle,rounded corners=1mm,draw=black,fill=lightblue,thick,inner sep=1.5mm,align=center},
    every edge/.style={draw=black,thick}
  ]
    \tikzexaquery[.5]
    \tikzexagraphMinusCopernicus[right=69mm of derev]
    \tikzhmph
  \end{tikzpicture}
  %  
  \todo[defer,inline]{Use straight lines (with 90° angles) for all edges? see URLs in comments here}
  % https://tex.stackexchange.com/questions/478165/how-to-draw-tikz-paths-composed-only-of-horizontal-vertical-and-diagonal-segmen (LAST ANSWER)
  % https://github.com/Qrrbrbirlbel/tikz-extensions
  % https://tex.stackexchange.com/questions/45347/vertical-and-horizontal-lines-in-pgf-tikz
  \caption{An example answer consisting of the homomorphism $h$ and the graph $G''$ on the right-hand side}
  \label{fig:example_answer}
\end{figure}

So far, we have built a model that captures data sources and query patterns such as \exaquery{Q2}.
In this model, answering queries amounts to finding homomorphisms between graphs.
The two natural next steps consist in examining the computational properties
of the problem of finding homomorphisms
and in extending the model to capture further query patterns.
We will realise these two steps in the following sections.

% ------------------------------------------------------------------
\section{Computational Properties}
\label{sec:decision_problems}
\label{sec:decision_procedures}
\label{sec:computational_properties}

As already indicated, the problem of finding homomorphisms is well-understood in the contexts of
database theory and classical logic. In particular, the favourable computational properties of this problem
are exploited in successful database systems. We review these in the following,
building on fundamental principles of complexity theory \autocite{Arora2009}.

The computation problem of query answering in which we are interested according to our model can be 
phrased as follows: Given a query and data source, compute all answers.
More formally:

\begin{center}
  \begin{tabular}{ll}
    \hline\rule{0pt}{12pt}%
    Input:  & a query $G$ and a data source $G'$ over some namespace \namespace \\[2pt]
    Output: & $\ans(G,G')$ \\[1pt]
    \hline
  \end{tabular}
\end{center}

For the implementation of a retrieval tool that is designed to answer questions,
it is crucial to verify that this problem can be solved at all (i.e., is \emph{computable})
and efficiently so (i.e., is \emph{tractable}).%
Here, computability means that there is some algorithm
which, given an arbitrary input $(G,G')$, returns the correct answer set $\ans(G,G')$
after a finite amount of time.
Tractability means that there is even an algorithm whose runtime is strongly limited
depending on the size of its input.
Usually a \emph{polynomial} upper bound on the amount of time is
considered sufficient for tractability; that is, the runtime of the algorithm
is bounded by $c \cdot n^k$, where $n$ is the size of the input (e.g., nodes and edges in the input graphs)
and $c$ and $k$ are some constants. Problems for which an algorithm with such guaranteed time bounds exists
are called \emph{solvable in polynomial time}.
A polynomial time bound on an algorithm provides the guarantee that its runtime as a function of the input size
increases moderately for all inputs, as opposed to, e.g., an exponential function.
Therefore, problems with a polynomial-time algorithms are generally considered tractable.
However, this is still an abstract notion, given that the constants $c$ and $k$ can be arbitrary.
It remains for practical purposes to find an optimal algorithm with $c$ and $k$ as small as possible,
and which performs particularly well on those inputs that occur in the application at hand.

In order to check whether a computation problem is computable and tractable,
is it usual to resort to a slightly more abstract level,
considering the \emph{decision problem} associated to the computation problem.
Typically, \emph{decidability} (i.e., computability) and tractability of
the decision problem imply those of the computation problem;
hence it suffices to consider these.

The associated decision problem additionally considers a candidate output as an input
and asks whether that candidate is in fact a valid output.
In the case of our computation problem, the associated decision problem is the following.

\begin{center}
  \begin{tabular}{ll}
    \hline\rule{0pt}{12pt}%
    Input:  & a query $G$ over \namespace, a data source $G'$ over \namespace, \\
            & a function $h$ from $G$ to $G'$, and a subgraph $G''$ of $G'$ \\[2pt]
    Output: & \YES if $(h,G'') \in \ans(G,G')$;\quad \NOO otherwise \\[1pt]
    \hline
  \end{tabular}
\end{center}

We denote it with \problem{QA}.

\problem{QA} is strongly related to the decision problem associated to
answering (conjunctive) queries over relational databases.
This problem has been studied from a computational point of view in seminal papers
by \textcite{Vardi1982} and \textcite{Chandra1977} with the following main result.
%
\begin{theorem}
  \label{thm:cplx_CQA}
  \textup{\autocite{Vardi1982}}~
  Answers to conjunctive queries over relational databases
  can be decided in time polynomial with respect to the size of the database.
\end{theorem}
%
This result means that the decision problem associated with answering conjunctive queries is decidable
and, furthermore, tractable if the size of the query is neglected.
Given that queries are typically very small compared to the size of the data,
tractability under this assumption is sufficient for practical purposes;
in fact, it is the basis for the success of modern (relational) database engines.
As a side remark, if this assumption is dropped and the complexity with respect
to the joint size of both the database and the query is considered,
then it follows from \citeauthor{Chandra1977}'s \autocite*{Chandra1977}
results that the problem 
is intractable (under the usual reasonable complexity-theoretic assumptions)
already in the presence of binary relations.

The strong relationship between our decision problem \problem{QA}
and the classical problem of answering conjunctive queries
implies that Theorem~\ref{thm:cplx_CQA} carries over directly to \problem{QA}.
In more precise terms, both problems are mutually \emph{reducible} in polynomial time,
which can be shown very easily using standard knowledge from graph theory and first-order logic.
Without immersing into the technical definitions for standard notions in complexity theory,
we note that polynomial-time reducibility between the two problems
means that an algorithm for one of them
cqn be turned into an algorithm for the other which uses only a polynomial
(i.e., relatively small) additional amount of time.

We therefore obtain the following decidability and complexity result
for our problem \problem{QA}.
%
\begin{corollary}
  \label{cor:cplx_QA}
  \problem{QA} is decidable
  in time polynomial with respect to the size of the data source.
\end{corollary}
%
As a consequence of Corollary~\ref{cor:cplx_QA}, 
there is an algorithm for deciding \problem{QA},
and thus also for computing the answer set $\ans(G,G')$
for a given query $G$ and data source $G'$ in time polynomial
with respect to the size of $G'$.
The polynomial time bound ensures that the growth of runtime
as a function of the input size is guaranteed to be bounded by a function
of moderate growth, independently of the specific input.
However, as explained above,
the existence of a polynomial-time algorithm does not immediately imply that there
is an algorithm that performs well for the purposes of our envisaged application,
which is the implementation of a retrieval system.
Fortunately, the close relationship to conjunctive query answering over databases
comes to our aid: 
given the existence of a (very straightforward) polynomial-time reduction
of \problem{QA} to conjunctive query answering,
%there are two options for obtaining an algorithm that is efficient in practice.
\problem{QA} can be solved by either re-implementing one of the existing
successful polynomial-time algorithms for conjunctive query answering
or by implementing the reduction feeding its output directly into a relational database system,
using the latter as a \enquote{black box}.
These two options correspond nicely to the
distinction between the dynamic and static scenario
that we have contemplated in Section~\ref{subsec:insights_from_SoNAR},
and which we will expand in Chapter~\ref{chap:retrieval}.

% ------------------------------------------------------------------
\section{Discussion of the Basic Model and Extensions}
\label{sec:modelling_discussion}

In Section~\ref{sec:modelling}, we have developed a basic model
that is conceptually simple and easy to implement.
We have shown how the exemplary query pattern \exaquery{2}
can be modelled, and we have indicated how implementations can be obtained.
What is missing is a wider analysis of the scope and suitability
of this model for capturing queries and data sources relevant
for provenance research.
It is clear that this analysis cannot be performed comprehensively
within the scope of a master's thesis.
However, we can estimate the extent of generality of our model
by relating it to the exemplary query patterns \exaquery{1}--\exaquery{9}.
In doing so, we will understand the limitations of the model
and be able to sketch possible extensions.
This is the aim of this section.

In the following discussion, we will continue to use \exaquery{1}--\exaquery{9}
as \emph{patterns}. That is, we will abstain from instantiating the variables $X$ and $Y$
because the considerations are independent of the specific instantiations.
As a consequence, the drawn graphs will contain nodes labelled with uppercase letters $X,Y$
as well as lowercase letters $x,y$ (or similar).
Only the latter type of nodes represents answer variables and anonymous variables;
the former represents specific objects, for which $X$ and $Y$ are used as placeholders.
We continue to distinguish these types of nodes by their shape and colour as in Figure~\ref{fig:graph_for_exa_query2'},
using blue rectangles for objects, yellow circles for answer variables, and orange diamonds 
for anonymous variables.

% - - - - - - - - - - - - - - - - - - - - - - - - - - - - - - - - -
\subsection{The Basic Model}

The basic model that we have so far developed captures not just the query pattern \exaquery{2}
but also \exaquery{4}, as well as \exaquery{1} and \exaquery{5} if we leave out the year(s) for the moment, i.e.,
the following query patterns.
%
\begin{enumerate}
  \item[\exaquery{1$^-$}]
    Who read
    work $X$, and
    in which manifestation?
%    and in which year?
  \item[\exaquery{4\phantom{$^-$}}]
    Which items from collection $X$ were passed on by its owner to a family member?
  \item[\exaquery{5$^-$}]
    Which items from the holdings of library $X$ were acquired from bookseller $Y$?
%    between 1933 and 1945?
  \item[\exaquery{8}\phantom{$^-$}]
    Which libraries own the items once owned by person $X$?
\end{enumerate}
%
In the case of \exaquery{1}, the relation \term{reads} is most certainly not recorded in provenance entries
in any bibliographic data source. Therefore, we need to resort to the \emph{substitute information}
(see Section~\ref{sec:HNA+SoNAR}) provided by the \term{owner} relation.
Now these two query patterns
can be modelled as shown in Figure~\ref{fig:modelling_Q145}.

\begin{figure}[ht]
  \centering
  \begin{tikzpicture}[
    >=Latex,baseline=0pt,
    every node/.style={on grid,rectangle,rounded corners=1mm,draw=black,fill=lightblue,thick,inner sep=1.5mm},
    every edge/.style={draw=black,thick}
  ]
    % ----- Q1- main nodes
    \node                             (Q1X)  {\fns\mystrut$~~~~X~~~~$};
    \node [anovar,below=16mm of Q1X]  (Q1y1) {\fns\mystrut$y_1$};
    \node [ansvar,below=16mm of Q1y1] (Q1x1) {\fns\mystrut$x_1$};
    \node [anovar,below=16mm of Q1x1] (Q1y2) {\fns\mystrut$y_2$};
    \node [ansvar,right=22mm of Q1y2] (Q1x2) {\fns\mystrut$x_2$};

    % ----- Q4 main nodes
    \node [right=44mm of Q1X]                        (Q4X)  {\fns\mystrut$~~~~X~~~~$};
    \node [ansvar,below=22mm of Q4X]                 (Q4x)  {\fns\mystrut$x$};
    \node [anovar,above right=7.4mm and 20mm of Q4x] (Q4y1) {\fns\mystrut$y_1$};
    \node [anovar,below right=7.4mm and 20mm of Q4x] (Q4y2) {\fns\mystrut$y_2$};

    % ----- Q5- main nodes
    \node [right=56mm of Q4X]        (Q5X) {\fns\mystrut$~~~~X~~~~$};
    \node [ansvar,below=14mm of Q5X] (Q5x) {\fns\mystrut$x$};
    \node [right=25mm of Q5x]        (Q5Y) {\fns\mystrut$~~~~Y~~~~$};

    % ----- Q8 main nodes
    \node [ansvar,below=34mm of Q5X] (Q8x) {\fns\mystrut$x$};
    \node [anovar,below=14mm of Q8x] (Q8y) {\fns\mystrut$y$};
    \node [right=25mm of Q8y]        (Q8X) {\fns\mystrut$~~~~X~~~~$};

    \begin{scope}[%
      every node/.style={draw=none,fill=none,inner sep=.2mm},
    ]
      \path[->]
        % ----- Q1- edges
        (Q1X)  edge[shorten >= -1.5pt] node[right=1mm]          {\fns\term{realization}} (Q1y1)
        (Q1y1) edge[shorten <= -1.5pt] node[right=1mm,pos=.4]   {\fns\term{embodiment}}  (Q1x1)
        (Q1x1) edge[shorten >= -1.5pt] node[right=1mm]          {\fns\term{exemplar}}    (Q1y2)
        (Q1y2) edge[shorten <= -1.5pt] node[above=.6mm,pos=.46] {\fns\term{owner}}       (Q1x2)

        % ----- Q4 edges
        (Q4X)  edge                                       node[right=1mm,pos=.4]   {\fns\term{part}}         (Q4x)
        (Q4x)  edge                                       node[sloped, above=.6mm] {\fns\term{owner}}        (Q4y1)
        (Q4x)  edge                                       node[sloped, below=.6mm] {\fns\term{owner}}        (Q4y2)
        (Q4y1) edge[shorten >= -1.5pt, shorten <= -1.5pt] node[right=1mm]          {\fns\term{familyMember}} (Q4y2)

        % ----- Q5- edges
        (Q5X) edge node[right=1mm,pos=.4]   {\fns\term{holds}}    (Q5x)
        (Q5x) edge node[above=.6mm,pos=.46] {\fns\term{provider}} (Q5Y)

        % ----- Q8 edges
        (Q8x) edge[shorten >= -1.5pt] node[right=1mm,pos=.4]   {\fns\term{holds}} (Q8y)
        (Q8y) edge[shorten <= -1.5pt] node[above=.6mm,pos=.46] {\fns\term{owner}} (Q8X)
      ;
        
      % ----- Q1- node labels
      \node[right=.5mm of Q1X]  ()                 {\fns\term{Work}};
      \node[right=.5mm of Q1y1] ()                 {\fns\term{Expression}};
      \node[right=.5mm of Q1x1] ()                 {\fns\term{Manifestation}};
      \node[below right=-.5mm and .5mm of Q1y2] () {\fns\term{Item}};
      \node[above=.5mm of Q1x2] ()                 {\fns\term{Person}};

      % ----- Q4 node labels
      \node[right=.5mm of Q4X] () {\fns\term{Collection}};
      \node[below=.5mm of Q4x] () {\fns\term{Item}};

      % ----- Q5- node labels
      \node[right=.5mm of Q5X] () {\fns\term{Library}};
      \node[below=.5mm of Q5x] () {\fns\term{Item}};
      \node[below=.5mm of Q5Y] () {\fns\term{ResponsibleEntity}};

      % ----- Q8 node labels
      \node[right=.5mm of Q8x] () {\fns\term{Library}};
      \node[below=.5mm of Q8y] () {\fns\term{Item}};
      
      % ----- labels with query numbers
      \node[below=6mm of $(Q1y2)!.6!(Q1x2)$] () {\exaquery{1$^-$}};
      \node[below=7mm of $(Q4x)!.4!(Q4y2)$]  () {\exaquery{4}};
      \node[below=7mm of $(Q5x)!.4!(Q5Y)$]   () {\exaquery{5$^-$}};
      \node[below=6mm of $(Q8y)!.4!(Q8X)$]   () {\exaquery{8}};
    \end{scope}
  \end{tikzpicture}
    
  \caption{Graph representations of example query patterns \exaquery{1$^-$},\exaquery{4}, \exaquery{5$^-$}, and \exaquery{8}}
  \label{fig:modelling_Q145}
\end{figure}

% - - - - - - - - - - - - - - - - - - - - - - - - - - - - - - - - -
\subsection{Relations with Attributes and Relations of Higher Arity}

The full wording of \exaquery{1} and~\exaquery{5} includes references to years
as attributes of relationships.
%
\begin{enumerate}
  \item[\exaquery{1}]
    Who read %\todo[color=red!30]{read $\neq$ own; make clear what is meant}
    work $X$, in which manifestation and \emph{in which year}?
  \item[\exaquery{5}]
    Which items from the holdings of library $X$ were acquired from bookseller $Y$
    \emph{between 1933 and 1945}?
\end{enumerate}
%
In~\exaquery{5}, the year is specified as an attribute of the \term{provider} relation
while, in~\exaquery{1}, it acts as an additional answer variable that represents
an attribute of the \term{owner} relation.
In order to capture~\exaquery{1} and~\exaquery{5} fully,
it is necessary to extend the model in a way that relationships can have attributes.
This is straightforward in the graph representation for both graphs and queries, as the attributes can be added to
the respective edge labels. This is shown in Figure~\ref{fig:attributes_on_relships}
for the two respective edges of~\exaquery{1} and~\exaquery{5}.

\begin{figure}[ht]
  \centering
  \begin{tabular}{@{}c@{\hspace*{20mm}}c@{}}
    \begin{tikzpicture}[
      >=Latex,baseline=0pt,
      every node/.style={on grid,rectangle,rounded corners=1mm,draw=black,fill=lightblue,thick,inner sep=1.5mm},
      every edge/.style={draw=black,thick}
    ]
      \node [anovar]                  (y2) {\fns\mystrut$y_2$};
      \node [ansvar,right=28mm of y2] (x2) {\fns\mystrut$x_2$};
  
      \begin{scope}[%
        every node/.style={draw=none,fill=none,inner sep=.2mm},
      ]
        \path[->]
          (y2) edge[shorten <= -1.5pt] node[above=.6mm,pos=.46] {\fns\term{owner}\mybold{{\boldmath\,:\,$x_3$}}} (x2)
        ;
          
        \node[below right=-.5mm and .5mm of y2] () {\fns\term{Item}};
        \node[above=.5mm of x2] ()                 {\fns\term{Person}};
      \end{scope}
    \end{tikzpicture}
    &
    \begin{tikzpicture}[
      >=Latex,baseline=0pt,
      every node/.style={on grid,rectangle,rounded corners=1mm,draw=black,fill=lightblue,thick,inner sep=1.5mm},
      every edge/.style={draw=black,thick}
    ]
      \node [ansvar]          (x) {\fns\mystrut$x$};
      \node [right=37mm of x] (Y) {\fns\mystrut$~~~~Y~~~~$};
  
      \begin{scope}[%
        every node/.style={draw=none,fill=none,inner sep=.2mm},
      ]
        \path[->]
          (x) edge node[above=.6mm,pos=.46] {\fns\term{provider}\mybold{\,:\,1935}} (Y)
        ;
          
        \node[below=.5mm of x] () {\fns\term{Item}};
        \node[below=.5mm of Y] () {\fns\term{ResponsibleEntity}};
      \end{scope}
    \end{tikzpicture}
    \\
%    \rule{0pt}{16pt}
    \exaquery{1}
    &
    \exaquery{5}~~~~~
  \end{tabular}

  \caption{Capturing attributes on relationships}
  \label{fig:attributes_on_relships}
\end{figure}

On the formal side, the notion of a graph (Definition~\ref{def:ld_graph})
needs to be extended such that the labelling function \Lmc maps each edge to a set
that consists not necessarily only of relation names (for relationships without attributes),
but which may also contain pairs of relation names and literals/objects/variables (for relationships with attributes).
The notion of a homomorphism (Definition~\ref{def:hmph}) does not need to be changed,
as the current formulation of Condition~\hmph{3} continues to ensure that edge labels are preserved.

Binary relations with attributes are in fact a special case of ternary relations: e.g., \term{owner} in \exaquery{1}
is a relation between items, persons, and literals (year numbers).
As the computational properties discussed in Section~\ref{sec:computational_properties}
hold for relations of higher arity as well, this modest extension of the model can be implemented
along with the basic model.

Ternary relations are also necessary to model \exaquery{9} (and the supporting data):
%
\begin{enumerate}
  \item[\exaquery{9}]
    Where did person $X$ acquire items and did they know the previous owners?
\end{enumerate}
%
In this case, the relation \term{acquires} needs to involve persons, items, and sellers,
and this constellation can no longer be visualised as a graph in a straightforward way.
A possible extension is the notion of a \emph{hypergraph} \autocite[§7.1]{Voloshin2009},
which allows edges of arbitrary arity.
While the mathematical object can be defined analogously to standard graphs,
it does no longer lend itself to intuitive visualisations.
As already mentioned, the notion of a homomorphism and the computational properties
carry over directly.

Another use case for relations of higher arity is the inclusion of metadata provenance,
i.e., information on the origins of a unit of data (such as a concept membership or relationship).
In our scenario, it makes sense to distinguish two kinds of
metadata provenance. The first is information provided by some data sources such as the \gls{GND}
(see Section~\ref{subsec:SoNAR_reports}), referring to original data sources one whose grounds
the respective data unit was entered.
The second is the a reference to the data source from which the data unit entered
the federated data source $G'$ against which the query $G$ is evaluated
(see Sections~\ref{subsec:insights_from_SoNAR} and~\ref{subsec:data_provenance}).
Both kinds of provenance information are reflected only in the data source
and not in the query, and they play a role only in the retrieved answers,
enabling users to consult the original data sources for further research.

% - - - - - - - - - - - - - - - - - - - - - - - - - - - - - - - - -
\subsection{Answer Variables for Sets of Objects}

Query pattern~\exaquery{3} contains a reference to a \emph{set} of objects:
%
\begin{enumerate}
  \item[\exaquery{3}]
    What are the relationships between the recipients of manifestation $Y$ of work $X$?
\end{enumerate}
%

%\begin{figure}[ht]
\begin{wrapfigure}[14]{o}{6.4cm}
  \centering
  \begin{tikzpicture}[
    >=Latex,baseline=0pt,
    every node/.style={on grid,rectangle,rounded corners=1mm,draw=black,fill=lightblue,thick,inner sep=1.5mm},
    every edge/.style={draw=black,thick}
  ]
    \node                             (Q3X)  {\fns\mystrut$~~~~X~~~~$};
    \node [anovar,below=14mm of Q3X]  (Q3y1) {\fns\mystrut$y_1$};
    \node [       below=14mm of Q3y1] (Q3Y)  {\fns\mystrut$~~~~Y~~~~$};
    \node [anovar,below=14mm of Q3Y]  (Q3y2) {\fns\mystrut$y_2$};
    \node [setvar,right=22mm of Q3y2] (Q3x)  {\fns\mystrut$\overline{x}$};

    \begin{scope}[%
      every node/.style={draw=none,fill=none,inner sep=.2mm},
    ]
      \path[->]
        (Q3X)  edge[shorten >= -1.5pt] node[right=1mm]          {\fns\term{realization}} (Q3y1)
        (Q3y1) edge[shorten <= -1.5pt] node[right=1mm,pos=.4]   {\fns\term{embodiment}}  (Q3Y)
        (Q3Y)  edge[shorten >= -1.5pt] node[right=1mm]          {\fns\term{exemplar}}    (Q3y2)
        (Q3y2) edge[shorten <= -1.5pt] node[above=.6mm,pos=.46] {\fns\term{owner}}       (Q3x)
      ;
        
      % ----- Q3- node labels
      \node[right=.5mm of Q3X]  ()                 {\fns\term{Work}};
      \node[right=.5mm of Q3y1] ()                 {\fns\term{Expression}};
      \node[right=.5mm of Q3Y]  ()                 {\fns\term{Manifestation}};
      \node[below right=-.5mm and .5mm of Q3y2] () {\fns\term{Item}};
      \node[above=.5mm of Q3x]  ()                 {\fns\term{Person}};
    \end{scope}
  \end{tikzpicture}

%  \caption{A graph representing example query pattern \exaquery{3}}
  \caption{Graph representation of \exaquery{3}}
  \label{fig:modelling_Q3}
\end{wrapfigure}
%\end{figure}

If we use the substitute information \enquote{ownership} for \enquote{reception},
then the expected answer to \exaquery{3} is a set of owners,
together with the relationships between them---%
or, more precisely, the subgraph of the data source induced by the set of all owners.
In order to capture this idea,
our basic model needs to be extended by \emph{set variables} representing sets of objects
and \emph{set nodes} corresponding to set variables.
In the case of \exaquery{3}, the set node acts nearly as an answer variable
because the query asks for the relationships between the set of owners.
In order to distinguish set variables and nodes from ordinary variables and nodes in the visualisation,
we use overlined letters and octagonal shapes.
Figure~\ref{fig:modelling_Q3} depicts the graph for \exaquery{3}.

Query pattern \exaquery{9} (see above) also requires a set variable
that represents the \enquote{previous owners}. In contrast to \exaquery{3},
the query does not ask for the relationships between the objects in that set
but for the relationships from these objects to person $X$.
These relationships are automatically included in each induced subgraph that is part of a query answer,
and they need to be evaluated manually by the user
in order to answer the yes/no question involving the relationship \term{know},
which is very likely not recorded explicitly in data sources.

In order to capture set variables in our definition of query answers,
the notion of a homomorphism needs to be extended such that
every set node in the query graph $G$ is mapped to a set of nodes in $G'$,
under preservation conditions analogous to~\hmph{2} and~\hmph{3}.
While all these extensions are straightforward,
their effect on the complexity of query answering is not obvious.
Since the number of subsets of a set is exponential in the number of its
elements, it is possible that there is no longer an algorithm
that requires only polynomial time (with respect to the size of the data).
This question remains to be investigated.

Finally, when our basic model is extended with set variables,
it can express Requirements \SonarR{016}--\SonarR{018} from the 
SoNAR report \autocite{Fangerau2022}, which are listed in Section~\ref{subsec:SoNAR_reports},
as queries.
The only additional feature required is the use of unlabelled edges in queries
in order to capture the (deliberately) unspecified relation \enquote{connected to}.
That addition is minor: homomorphisms simply need to be required to map unlabelled
edges in the query graph $G'$ to arbitrary edges in the data source graph $G$.
The computational properties are unaffected because 
query answering with unlabelled edges can be reduced to the basic query answering problem
by adding a fresh relation name to all unlabelled edges in $G'$ and all edges in $G$.

In contrast to Requirements \SonarR{016}--\SonarR{018},
most of the other requirements that 
we listed in Section~\ref{subsec:SoNAR_reports}
because they relate to the topic of this thesis
are of a rather exploratory nature
(in particular, \SonarR{029}, \SonarR{030}, and \SonarR{061}).
They are therefore outside the scope of our model for query answering.
The issue of exploration should be dealt with in a separate
thesis or paper.

% - - - - - - - - - - - - - - - - - - - - - - - - - - - - - - - - -
\subsection{Further Extensions}

\dots

% - - - - - - - - - - - - - - - - - - - - - - - - - - - - - - - - -
\subsection{Substitute Information}

\dots





\dots

\todo[think,inline]{TODO: Discuss further extensions:}
%
\begin{itemize}
  \item
    attributes on relationships:
    %
    \begin{itemize}
      \item
        sketch idea: e.g., provide year for relationship \term{owner} -- example: \enquote{passed on to} requires descending year numbers \emph{and} no successor with intermediate year number
      \item
        solution: quads instead of triples (in \gls{RDF} speak); add attributes to the graph model? (Markus Krötzsch's work?)
      \item
        explain difficulties: more complex formal machinery (def.\ of graphs, queries, and matches); missing data, e.g.:
        %
        \begin{itemize}
          \item
            From which year \emph{to which year} did person $X$ own item $Y$?
          \item
            Was person $Z$ a student of person $X$'s \emph{at the point in time when $X$ passed the item on to $Z$}?
        \end{itemize}
      \item
        discuss usefulness: false positives due to incomplete data as discussed in Section~\ref{sec:quality_of_answers}
        $\leadsto$ manual inspection is necessary anyway; attributes may still help hide false answers
    \end{itemize}
    %
  \item
    disjunctions, e.g., on edges (\enquote{student or collaborator} in \exaquery{2}) $\leadsto$ PEQs?
  \item
    Discuss \gls{SoNAR} requirements R0xx:
    
    Some requirements can be addressed directly with our queries, e.g., R006! (R016?!?)

    R016 seems to require the \enquote{$\top$-role} in queries!

    Edge weights;

    E.g. R029, 30, 61: explorative (for every node/edge the available attributes) vs. ??? (ask query knowing those attributes)
    
  \item
    Discuss further SoNAR insights, e.g., substitute relations
    
\end{itemize}
