% !TeX spellcheck = de_DE
% =================================================================
\chapter{Research Data Management Plan (in German)}
\label{chap:DMP}

\todo[defer,inline]{list data and link to repository; paste and fill in data management plan)}

see also \url{https://www.dfg.de/foerderung/grundlagen_rahmenbedingungen/forschungsdaten/empfehlungen/index.html}


% command for excluding (appendix) section from toc
\newcommand{\nocontentsline}[3]{}
\newcommand{\tocless}[2]{\bgroup\let\addcontentsline=\nocontentsline#1{#2}\egroup}

% Number also subsubsections
\setcounter{secnumdepth}{\subsubsectionnumdepth}
% ------------------------------------------------------------------
\tocless\section{Allgemein}

% - - - - - - - - - - - - - - - - - - - - - - - - - - - - - - - - -
\tocless\subsection{Thema}

% ..................................................................
\tocless\subsubsection{Wie lautet die primäre Forschungsfrage der Abschlussarbeit?}

Wie können Provenienzbeziehungen modelliert und maschinell gestützt aufgefunden werden?

% ..................................................................
\tocless\subsubsection{Bitte geben Sie einige Schlagwörter zur Forschungsfrage bzw. Fragestellung an.}

DDC:\footnote{\url{https://deweysearchde.pansoft.de/webdeweysearch/mainClasses.html?catalogs=DNB}}
%
\begin{description}
%  \item[003.3]
%    Computermodellierung und Computersimulation 
  \item[020.0113]
    Computermodellierung in Bibliotheks- \& Informationswissenschaften
  \item[???]
    *** Schauen, wie verwandte Arbeiten in der DNB in DDC klassifiziert sind! ***
\end{description}
GND:\footnote{\url{https://gnd.network/Webs/gnd/DE/Home/home_node.html}} *********


\par\bigskip
**** CONTINUE ****

%1.1.3. Welchen Regeln oder Richtlinien (HU) zum Umgang mit den in der Abschlussarbeit erhobenen Forschungsdaten folgen Sie für den DMP? Bitte referenzieren Sie diese hier inklusive Version bzw. Veröffentlichungsjahr.
%2. Inhaltliche Einordnung
%NB: Bitte beschreiben Sie jeden Datensatztyp oder Datensammlung einzeln in dem jeweiligen Kapitel, wo sinnvoll.
%2.1. Datensatz
%2.1.1 Um welche Arten von Daten handelt es sich? Bitte in wenigen Zeilen kurz beschreiben.
%2.2 Datenursprung
%2.2.1 Werden die Daten selbst erzeugt oder nachgenutzt?
%2.2.2 Wenn die Daten nachgenutzt werden, wer hat die Daten erzeugt? Bitte mit Angabe des Identifiers, falls vorhanden, z.B. DOI3.
%2.3. Reproduzierbarkeit
%2.3.1 Sind die Daten reproduzierbar, d. h. ließen sie sich, wenn sie verloren gingen, erneut erstellen oder erheben?
%2.4 Nachnutzung
%2.4.1 Für welche Personen, Gruppen oder Institutionen könnte dieser Datensatz (für die Nachnutzung) von Interesse sein? Für welche Szenarien ist dies denkbar?
%2 Hier gern auch ein anerkanntes (Fach)Vokabular nutzen, die Dewey Dezimalklassifikation (DDC deutsch, https://deweysearchde.pansoft.de/webdeweysearch/mainClasses.html?catalogs=DNB) oder Gemeinsame Normdatei (GND, https://gnd.network/Webs/gnd/DE/Home/home_node.html) bieten sich für fachübergreifende Terme an. Bitte angeben, ob und falls ja, welches Vokabular benutzt wurde bzw. ob freie Schlagwortvergabe angewendet wird.
%3 DOI, https://www.doi.org/
%    Datum der letzten Aktualisierung 08.12.20213. Technische Einordnung
%3.1 Datenerhebung
%3.1.1 Wann erfolgt(e) die Erhebung bzw. Erstellung der Daten?
%3.1.2 Wann erfolgt(e) die Datenbereinigung / -aufbereitung bzw. Datenanalyse?
%3.1 Datengröße
%3.1.1 Was ist die tatsächliche oder erwartete Größe der Daten(typen)? 3.2 Formate
%3.2.1 In welchen Formaten4 liegen die Daten vor?
%3.3 Werkzeuge
%3.3.1 Welche Instrumente, Software, Technologien oder Verfahren werden zur Erzeugung, Erfassung, Bereinigung, Analyse und/oder Visualisierung der Daten genutzt? Bitte (falls möglich) mit Versionsnummer und Referenz in Form einer Adresse jeweils angeben.
%3.3.2 Welche Software, Verfahren oder Technologien sind notwendig, um die Daten zu nutzen?
%3.4 Versionierung
%3.4.1 Werden verschiedene Versionen der Daten erzeugt (z. B. durch verschiedene Weiterbearbeitungsprozesse bzw. Bereinigung von Daten)?
%4. Datennutzung
%4.1 Datenorganisation
%4.1.1 Gibt es eine Strategie zur Benennung der Daten? Wenn ja, bitte skizzieren Sie sie kurz.
%4.2 Datenspeicherung und -sicherheit
%4.2.1 Wer darf (zukünftig) auf die Daten zugreifen?
%4.2.2 Wie und wie oft werden Backups der Daten erstellt?
%4 Vgl. z. B. DROID zur Format-Erkennung, http://digital-preservation.github.io/droid/
%Datum der letzten Aktualisierung 08.12.2021
%  4.3 Interoperabilität
%4.3.1 Sind die Datenformate im Sinne der FAIR-Prinzipien interoperabel, d.h. geeignet für den Datenaustausch und die Nachnutzung zwischen bzw. von unterschiedlichen Forschenden, Institutionen, Organisationen und Ländern?
%4.4 Weitergabe und Veröffentlichung
%4.4.1 Ist es geplant, die Daten nach Abgabe der Abschlussarbeit zu veröffentlichen oder zu teilen?
%4.4.2 Wenn nicht, skizzieren Sie kurz rechtliche und/oder vertragliche Gründe und freiwillige Einschränkungen.
%4.4.3 Wenn ja, unter welchen Nutzungsbedingungen oder welcher Lizenz sollen die Daten veröffentlicht bzw. geteilt werden?
%4.5 Qualitätssicherung
%4.5.1 Welche Maßnahmen zur Qualitätssicherung (z. B. Plausibilitätsprüfung von Datenwerten) werden für die Daten ergriffen?
%4.6 Datenintegration
%4.6.1 Falls Daten aus verschiedenen Quellen (z. B. Anpassung Skalierung, Zeiträume, Ortsangaben) integriert werden, wie wird dies gewährleistet?
%5. Metadaten und Referenzierung
%5.1 Metadaten
%5.1.1 Welche Informationen sind für Außenstehende notwendig, um die Daten zu verstehen (d. h. ihre Erhebung bzw. Entstehung, Analyse sowie die auf ihrer Basis gewonnenen Forschungsergebnisse nachvollziehen) und nachnutzen zu können?
%5.1.2 Welche Standards, Ontologien, Klassifikationen etc. werden zur Beschreibung der Daten und Kontextinformation genutzt?
%6. Rechtliche und ethische Fragen
%6.1 Personenbezogene Daten
%6.1.1 Enthalten die Daten personenbezogene Informationen? 6.2 Sensible Daten
%Datum der letzten Aktualisierung 08.12.2021
%6.2.1 Enthalten die Forschungsdaten besondere Kategorien personenbezogener Daten nach Artikel 9 der DSGVO (“Angaben über die rassische und ethnische Herkunft, politische Meinungen, religiöse oder philosophische Überzeugungen, Gewerkschaftszugehörigkeit, Gesundheit oder Sexualleben”)5?
%6.2.2 Werden die Daten anonymisiert oder pseudonymisiert?
%6.2.3 Haben Sie eine "informierte Einwilligung" der Betroffenen eingeholt? Fügen Sie bitte ein Template der Einverständniserklärung als Anlage bei.
%6.2.4 Wenn keine "informierte Einwilligung" eingeholt wird, begründen Sie dies bitte.
%6.2.5 Wo und wie sind die "informierten Einwilligungen" abgelegt?
%6.2.6 Bis wann werden die (un-anonymisierten bzw. un-pseudonymisierten) Originaldaten spätestens sicher vernichtet?
%6.3 Urheber- oder verwandte Schutzrechte
%6.3.1 Werden Daten genutzt und/oder erstellt, die durch Urheber- oder verwandte Schutzrechte geschützt sind?
%7. Speicherung und Langzeitarchivierung
%7.1 Wo werden die Daten (einschließlich Metadaten, Dokumentation und ggf. relevantem Code bzw. relevanter Software) während Phase der Erarbeitung der Abschlussarbeit gespeichert?
%7.2 Wo werden die Daten (einschließlich Metadaten, Dokumentation und ggf. relevantem Code bzw. relevanter Software) nach dem Ende der Abschlussarbeit gespeichert bzw. archiviert6?
%7.3 Handelt es sich dabei um ein zertifiziertes Repositorium oder Datenzentrum (z.B. durch das CoreTrustSeal7, nestor-Siegel8 oder ISO 163639)? (Wurden mehrere Langzeitarchivierungsoptionen ausgewählt, kann die Frage bejaht werden, wenn dies auf mindestens eine der Optionen zutrifft).

