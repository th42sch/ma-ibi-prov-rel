% !TeX spellcheck = de_DE
% =================================================================
\chapter{Research Data Management Plan (in German)}
\label{chap:DMP}

\todo[defer,inline]{list data and link to repository; paste and fill in data management plan)}

see also \url{https://www.dfg.de/foerderung/grundlagen_rahmenbedingungen/forschungsdaten/empfehlungen/index.html}


%% command for excluding (appendix) section from toc
%\newcommand{\nocontentsline}[3]{}
%\newcommand{\tocless}[2]{\bgroup\let\addcontentsline=\nocontentsline#1{#2}\egroup}
\newcommand{\tocless}{\relax}

% Number also subsubsections
\setcounter{secnumdepth}{\subsubsectionnumdepth}
% ------------------------------------------------------------------
\section{Allgemein}

% - - - - - - - - - - - - - - - - - - - - - - - - - - - - - - - - -
\subsection{Thema}

% ..................................................................
\subsubsection{Wie lautet die primäre Forschungsfrage der Abschlussarbeit?}

Wie können Provenienzbeziehungen modelliert und maschinell gestützt aufgefunden werden?

% ..................................................................
\subsubsection{Bitte geben Sie einige Schlagwörter zur Forschungsfrage bzw. Fragestellung an.}

\begin{itemize}
  \item 
    DDC:\footnote{\url{https://deweysearchde.pansoft.de/webdeweysearch/mainClasses.html?catalogs=DNB}}
    %
%    \begin{description}
%      \item[005.72]
%        Datenaufbereitung und Datenrepräsentation
%      \item[006.332]
%        Wissensrepräsentation
%      \item[020.0113]
%        Computermodellierung in Bibliotheks- \& Informationswissenschaften
%    \end{description}  
    \begin{itemize}
      \item
        005.72~
        Datenaufbereitung und Datenrepräsentation
      \item
        006.332~
        Wissensrepräsentation
      \item
        020.0113~
        Computermodellierung in Bibliotheks- \& Informationswissenschaften
    \end{itemize}  
%  \item 
%    GND:\footnote{\url{https://gnd.network/Webs/gnd/DE/Home/home_node.html}} *********
  \item 
    2012 ACM Computing Classification System:\footnote{\url{https://dl.acm.org/ccs}}
%    
%    \begin{CCSXML}
%    <ccs2012>
%    <concept>
%    <concept_id>10002951.10003317.10003347.10003348</concept_id>
%    <concept_desc>Information systems~Question answering</concept_desc>
%    <concept_significance>500</concept_significance>
%    </concept>
%    <concept>
%    <concept_id>10002951.10003317.10003347.10003352</concept_id>
%    <concept_desc>Information systems~Information extraction</concept_desc>
%    <concept_significance>300</concept_significance>
%    </concept>
%    </ccs2012>
%    \end{CCSXML}
%    
%    \ccsdesc[500]{Information systems~Question answering}
%    \ccsdesc[300]{Information systems~Information extraction}
%    
    \begin{itemize}
      \item 
        Information systems / Information retrieval / Retrieval tasks and goals / Question answering
      \item 
        Information systems / Information retrieval / Retrieval tasks and goals / Information extraction
    \end{itemize}
\end{itemize}

% ..................................................................
\subsubsection{Welchen Regeln oder Richtlinien (HU) zum Umgang mit den in der Abschlussarbeit erhobenen Forschungsdaten folgen Sie für den DMP? Bitte referenzieren Sie diese hier inklusive Version bzw. Veröffentlichungsjahr.}

Institut für Bibliotheks- und Informationswissenschaft:
Leitlinie zum Umgang mit Forschungsdaten in Abschlussarbeiten.
Beschlossen im Institutsrat des IBI am 08.12.2021, in Kraft getreten am 01.02.2022.%
\footnote{\url{https://www.ibi.hu-berlin.de/de/studium/rundumdasstudium/fdm-fuer-studierende}}

% ------------------------------------------------------------------
\section{Inhaltliche Einordnung}

\mybold{NB: Bitte beschreiben Sie jeden Datensatztyp oder Datensammlung einzeln in dem jeweiligen Kapitel, wo sinnvoll.}

% - - - - - - - - - - - - - - - - - - - - - - - - - - - - - - - - -
\subsection{Datensatz}

% ..................................................................
\subsubsection{Um welche Arten von Daten handelt es sich? Bitte in wenigen Zeilen kurz beschreiben.}

Für die Literaturstudie und Analyse der Datenquellen wurden folgende Daten gesammelt:
%
\begin{enumerate}[(1)]
  \item
    bibliographische Metadaten zu relevanten Arbeiten aus der Literatur 
  \item
    für ausgewählte Arbeiten: Volltexte im PDF-Format
  \item
    Metadaten von relevanten Webseiten
  \item
    statistische Angaben zu Datenquellen, z.\,B. die Anzahl der darin enthaltenen Datensätze
\end{enumerate}

% - - - - - - - - - - - - - - - - - - - - - - - - - - - - - - - - -
\subsection{Datenursprung}

% ..................................................................
\subsubsection{Werden die Daten selbst erzeugt oder nachgenutzt?}

\begin{enumerate}[(1)]
  \item
    teilweise nachgenutzt aus Portalen, teilweise selbst erzeugt
  \item
    heruntergeladen von Verlags- und Aggregatorplattformen
  \item
    selbst erzeugt
  \item
    aus der Literatur und von Webseiten selbst extrahiert
\end{enumerate}

% ..................................................................
\subsubsection{Wenn die Daten nachgenutzt werden, wer hat die Daten erzeugt? Bitte mit Angabe des Identifiers, falls vorhanden, z.B. DOI.}

Volltexte und deren Metadaten wurden von diversen Portalen heruntergeladen, darunter:
%
\begin{itemize}
  \item
    \url{https://hu-berlin.hosted.exlibrisgroup.com/primo-explore/search?vid=hub_ub}
  \item
    \url{https://www.sciencedirect.com/}
  \item
    \url{https://ebookcentral.proquest.com/lib/huberlin-ebooks/home.action}
  \item
    \url{https://www.tandfonline.com/}
  \item
    \url{https://dblp.uni-trier.de/}
  \item
    \url{https://gvk.k10plus.de}
\end{itemize}

% - - - - - - - - - - - - - - - - - - - - - - - - - - - - - - - - -
\subsection{Reproduzierbarkeit}

% ..................................................................
\subsubsection{Sind die Daten reproduzierbar, d. h. ließen sie sich, wenn sie verloren gingen, erneut erstellen oder erheben?}

Die Daten sind nur bedingt reproduzierbar:
Anhand des Literaturverzeichnisses der Arbeit lassen sich alle Quellen wiederfinden,
aber bei den meisten Webseiten besteht die Gefahr, dass der Inhalt sich ändert oder
die Webseite inaktiv wird. Letzteres kann auch bei denjenigen der wissenschaftlichen
Arbeiten passieren, die nur online verfügbar sind.

% - - - - - - - - - - - - - - - - - - - - - - - - - - - - - - - - -
\subsection{Nachnutzung}

% ..................................................................
\subsubsection{Für welche Personen, Gruppen oder Institutionen könnte dieser Datensatz (für die Nachnutzung) von Interesse sein? Für welche Szenarien ist dies denkbar?}

Der Datensatz könnte von Interesse sein für Forschende, die sich mit derselben Thematik befassen
und einige Aspekte aus oder neben dieser Arbeit vertiefen möchten.
Er könnte auch als (unvollständige) Basis für einen Übersichtsartikel über bibliographische Datenquellen
oder digitale Provenienzforschung dienen.

% ------------------------------------------------------------------
\section{Technische Einordnung}

% - - - - - - - - - - - - - - - - - - - - - - - - - - - - - - - - -
\subsection{Datenerhebung}

% ..................................................................
\subsubsection{Wann erfolgt(e) die Erhebung bzw. Erstellung der Daten?}

Die Sammlung der Daten erfolgte über gesamten Bearbeitungszeitraum der Masterarbeit hinweg, d.\,h.\ vom 16.02.2023 bis 14.06.2023.

% ..................................................................
\subsubsection{Wann erfolgt(e) die Datenbereinigung/-aufbereitung bzw. Datenanalyse?}

\begin{enumerate}[(1)]
  \item
    Bei jedem Fund einer wissenschaftlichen Arbeit wurden die bibliographischen Metadaten von der entsprechenden Plattform heruntergeladen
    und eine Masterdatei importiert (bzw.\ von Hand eingetragen, falls nötig). Die Daten wurden bei jedem Import sofort geprüft und ggf.\
    korrigiert. Eine weitere Datenbereinigung oder -aufbereitung ist nicht erforderlich.
  \item
    Die elektronisch vorliegenden wissenschaftlichen Arbeiten, von denen abzusehen war, dass sie im Bearbeitungszeitraum der Masterarbeit
    länger oder mehrfach konsultiert werden mussten, wurden als PDF-Datei gespeichert.
    Eine Datenbereinigung oder -aufbereitung scheint hier nicht sinnvoll.
  \item
    Wie (1); hier wurden alle Metadaten von Hand eingegeben.
  \item
    Die statistischen Daten der Datenquellen wurden am 09.06.2023 in einer Datei gebündelt.
\end{enumerate}

% - - - - - - - - - - - - - - - - - - - - - - - - - - - - - - - - -
\subsection{Datengröße}

% ..................................................................
\subsubsection{Was ist die tatsächliche oder erwartete Größe der Daten(typen)?}

Stand 09.06.2023:
%
\begin{enumerate}[(1)]
  \item
    115 kB
  \item
    412,8 MB
  \item
    in (1) enthalten
  \item
    1,4 kB
\end{enumerate}

\goodbreak
% - - - - - - - - - - - - - - - - - - - - - - - - - - - - - - - - -
\subsection{Formate}

% ..................................................................
\subsubsection{In welchen Formaten liegen die Daten vor?}
%4 Vgl. z. B. DROID zur Format-Erkennung, http://digital-preservation.github.io/droid/

\begin{enumerate}[(1)]
  \item
    .bib (BibTeX)
  \item
    .pdf (Portable Document Format)
  \item
    siehe (1)
  \item
    .md (Markdown)
\end{enumerate}


*** TBC ***

\todo[inline]{finish DMP}

%3.4 Werkzeuge
%3.4.1 Welche Instrumente, Software, Technologien oder Verfahren werden zur Erzeugung, Erfassung, Bereinigung, Analyse und/oder Visualisierung der Daten genutzt? Bitte (falls möglich) mit Versionsnummer und Referenz in Form einer Adresse jeweils angeben.
%3.4.2 Welche Software, Verfahren oder Technologien sind notwendig, um die Daten zu nutzen?
%3.4 Versionierung
%3.4.1 Werden verschiedene Versionen der Daten erzeugt (z. B. durch verschiedene Weiterbearbeitungsprozesse bzw. Bereinigung von Daten)?
%4. Datennutzung
%4.1 Datenorganisation
%4.1.1 Gibt es eine Strategie zur Benennung der Daten? Wenn ja, bitte skizzieren Sie sie kurz.
%4.2 Datenspeicherung und -sicherheit
%4.2.1 Wer darf (zukünftig) auf die Daten zugreifen?
%4.2.2 Wie und wie oft werden Backups der Daten erstellt?
%  4.3 Interoperabilität
%4.3.1 Sind die Datenformate im Sinne der FAIR-Prinzipien interoperabel, d.h. geeignet für den Datenaustausch und die Nachnutzung zwischen bzw. von unterschiedlichen Forschenden, Institutionen, Organisationen und Ländern?
%4.4 Weitergabe und Veröffentlichung
%4.4.1 Ist es geplant, die Daten nach Abgabe der Abschlussarbeit zu veröffentlichen oder zu teilen?
%4.4.2 Wenn nicht, skizzieren Sie kurz rechtliche und/oder vertragliche Gründe und freiwillige Einschränkungen.
%4.4.3 Wenn ja, unter welchen Nutzungsbedingungen oder welcher Lizenz sollen die Daten veröffentlicht bzw. geteilt werden?
%4.5 Qualitätssicherung
%4.5.1 Welche Maßnahmen zur Qualitätssicherung (z. B. Plausibilitätsprüfung von Datenwerten) werden für die Daten ergriffen?
%4.6 Datenintegration
%4.6.1 Falls Daten aus verschiedenen Quellen (z. B. Anpassung Skalierung, Zeiträume, Ortsangaben) integriert werden, wie wird dies gewährleistet?
%5. Metadaten und Referenzierung
%5.1 Metadaten
%5.1.1 Welche Informationen sind für Außenstehende notwendig, um die Daten zu verstehen (d. h. ihre Erhebung bzw. Entstehung, Analyse sowie die auf ihrer Basis gewonnenen Forschungsergebnisse nachvollziehen) und nachnutzen zu können?
%5.1.2 Welche Standards, Ontologien, Klassifikationen etc. werden zur Beschreibung der Daten und Kontextinformation genutzt?
%6. Rechtliche und ethische Fragen
%6.1 Personenbezogene Daten
%6.1.1 Enthalten die Daten personenbezogene Informationen? 6.2 Sensible Daten
%Datum der letzten Aktualisierung 08.12.2021
%6.2.1 Enthalten die Forschungsdaten besondere Kategorien personenbezogener Daten nach Artikel 9 der DSGVO (“Angaben über die rassische und ethnische Herkunft, politische Meinungen, religiöse oder philosophische Überzeugungen, Gewerkschaftszugehörigkeit, Gesundheit oder Sexualleben”)5?
%6.2.2 Werden die Daten anonymisiert oder pseudonymisiert?
%6.2.3 Haben Sie eine "informierte Einwilligung" der Betroffenen eingeholt? Fügen Sie bitte ein Template der Einverständniserklärung als Anlage bei.
%6.2.4 Wenn keine "informierte Einwilligung" eingeholt wird, begründen Sie dies bitte.
%6.2.5 Wo und wie sind die "informierten Einwilligungen" abgelegt?
%6.2.6 Bis wann werden die (un-anonymisierten bzw. un-pseudonymisierten) Originaldaten spätestens sicher vernichtet?
%6.3 Urheber- oder verwandte Schutzrechte
%6.3.1 Werden Daten genutzt und/oder erstellt, die durch Urheber- oder verwandte Schutzrechte geschützt sind?
%7. Speicherung und Langzeitarchivierung
%7.1 Wo werden die Daten (einschließlich Metadaten, Dokumentation und ggf. relevantem Code bzw. relevanter Software) während Phase der Erarbeitung der Abschlussarbeit gespeichert?
%7.2 Wo werden die Daten (einschließlich Metadaten, Dokumentation und ggf. relevantem Code bzw. relevanter Software) nach dem Ende der Abschlussarbeit gespeichert bzw. archiviert6?
%7.3 Handelt es sich dabei um ein zertifiziertes Repositorium oder Datenzentrum (z.B. durch das CoreTrustSeal7, nestor-Siegel8 oder ISO 163639)? (Wurden mehrere Langzeitarchivierungsoptionen ausgewählt, kann die Frage bejaht werden, wenn dies auf mindestens eine der Optionen zutrifft).

