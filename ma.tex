% !TeX spellcheck = en_GB
%\documentclass[11pt,DIV12,a4paper,BCOR15mm]{scrbook}
%\documentclass[11pt,DIV=11,a4paper,bibliography=leveldown]{scrreprt}
\documentclass[10pt,DIV=11,a4paper]{scrartcl}
\usepackage{ma}

\title{Modelling and Automated Retrieval of Provenance Relationships}
\author{Thomas Schneider}

\begin{document}

  \maketitle

  % =================================================================
  \section{Modelling}
  
  ... FRBR ...
  
  % ------------------------------------------------------------------
  \subsection{Provenance Relationship Networks}
  
  The definition of our central notion of a provenance relationship network
  is based on the abstract data structure of a directed graph with labelled edges,
  which is a standard notion in computer science and discrete mathematics.
  \todo{citation}
  We fix a set $N$ of \emph{node names} and a set $R$ of \emph{relation names}.
  A \emph{directed edge-labelled graph over $(N,R)$} is a triple $G = (V,E,\Lmc)$,
  where
  %
  \begin{itemize}
    \item
      $V \subseteq N$ is a set, whose members are called \emph{vertices} or \emph{nodes};
    \item 
      $E \subseteq V \times V$ is a set of pairs of nodes, whose members are called \emph{edges};
    \item
      $\Lmc : E \to 2^R$ is a function that assigns to each edge a non-empty set of relation names,
      called the \emph{labels} of that edge; we call \Lmc a \emph{labelling function}.
  \end{itemize}
  %
  \dots\todo{example}
  
  A \emph{provenance relationship network} is a directed edge-labelled graph over $(N,R)$,
  where $N$ ... and $R$ ... \todo{continue}
  
     


%  % =================================================================
%  \bibliographystyle{babplain}
%  \bibliography{expose}


\end{document}
