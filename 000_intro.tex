% !TeX spellcheck = en_GB

% =================================================================
\chapter{Introduction}
\label{chap:intro}

\begin{itemize}
  \item
    see Exposé
  \item
    Example queries:
    %
    \begin{enumerate}
      \item[\exaquery{1}]
        Who read work $W$, in which manifestation and in which year?
      \item[\exaquery{2}]
        Which copies of work $W$ were passed from one of its owners to a collaborator (or a student)?
      \item[\exaquery{3}]
        What are the relationships between the recipients of work $W$
        (or of manifestation $M$ of $W$ or of copy $C$ of $W$, respectively)?
    \end{enumerate}
\end{itemize}
%
To illustrate these abstract queries and possible approaches to answering them, let us fix work $W$ to be the seminal work \emph{De revolutionibus orbium coelestium}
(short: \emph{De revolutionibus}; English translation: \emph{On the Revolutions of the Heavenly Spheres}) by the astronomer Nicolaus Copernicus (1473–1543) \todo{citation}.
We consider the following more precise variant of \exaquery{2}:
%
\begin{enumerate}
  \item[\exaquery{2$'$}]
    Which copies of \emph{De revolutionibus} were owned by scientists who passed them on to a student?
\end{enumerate}
%
If a researcher wants to answer this question, they could proceed as follows: first, they find copies of \emph{De revolutionibus} 
in online catalogues of libraries and library networks. For each such copy, they inspect the provenance entries
that name owners who were people (not corporations). They will then have to find those names in databases such as
authority files or Wikidata and, for each entry, explore the specified profession (``scientist'') and relationships to other people (``student'').

For example, the online catalogue of the Gotha Research Library of the University of Erfurt (Forschungsbibliothek Gotha) lists two printed copies
of \emph{De revolutionibus}.%
\footnote{%
  \url{https://opac.uni-erfurt.de/DB=1/CMD?ACT=SRCHA&IKT=1016&SRT=YOP&TRM=tit+de+revolutionibus+and+per+kopernikus+and+jah+15**+and+bbg+a*}%
}
One of these copies bears the signature \sig{Druck~4°~00466}, and its provenance entries lists three previous owners that were scientists:%
\footnote{%
  \url{https://opac.uni-erfurt.de/DB=1/XMLPRS=N/PPN?PPN=567506266}%
}
%
*****

\todo[inline]{!! ``scientist'' isn't given in GND for, e.g., Thau -- only more specific relations such as ``mathematician'', ``astronomer'', ``lawyer''. $\Rightarrow$ we need QA with taxonomies/ontologies !!}

(OPAC, GND \todo{link}) .....

This manual process is cumbersome, laborious, and prone to errors and omissions: The first step
requires a manual expert search in a collection of catalogues. These catalogues need to be selected in advance,
and each catalogue may come with its own search functionality and syntax. \todo{give examples of differences?}
Traversing through all copies and following each relevant provenance entry in each copy requires
a lot of manual work. \todo{rephrase using ``combinatorics'' or ``combinatorial''}
Finally, it is not clear what an effective and efficient way to ``explore'' the relationships would be:
while it is easy to find direct relationships such as ``person $P_1$ is student of person $P_2$'' in the view for a person's entry
in databases such as GND or Wikidata, there are relationships that cannot be discovered easily by hand,
e.g., ``$P_1$ and $P_2$ are students of the same scholar''.
In order to discover such more complex and indirect relationships, automated support is essential.



or 

...
%
\begin{itemize}
  \item
    Give concrete examples for \exaquery{1}--\exaquery{3}, with data sources and answers; explain how these queries differ from each other.
  \item
    In this thesis, focus on these three types of queries.
    This is a deliberate restriction of what one can generally ask – scope of master's thesis!
\end{itemize}



\dots

